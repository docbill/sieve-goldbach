% Copyright (C) 2025 Bill C. Riemers
% SPDX-License-Identifier: CC-BY-4.0
%
% Licensed under the Creative Commons Attribution 4.0 International License.
% You may obtain a copy of the License at:
%     https://creativecommons.org/licenses/by/4.0/
%
% You are free to:
%   - Share — copy and redistribute the material in any medium or format
%   - Adapt — remix, transform, and build upon the material for any purpose
% Under the following terms:
%   - Attribution — You must give appropriate credit, provide a link to the license,
%     and indicate if changes were made.

\documentclass[11pt]{article}
\usepackage{amsmath,amssymb,amsthm,fullpage,mathtools,microtype}
\setlength{\parskip}{0.8em}
\setlength{\parindent}{0pt}
\usepackage[utf8]{inputenc}
\usepackage{geometry}
\usepackage{hyperref}
\usepackage{graphicx}
\usepackage{float}
\usepackage{needspace}
\usepackage{url}
\usepackage{longtable,booktabs}
\usepackage[T1]{fontenc}
\usepackage{lmodern}
\renewcommand{\ttdefault}{lmtt}
\usepackage{microtype}
\usepackage{mathtools}
\usepackage{siunitx}
\usepackage{pgfplots}
\usepackage{pgfplotstable}
\pgfplotsset{compat=1.18}
\pgfplotstableset{empty cells with={nan}} % treat empty CSV cells as NaN

\DeclarePairedDelimiter{\card}{\lvert}{\rvert}

\microtypecontext{expansion=default} % or, e.g., disable for tt:
\SetExpansion{encoding = *, family = tt*}{}

\newtheoremstyle{inline}% for remarks/notes
  {}{}{\normalfont}{}{\itshape}{.}{ }{}
  
\newtheoremstyle{break}  % Name
  {1ex}                  % Space above
  {1ex}                  % Space below
  {\normalfont}          % Body font
  {}                     % Indent
  {\bfseries}            % Theorem head font
  {.}                    % Punctuation after theorem head
  {\newline}             % Space after theorem head (THIS FORCES LINE BREAK)
  {}                     % Theorem head spec

\theoremstyle{inline}
\newtheorem*{remark}{Remark}

\theoremstyle{break}
\newtheorem{lemma}{Lemma}

\makeatletter
\renewenvironment{proof}[1][\proofname]{%
  \par\pushQED{\qed}%
  \normalfont \topsep6\p@\@plus6\p@\relax
  \trivlist
  \item[\hskip\labelsep
        \itshape
    #1\@addpunct{.}]\mbox{}\\  % This line forces the break
}{%
  \popQED\endtrivlist\@endpefalse
}
\makeatother

\theoremstyle{break}
\newtheorem*{conclusion}{Conclusion}

\theoremstyle{break}
\newtheorem{theorem}{Theorem}

\theoremstyle{break}
\newtheorem{proposition}{Proposition}

\theoremstyle{break}
\newtheorem{conjecture}{Conjecture}

\theoremstyle{break}
\newtheorem{corollary}{Corollary}

\theoremstyle{break}
\newtheorem{definition}{Definition}

\theoremstyle{inline}
\newtheorem*{note}{Note}

\newcommand{\sieveprod}{\mathop{\prod\nolimits}}
\newcommand{\abs}[1]{\left|#1\right|}

\newcommand{\xPreMertens}{5416} % last data point with delta < 0.
\newcommand{\xMertens}{6353} % threshold for Lemma~\ref{lem:explicit-mertens}
\newcommand{\xNasymStar}{8777}  % = \nasymstar
\newcommand{\xNzeroStat}{71633} % lowest minimum past n_5%
\newcommand{\xEtaStat}{0.2693} % delta at \xNzeroStat
\newcommand{\xMinGlobal}{7219}       % argmin location of the empirical gap over the verified range
\newcommand{\xDeltaGlobal}{0.0526}   % min gap value at \xMinGlobal
\newcommand{\xDeltaMertens}{0.1149}  % gap at the explicit Mertens threshold \xMertens
\newcommand{\xKemKem}{2.152} % Kem^2 = 
\newcommand{\xSecondLambdaMinLimit}{1.3\cdot 10^{-2}} % Decade 8, LambdaAvg2ndAbsMax value
\newcommand{\xSecondLambdaMaxLimit}{2.1\cdot 10^{-3}} % Decade 8, LambdaAvg2ndAbsMax value
\newcommand{\xSecondLambdaAvgLimit}{2.2\cdot 10^{-4}} % Decade 8, LambdaAvg2ndAbsAvg value
\newcommand{\xLambdaMinLimit}{1.8\cdot 10^{-2}} % Decade 8, LambdaAvg2ndAbsMax value
\newcommand{\xLambdaMaxLimit}{3.9\cdot 10^{-3}} % Decade 8, LambdaAvg2ndAbsMax value
\newcommand{\xLambdaAvgLimit}{3.2\cdot 10^{-4}} % Decade 8, LambdaAvg2ndAbsAvg value

% tiny tags
\newcommand{\tavg}{{\scriptscriptstyle\mathrm{avg}}}
\newcommand{\tpairs}{{\scriptscriptstyle\mathrm{pairs}}}
\newcommand{\tsem}{{\scriptscriptstyle\mathrm{sem}}}
\newcommand{\twin}{{\scriptscriptstyle\mathrm{win}}}
\newcommand{\tref}{{\scriptscriptstyle\mathrm{ref}}}
\newcommand{\tana}{{\scriptscriptstyle\mathrm{analytical}}}

% measured vs predicted
\newcommand{\Cmeas}{C}              % measured (exact)
\newcommand{\Cpred}{\mathring{C}}   % predicted/model, robust (no double-^ issue)
\newcommand{\Cminus}{\mathring{C}_{-}}   % predicted/model, robust (no double-^ issue)
\newcommand{\Nmeas}{n}              % measured (exact)
\newcommand{\Npred}{\mathring{n}}   % predicted/model, robust (no double-^ issue)
\newcommand{\Ssem}{\mathcal{S}_\tsem}
\newcommand{\SGB}{\mathcal{S}_{\scriptscriptstyle\mathrm{GB}}}
\newcommand{\Bwin}{\mathcal{B}_\twin}
\newcommand{\Bref}{\mathcal{B}_\tref}
\newcommand{\betacal}{\beta_{\mathrm{eval}}}
\newcommand{\Ipar}{I^{\mathrm{par}}}
\newcommand{\Ngeom}{n_\mathrm{geom}}
\newcommand{\HLCorr}{\mathcal{H}}
\newcommand{\CminusProduct}{C_{-}}
\newcommand{\CminusProductAlpha}[2]{\mathcal{C}_{-,#2}\!\left(#1\right)}
\newcommand{\CminusAsymp}{C_{-}^{\mathrm{asymp}}}
\newcommand{\CminusUnknown}{\textbf{fix me}}
\newcommand{\Kem}{K_{\mathrm{EM}}}              % = 4 e^{-\gamma} C_2 (Euler–Mertens/product)
\newcommand{\Kminor}{K_{\mathfrak m}}           % minor-arc implied constant
\newcommand{\Szero}{S_0}                        % lower bound for singular series (e.g. 2 C_2)
\newcommand{\nprodstar}{n_{\ast}}  % threshold for product-form bound
\newcommand{\nasymstar}{n_{\ast}^{\mathrm{asym}}}  % threshold for asymptotic surrogate
\newcommand{\kappaprod}{\kappa}    % infimum of product constant on tail
% optional compact args
\newcommand{\NI}{(n;I)}

% \FigLambdaAbsSignSplit{csv}{xcol}{ycol}{title}
\newcommand{\FigLambdaAbsSignSplit}[6]{%
  \begingroup
  \edef\CSVPATH{\detokenize{#1}}%
  \pgfplotstableread[col sep=comma]{\CSVPATH}\datatable
  \begin{tikzpicture}
  \begin{axis}[
    width=\textwidth, height=0.62\textwidth,
    xmode=log, ymode=log,
    xlabel={#4 (log scale)}, ylabel={#5 (log scale)},
    title={#6},
    grid=both,
    legend style={at={(0.98,0.98)}, anchor=north east},
    unbounded coords=discard,       % drop NaN/Inf
    filter discard warning=false,
    tick align=outside,
  ]

    % ------- Lambda >= 0 (plot as |Lambda|) -------
    \addplot+[only marks, mark=*, mark size=1.2pt]
      table[
        x={#2},
        y expr=abs(\thisrow{#3}),
        restrict expr to domain={\thisrow{#3}}{0:inf}, % <— gate on the original y
      ] {\datatable};
    \addlegendentry{\(\Lambda \ge 0\)}

    % ------- Lambda < 0 (plot as |Lambda|) -------
    \addplot+[only marks, mark=*, mark size=1.2pt]
      table[
        x={#2},
        y expr=abs(\thisrow{#3}),
        restrict expr to domain={\thisrow{#3}}{-inf:0}, % <— gate on the original y
      ] {\datatable};
    \addlegendentry{\(\Lambda < 0\)}

  \end{axis}
  \end{tikzpicture}%
  \endgroup
}

\geometry{margin=1in}

\title{A Sieve-Theoretic Reformulation of the Goldbach Conjecture}
\author{Bill C Riemers}
\date{\today}

\begin{document}

\maketitle

\begin{abstract}
A windowed sieve framework for Goldbach is presented based on the quadratic form
\begin{equation}
Q(n,m)\;=\;(n-m)(n+m),
\end{equation}
which centers analysis at the midpoint \(n\) and treats offsets \(m\) symmetrically. This formulation interfaces with Eratosthenes–type sieves below the prime–forcing cutoff, avoids the classical parity obstruction, and yields certified lower bounds as a product of conservative Euler factors.

Unconditionally, a \emph{certified analytic lower bound} on the windowed Goldbach count \(\mathcal{G}(n;M)\) (with \(M=\lfloor n/2\rfloor\)) is proved via explicit Euler–Mertens products \cite{Mertens1874,MontgomeryVaughan2007,RosserSchoenfeld1962,Dusart2010}, valid for all \(n\ge \xMertens\). A \emph{rescaling lemma} shows this bound holds uniformly for every smaller window \(M=\alpha n\) with \(0<\alpha\le \tfrac12\), and a monotonicity corollary extends it to larger windows by set inclusion.

Computationally, windowed Goldbach pairs are exhaustively enumerated for every even \(2n<2\cdot 10^8\). 
Across seven decades, the normalized deviations from the parameter-free Hardy–Littlewood baseline (HL–A) \cite{HardyLittlewood1923} decrease steadily. 
In the seventh decade (\(10^7 \le n < 10^8\)) the \emph{raw decade maxima} already satisfying
\begin{equation}
\lvert\Lambda_{\min}\rvert\le \xLambdaMinLimit,\qquad
\lvert\Lambda_{\max}\rvert\le \xLambdaMaxLimit,\qquad
\lvert\Lambda_{\tavg}\rvert\le \xLambdaAvgLimit,
\end{equation}
providing large-scale validation of HL–A in this windowed setting.
As a robustness check, for each decade and for each metric, the \emph{second-largest} absolute deviation is \emph{strictly decreasing} across the seven decades, as expected when occasional outliers persist; the conclusions are unchanged under this pruning.
Observed blockwise maxima align with the singular-series structure: peaks occur when \(n\) is a multiple of \emph{half a primorial} (i.e., \(p\#/2\)), yielding the expected primorial plateaus.

A \emph{conditional reduction} is also established: assuming a short-interval Bombieri–Vinogradov hypothesis (SI–BV\(_\theta\) with \(\theta>\tfrac12\), strictly weaker than the full Hardy–Littlewood asymptotic), the minor-arc error is uniformly dominated and \(R_2(N)>0\) holds for all sufficiently large even \(N\). Combined with exhaustive verification up to \(2\nprodstar\), this yields Goldbach for all even integers \(>2\).

In summary, contributions are: (i) a certified sieve–theoretic lower bound with explicit constants, uniform in the window; (ii) the first broad, decade-by-decade numerical validation of HL–A at sub-percent scale in this framework; (iii) a structural explanation of maxima via singular-series (odd-primorial) plateaus; and (iv) a sharp reduction of the remaining analytic task to short-interval equidistribution of primes.
\end{abstract}

\needspace{10\baselineskip}

\section{Introduction}

\subsection{Motivation}
The Goldbach Conjecture, the Twin Prime Conjecture, and Polignac’s Conjecture each address the distribution of prime pairs in different settings.  
A natural generalization emerges from considering these problems together: along suitable arithmetic or algebraic paths in the \( (n,m) \)-grid, prime pairs appear with a density consistent with the heuristic \( \#\mathcal{S}_N / \log^2 N \).  
Formulated precisely, this leads to the following working conjecture.

\begin{conjecture}[General Prime-Pair Density (motivating observation)]\label{conj:gppd}
Let \( (p,q) \) be an odd prime pair satisfying
\begin{equation}
p+q = 2n,\quad p-q = 2m,\quad pq = n^2 - m^2,\quad |m| \le n,
\end{equation}
with \( (m,n) \) constrained to a fixed line or other low-degree polynomial path in the \( (m,n) \)-grid.  
Then there exists \( N_0 \) such that for all \( N \ge N_0 \), every interval of length \( O(N) \) odd integers along that path contains at least \( O(N / \log^2 N) \) such prime pairs.
\end{conjecture}

The statement above is given in a simplified form, with notation consistent throughout this paper.  
A more detailed formulation, including explicit bounds in terms of \( \Cmeas_{\min} \) and \( \Cmeas_{\max} \) and the role of the reference sieve factor \( \Bref \), appears in Conjecture~\ref{conj:uniform-C} and the definition of admissibility (Definition~\ref{def:admissible}).

While unproven, this conjecture provides a coherent framework in which the problems above are special cases.  
In what follows, the Goldbach setting becomes the concrete instance for developing and testing the sieve-theoretic methods,  
before considering broader applications.

\begin{remark}[Scope]
Conjecture~\ref{conj:gppd} motivates the windowed sieve setup only; no theorem, lemma, or corollary in this paper depends on it. Unconditional results use sieve bounds and Euler–Mertens products; the only conditional input appears in Corollary~\ref{cor:SI-BV}.
\end{remark}

\needspace{10\baselineskip}

\subsection{Contributions and Reduction Overview.}
To begin, the certified bounds, the large–scale validation of the heuristic baseline, the window–scalability results, the structure of extrema, and the final conditional reduction to short–interval equidistribution, contributions are summarized.

\begin{enumerate}
  \item \textbf{Calibrated Sieve–Heuristic and Per–Term Normalization.}
  Formalizing the sieve–heuristic baseline on the structured family \(Q_m=n^2-m^2\), introduces a per–term normalization \( C_\star(n;I) \) that is consistent with Conjecture~A and yields asymptotic predictions proportional to \( \SGB(2n) \). This fixes units and removes binning artefacts for all subsequent comparisons.

  \item \textbf{Statistical Convergence of Normalized Deviations (validation of HL–A).}
  Normalized deviations between measured and predicted pair counts are defined as an evaluation across seven decades up to \(2n=2\cdot 10^8\). By the final decade the deviations fall below
  \( \xLambdaMinLimit \) (minimum), \( \xLambdaMaxLimit \) (maximum), and \( \xLambdaAvgLimit \) (average), with monotone decay across decades, providing the first large–data validation that windowed counts converge to the HL–A baseline, justifying extrapolation beyond the tested range.

  \item \textbf{Certified Analytic (Shifted–Product) Lower Bound (Pairs).}
  By Lemma~\ref{lem:analytic-lower-bound-reduction}, in the extremal out–of–sync case so that
  \begin{equation}\label{eq:shifted-product-certified-2}
    \mathcal{G}(n;M)\ \ge\ \frac{n}{2}\,
    \prod_{\substack{p>2\\ p\le \sqrt{n}}}\!\Bigl(1-\frac{1}{p-1}\Bigr)\,
    \prod_{\substack{p>2\\ p\le \sqrt{\tfrac{3n}{2}}}}\!\Bigl(1-\frac{1}{p-1}\Bigr),
  \end{equation}
  where \( M=\lfloor n/2\rfloor \). Using the explicit Mertens enclosure
  \cite{RosserSchoenfeld1962, Dusart2010, HardyLittlewood1923, MontgomeryVaughan2007}
  \begin{equation}\label{eq:mertens-enclosure-2}
    \prod_{p\le \sqrt{x}}\Bigl(1-\tfrac{1}{p-1}\Bigr)\ \sim\ \frac{\Kem}{\log x},
    \qquad \Kem=4e^{-\gamma}C_2,
  \end{equation}
  yields, for large \(n\),
  \begin{equation}\label{eq:asymp-certified-2}
    \mathcal{G}(n;M)\ \gtrsim\ \frac{\Kem^2\,M}{\log n\;\log(\tfrac{3n}{2})}.
  \end{equation}
  On the tested range this specializes to the concrete inequality
  \begin{equation}\label{eq:numeric-certified-2}
    \mathcal{G}(n;M)\ \ge\ \frac{2.1518\,M}{\log^2 n}.
  \end{equation}
  The certification \eqref{eq:shifted-product-certified-2}–\eqref{eq:numeric-certified-2} is unconditional: it does not invoke HL–A, \( \SGB \), or \( \betacal \).

  \item \textbf{Uniform Window Scalability and Monotone Extension.}
  The certified lower bound extends \emph{uniformly in the window size} for every \( \alpha\in(0,\tfrac12] \) by Lemma~\ref{lem:alpha-rescale}:
  \begin{equation}\label{eq:alpha-certified-bridge-2}
    \mathcal{G}(n;\,\alpha n)\ \ge\ \frac{\CminusProductAlpha{\alpha}{n}}{\log^2 n}\,(\alpha n),
  \end{equation}
  with the natural right–edge cutoff \( \sqrt{n+\alpha n} \) inside \( \CminusProductAlpha{\alpha}{n} \).
  By set inclusion, the count is monotone in the window, so for all \( \alpha\in[\tfrac12,1) \)
  \begin{equation}\label{eq:alpha-superset-bridge-2}
    \mathcal{G}(n;\,\alpha n)\ \ge\ \mathcal{G}\!\left(n;\,\tfrac12 n\right),
  \end{equation}
  as recorded in Corollary~\ref{cor:alpha-superset}.

  \item \textbf{Extrema Structured by the Singular Series (Primorial Plateaus).}
  Writing
  \begin{equation}\label{eq:sing-series-bridge-2}
    \mathfrak S(2n)\;=\;2C_2\prod_{\substack{p\mid n\\ p\ge 3}}\frac{p-1}{p-2},
  \end{equation}
  Proposition~\ref{prop:primorial-plateau} shows that \( \mathfrak S(2n) \); and hence the normalized \( C \)-statistic; achieves record and local plateaus when the odd part of \(n\) is divisible by the odd primorial \( P_y=\prod_{3\le p\le p_y}p \); in particular, on \( [P_y,\,p_{y+1}P_y) \) the maxima occur precisely at multiples of \( P_y \).

  \item \textbf{Pointwise Positivity Under Short–Interval Equidistribution (Reduction).}
  Assuming the short–interval Bombieri–Vinogradov hypothesis \eqref{eq:SIBV}, Corollary~\ref{cor:SI-BV} yields an explicit \( N_0 \) such that
  \begin{equation}\label{eq:pointwise-positivity-2}
    R_2(N)\ >\ 0\qquad\text{for every even }N\ge N_0.
  \end{equation}
  Together with our exhaustive verification up to \( 2\nprodstar \), this reduces Goldbach for all even \( N\ge 4 \) to \eqref{eq:SIBV} on a tail interval; no Hardy–Littlewood asymptotic is assumed.

  \item \textbf{Bridging Computation and Analytic Bounds (no gaps).}
  Explicit computation verifies all even numbers up to \(2n=2\nprodstar\).  The certified lower bound applies uniformly for all \( n\ge \xMertens \) (cf.\ Fig.~\ref{fig:lower-analytic-bound-comparison}), and by \eqref{eq:alpha-certified-bridge-2}–\eqref{eq:alpha-superset-bridge-2} the same holds for all window sizes under consideration. Hence the verified initial segment and the certified asymptotic regime overlap without gaps; under \eqref{eq:SIBV} the pointwise positivity \eqref{eq:pointwise-positivity-2} completes the reduction.
\end{enumerate}

\begin{remark}
Use of the singular series and the \(  \frac{n}{log^2 n} \) scale follows the classical circle-method heuristic of Hardy–Littlewood.\cite{HardyLittlewood1923} This paper does not claim novelty for these ingredients. The contributions here are (i) a per-term, windowed adaptation tailored to \( Q_m = n^2 - m^2 \)  with explicit calibration via \( \Bwin \); (ii) a certified sieve lower bound in this setting; and (iii) a statistical protocol that tests the parameter-free curve \( 2\SGB(2n) \) against data across decades. All statements relying on Hardy-Littlewood Conjecture A (HL-A) are clearly labeled as model-based; certified results are unconditional.
\end{remark}

\needspace{10\baselineskip}

\subsection{Readers’ Guide}
Section~\ref{sec:framework} sets up the sieve–heuristic framework: the quadratic form \(Q(n,m)\), the window \(M(n)\), and the HL–A baseline and normalizations used throughout. 
Section~\ref{sec:stats} presents the computational study up to \(2n<2\cdot 10^8\), including decade-wise deviations and the primorial plateaus.
Section~\ref{sec:sgb} contains the core sieve-theoretic results: the reduction lemma (Sec.~\ref{sec:reduction}), the certified lower bound (Thm.~\ref{thm:main}), its conditional corollary under short-interval Bombieri-Vinogradov equidistribution hypothesis (Sec.~\ref{sec:SI-BV}), and the primorial maxima proposition (Sec.~\ref{sec:primorial-prop}).
Appendices collect technical enclosures and window rescaling.

\needspace{10\baselineskip}

\section{Sieve--Heuristic Framework}\label{sec:framework}

\begin{remark}[Terminology: ``model'' vs.\ ``theorem'']
Throughout, ``model'' refers to the sieve–heuristic framework combining the local
factors \( \prod_{p\ge3}\!\bigl(1-\tfrac{2}{p}\bigr) \), the semiprime singular series
\( \SGB(2n) \), and the evaluation calibration \( \betacal(I) \), yielding
predicted quantities such as \( \Cpred \) and \( C_\star \). These are
\emph{model-based predictions} (heuristic expectations), not theorems.
Measured quantities (e.g.\ \( \Cmeas \)) are exact given the data.

The rigorous component is developed in the Sieve–Theoretic section, where a certified analytic lower bound is established that supports the later arguments.
Other relationships stated here (e.g.\ \( C_\star(n;I)\to \betacal(I)\,\SGB(2n) \))
are presented to convey heuristic understanding and are not required
for the rigorous result itself.
\end{remark}
Starting with the sequence:

\begin{equation}
Q_m = n^2 - m^2 = (n - m)(n + m)
\end{equation}

Let \( S_n = \{ p \in \mathbb{P} \mid p < \sqrt{n} \} \) be the set of all primes less than \( \sqrt{n} \).

A sieve is constructed over the range \( m \in [1, M] \), for some \( M = O(n) \), to eliminate values of \( m \) for which \( Q_m = (n - m)(n + m) \) has small prime divisors. Initially, all \( m \) in the range are candidates, and those for which \( Q_m \equiv 0 \mod p \) for any \( p \in S_{\sqrt{N+M}} \) are iteratively removed. This process is equivalent to eliminating values of \( m \) lying in specific residue classes modulo each small prime, as described by standard sieve methods (see~\cite{HalberstamRichert1974,IwaniecKowalski2004,FriedlanderIwaniec2010}).  \footnote{A related idea appears in work by Song~\cite{Song2008}, who proposed a sieve partitioning method to preserve minimal composite structure when analyzing Goldbach pairs. While his approach differs significantly in formulation and does not employ the multiplicative structure used herein, it reflects a similar intuition; that full prime sieving is not always necessary.}\footnote{Recasting the Goldbach condition in terms of the quadratic form \(Q(n,m)=(n-m)(n+m)\) does not alter the underlying problem, but it provides a parametrization in which windowing and sieve reduction steps are expressed more cleanly, and where the semiprime structure is explicit.}

For each \( p \in S_{\sqrt{N+M}} \), note:

\begin{equation}
Q_m \equiv 0 \mod p \iff (n - m)(n + m) \equiv 0 \mod p
\end{equation}

which implies:

\begin{equation}
m^2 \equiv n^2 \mod p
\end{equation}

A convenient \emph{reference sieve product} that captures the idealized effect of
eliminating two residue classes per odd semi-prime candidate \( Q_m \) in the absence of any
alignment or discretization artefacts, and is defined as follows:

\begin{definition}[Reference Sieve Product]
\label{def:Bref}
Let \( \mathcal{P} \) denote the set of odd primes up to some bound \( y \).
Following the analysis of Iwaniec–Kowalski \cite{IwaniecKowalski2004}, the reduction for odd semiprimes is expressed by the sieve product
\begin{equation}
\Bref(y)
    := \prod_{\substack{3 \le p \le y \\ p \in \mathcal{P}}}
       \left( 1 - \frac{2}{p} \right),
\end{equation}
since the congruence \( n^2 - m^2 \equiv 0 \pmod{p} \) has exactly two solutions for each odd prime \( p \).

This product represents the multiplicative reduction factor in the
\emph{idealized} case where precisely two residue classes are eliminated
for every odd prime, with no further perturbations. The prime \( p = 2 \)
is omitted, as the halving from restricting to odd \( m \)-values is already
absorbed into the initial count \( M \).
\end{definition}

Before proceeding further with model definitions, it is important to define what is measured, so that functions using \( \Bref \) can be 
defined to allow an accurate parameter free comparison.

\begin{definition}[Empirical HL–Normalized Measurements (from semiprime survivors)]\label{semiprime-survivors}
Let
\begin{equation}
\Ipar := \{\, m\in I : n^2-m^2 \text{ is odd} \,\}
= \{\, m\in I : n+m \equiv 1 \pmod 2 \,\}.
\end{equation}

Far a window \(I\) with \(M:=\left|\Ipar\right|\).
For each \( m \in \Ipar \), set
\begin{equation}
y(n,m):=\bigl\lfloor \sqrt{\,n+|m|\,}\bigr\rfloor .
\end{equation}
Let
\begin{equation}
N_M(2n;I)\ :=\ \#\{\,m\in \Ipar\}:\ p\nmid (n^2-m^2)\ \text{for all }p\le y(n,m)\,\}
\end{equation}
be the number of surviving \emph{semiprimes} in \(I\).

Defining the measured (pairs-scale) constant:
\begin{equation}
\Cmeas(n;I)\ :=\ \frac{2\log^2 n}{M}\,N_M(2n;I).
\end{equation}
(Equivalently, the semiprime-scale version is
\(\Cmeas^{(\mathrm{sem})}(n;I):=\frac{\log^2 n}{M}N_M(2n;I)\) with
\(\Cmeas=2\,\Cmeas^{(\mathrm{sem})}\).)

For a decimal block \(B_{d,k}=[d\cdot10^k,(d+1)\cdot10^k)\),
\begin{equation}
\Nmeas_0:=\arg\min_{n\in B_{d,k}}\Cmeas(n;I),\qquad
\Nmeas_1:=\arg\max_{n\in B_{d,k}}\Cmeas(n;I),
\end{equation}
\begin{equation}
\Cmeas_{\min}(d,k):=\Cmeas(\Nmeas_0;I),\quad
\Cmeas_{\max}(d,k):=\Cmeas(\Nmeas_1;I),\quad
\Cmeas_{\tavg}(d,k):=\frac{1}{\#B_{d,k}}\sum_{n\in B_{d,k}}\Cmeas(n;I).
\end{equation}
\end{definition}

The \( \Bref \) gives us a good way to evaluate a probably of a single semiprime reduction.  However, what is really useful is a baseline of how many semiprimes to expect for a given value \( n \).  For this \( C_\star \) and related expressions are defined as follows:

\begin{definition}[Per-Term Window Baseline]\label{def:window-baseline}
Let \(I\subset\mathbb Z\setminus\{0\}\) be a finite window and
\(I^{\mathrm{par}}:=\{\,m\in I:\ n+m\equiv 1 \pmod 2\,\}\).
For each \(m\in I^{\mathrm{par}}\) set
\begin{equation}
y(n,m):=\bigl\lfloor\sqrt{\,n+|m|\,}\bigr\rfloor.
\end{equation}
Define the \emph{window baseline}
\begin{equation}
\Bwin(n;I):=\sum_{m\in I^{\mathrm{par}}}\ \prod_{\substack{3\le p\le y(n,m)\\ p\in\mathbb P}}
\Bigl(1-\frac{2}{p}\Bigr).
\end{equation}
Let \(C_2\) be the twin–prime constant and \(\kappa:=4e^{-2\gamma}C_2\).\cite{HardyLittlewood1923, MontgomeryVaughan2007}
Define the Goldbach singular series (pairs–scale)
\begin{equation}
S_{\mathrm{GB}}(2n):=2\,C_2\!\!\prod_{\substack{p\mid n\\ p\ge 3}}\frac{p-1}{p-2}.
\end{equation}

\emph{Heuristic counts on \(I\):}
\begin{equation}
\begin{aligned}
\mathbb{E}[\text{Goldbach representations (unordered)}]\ \approx\ S_{\mathrm{GB}}(2n)\,\Bwin(n;I), \\
\mathbb{E}[\text{Goldbach pairs (ordered)}]\ \approx\ 2\,S_{\mathrm{GB}}(2n)\,\Bwin(n;I).
\end{aligned}
\end{equation}

\emph{Per-term HL–normalized constant (baseline):}
\begin{equation}
C_\star(n;I)
\ :=\
\frac{1}{\kappa}\,\frac{\log^2 n}{\lvert I^{\mathrm{par}}\rvert}\,
S_{\mathrm{GB}}(2n)\,\Bwin(n;I).
\end{equation}
Introduce the evaluation calibration
\begin{equation}
\betacal(I)\ :=\ \lim_{n\to\infty}\frac{1}{\kappa}\,\frac{\log^2 n}{\lvert I^{\mathrm{par}}\rvert}\,\Bwin(n;I),
\end{equation}
so that \(C_\star(n;I)\to \betacal(I)\,S_{\mathrm{GB}}(2n)\) as \(n\to\infty\) with \(\lvert I\rvert=o(n)\).

\noindent\emph{Convention.} “Unordered” counts \(\{p,q\}\) once; “ordered” counts \((p,q)\) and \((q,p)\) separately, hence the extra factor \(2\).
\end{definition}

Next apply \( C_\star \) function in definitions that provides a clean way to define predicted to match the empirical measured values.

\begin{definition}[HL–A Normalized Predictions (Goldbach, Pairs)]
The windowed log effect is absorbed into the prediction via a Harding Littlewood Circle correction factor:
\begin{equation}
\HLCorr(n;I)\ :=\ \frac{\log^2 n}{\lvert \Ipar\rvert}\,
\sum_{m\in \Ipar}\frac{1}{\log(n-m)\,\log(n+m)}\,,
\end{equation}
for \(n\) and \(I\) such that \(n\pm m\ge 3\) for all \(m\in\Ipar\).

Fix a window \(I\subset\mathbb Z\setminus\{0\}\) with \(M:=\lvert\Ipar\rvert\).
Let \(C_\star(n;I)\) be the per-term HL–normalized constant (unordered scale).
Define the \emph{predicted (pairs-scale) constant} by
\begin{equation}
\Cpred(n;I)\ :=\ 2\,C_\star(n;I)\,\HLCorr(n;I).
\end{equation}

For decimal blocks
\begin{equation}
B_{d,k}:=\big[d\cdot10^k,\,(d+1)\cdot10^k\big),\qquad d\in\{1,\dots,9\},\ k\in\mathbb N,
\end{equation}
select extremizers by \(C_\star\) (equivalently by \(\Cpred/\HLCorr\)):
\begin{equation}
\begin{aligned}
\Npred_0 &:= \arg\min_{n\in B_{d,k}} \frac{\Cpred(n;I)}{\HLCorr(n;I)}, \\
\Npred_1 &:= \arg\max_{n\in B_{d,k}} \frac{\Cpred(n;I)}{\HLCorr(n;I)},
\end{aligned}
\qquad
\begin{aligned}
\Cpred_{\min}(d,k) &:= \Cpred(\Npred_0;I),\\
\Cpred_{\max}(d,k) &:= \Cpred(\Npred_1;I).
\end{aligned}
\end{equation}

For the block average, approximate the slowly varying \(\HLCorr\) by a two-point
proxy at the geometric center:
\begin{equation}
\Cpred_{\tavg}(d,k)
\;:=\;
\frac{\HLCorr(\Ngeom;I)+\HLCorr(\Ngeom+1;I)}{2\,\card{B_{d,k}}}
\sum_{n\in B_{d,k}} \frac{\Cpred(n;I)}{\HLCorr(n;I)},
\end{equation}
where \(\Ngeom\) is the nearest integer (optionally: nearest \emph{odd} integer) to \(10^k\sqrt{d(d+1)}\).
\end{definition}

\begin{remark}
Choosing \(\Npred_0,\Npred_1\) via \(\Cpred/\HLCorr = 2C_\star\) avoids recomputing \(\HLCorr\) on the block; since
\(\HLCorr(n;I)=1+O(1/\log n)\) varies slowly, these extremizers coincide with those for \(\Cpred\) up to \(O(1/\log n)\).
The two-point proxy \(\big(\HLCorr(\Ngeom)+\HLCorr(\Ngeom+1)\big)/2\) captures the parity drift.
\end{remark}

\noindent\emph{Convention.} \(\Cpred\) is on the \textbf{ordered-pairs} scale; for the unordered version use \(C_\star\).

Finally a \( \Lambda \) can be defined and used to test the model.

\begin{definition}[Relative Discrepancy Between Predicted and Measured]
\label{def:lambda}
All symbols are as defined above. For any finite index set \(B\) (e.g.\ a decimal block \(B_{d,k}\)),
define the dimensionless relative discrepancies:
\begin{equation}
\Lambda_{\tavg}(B)\;:=\;\;\log{\frac{\Cmeas_{\tavg}(B)}{\Cpred_{\tavg}(B)}}.
\end{equation}
\begin{equation}
\Lambda_{\min}(B)\;:=\;\log{\frac{\Cmeas_{\min}(B)}{\Cpred_{\min}(B)}},
\qquad
\Lambda_{\max}(B)\;:=\;\log{\frac{\Cmeas_{\max}(B)}{\Cpred_{\max}(B)}}.
\end{equation}
Optionally, the per-\(n\) pointwise discrepancy is
\begin{equation}
\Lambda(n;I)\;:=\;\log{\frac{\Cmeas(n;I)}{\Cpred(n;I)}}.
\end{equation}
These are on the ordered-pairs scale and satisfy \(\Lambda\to 0\) when the model matches measurements.
If the percent error is of interest use \( \left( e^{\Lambda} - 1 \right) 100\% \).
\end{definition}

\begin{remark}[Order-of-magnitude decay from window log rescaling]
If the effective density is proportional to \(\frac{1}{\log^2{x}}\) and the window spans
\(\left[\frac{n}{2},\,\frac{3n}{2}\right]\), replacing \(\log^2 n\) by a window edge produces the envelope
\begin{equation}
F(n):=\frac{\log^2{\frac{3n}{2}}}{\log^2{\frac{n}{2}}}
=1+\frac{2\log 3}{\log{\frac{n}{2}}}+O\!\Big(\frac{1}{\log^2 n}\Big).
\end{equation}
Thus the deterministic drift from freezing the log decays like \(1/\log n\) (slowly).
In practice the numerator is not attained at the extreme edge, so realized drift
is smaller but has the same \(1/\log n\) scale. This effect is distinct from any
circle-method correction \(\Lambda(n;I)\).
\end{remark}

\begin{remark}[Consistency with the independent-pair heuristic]
A naïve independence model would replace the factor
\(\prod_{3\le p\le y}(1-\tfrac{2}{p})\) by \(\prod_{3\le p\le y}(1-\tfrac{1}{p})^2\).
Since
\begin{equation}
\frac{1-\tfrac{2}{p}}{(1-\tfrac{1}{p})^2}=1-\frac{1}{(p-1)^2},
\qquad
\prod_{p\ge3}\Bigl(1-\frac{1}{(p-1)^2}\Bigr)=C_2,
\end{equation}
one has \cite{HardyLittlewood1923, MontgomeryVaughan2007}
\begin{equation}
\prod_{3\le p\le y}\!\Bigl(1-\frac{2}{p}\Bigr)
\sim \frac{4e^{-2\gamma}C_2}{\log^2 y},
\qquad
\prod_{3\le p\le y}\!\Bigl(1-\frac{1}{p}\Bigr)^2
\sim \frac{4e^{-2\gamma}}{\log^2 y}.
\end{equation}
Thus, if an independent baseline is used, the missing twin-correlation factor
is exactly \(C_2\); using the pairs singular series \(S_{\mathrm{GB}}(2n)\) (which
already incorporates this correlation) restores the same \(M/\log^2 n\) scale
constant as the \( (1-\tfrac{2}{p}) \) baseline. Since the analysis is conducted exclusively on the
pairs scale with \(S_{\mathrm{GB}}\), then the two viewpoints agree.
\end{remark}

\begin{remark}[Scope and Validation]
The constructions and normalizations above (e.g.\ \( C_\star \), \( \Cpred \), \( \betacal \))
are heuristic and conditioned on the Hardy–Littlewood Conjecture~A and the usual
independence assumptions behind the sieve baseline. They are presented to define
the \emph{predicted} quantities that should track the \emph{measured} constants.
No quantitative error bounds are proved here. The degree of agreement between
predicted and empirical values is established \emph{a posteriori} in the
Statistical Analysis section, where \( \Cpred \) to \( \Cmeas \) are compared across ranges,
windows, and extremal cases.
\end{remark}

\needspace{10\baselineskip}

\section{In-Window Statistical Analysis}\label{sec:stats}

The Hardy–Littlewood Conjecture A (HL-A) is adopted as a modelling assumption for interpreting per-\( n \) counts; no claim of proof is made. The sieve bounds and \( \Bwin \) identities are independent of this assumption.

To the author’s knowledge, there is no precedent for a systematic in-window statistical analysis of HL-A. Previous computational efforts (e.g. \cite{OliveiraESilva2014}) verified the strong Goldbach conjecture globally, while analytic studies examined distributions of primes in short intervals \cite{MontgomerySoundararajan2004,GranvilleSoundararajan2007}. The present work provides the first statistical locking-down of HL-A within analysis windows, analogous to how earlier computations statistically locked down Goldbach itself.  The rigorous sieve bounds are established independent of this analysis.

\subsection{Modelling Assumption}
Assumption (HL–A, windowed form). For admissible windows \( I \)  with \( \lvert I \rvert = o(n) \):

\begin{equation}
N_M^{(\mathrm{pairs})}(2n;I)  = \Big(2,\SGB(2n) + o(1)\Big),\frac{M}{\log^2 n}
\qquad(n\to\infty),
\end{equation}
where \( \SGB(2n) = 2C_2\prod_{p\mid n,,p\ge3} \frac{p-1}{p-2} \)

\begin{remark}
We use HL–A as a statistical model (heuristic baseline) to interpret data and form predictions. All certified bounds in this paper are independent of HL–A.
\end{remark}

While nearly a century old, HL–A remains the best parameter-free baseline to compare empirical data against. It is designed to capture the correct pointwise median of empirical data for large \( n \). Accordingly, the locations of minima and maxima in predicted values can be expected to align with measured data. Both the empirical values and predictions should approach the same asymptotic limits. To improve finite-range agreement, a correction factor \( \HLCorr \) reproduces the effects of the non-uniform distribution of primes.

\needspace{10\baselineskip}

\subsection{Data Collection}

For stability, Goldbach pairs \( (n-m, n+m) \) are measured and analyzed with \( m \in [-M,M] \setminus{0} \), where \( M=\lfloor \frac{n}{2} \rfloor \). This symmetric range was chosen for numerical stability and predictability, but the framework applies equally to other ranges.

The programs used to generate primes and sieve the data were written in C and AWK, and executed on an Intel i5 processor in a 2015 laptop normally used as a home server.  Full analysis requires several weeks, but partial results are available in minutes.  The source code is released under GPL-3.0-or-later, and the manuscript under CC-BY-4.0.  All source code and certified datasets are permanently archived on Zenodo~\cite{Riemers2025SieveGoldbach}.

The measured variables \( \Nmeas_0, \Nmeas_1, \Cmeas_{\min}, \Cmeas_{\max}, \Cmeas_{\tavg} \) are defined in Definition~\ref{semiprime-survivors}. Appendix Table~\ref{tab:raw-stats} provides raw (unnormalized) data for verification. One may notice reported minima of zero pairs for \( n=7,11,43 \). These do not contradict Goldbach’s conjecture; they arise because certain pairs such as \( (7,7) \), \( (3,19) \), and \( (7,79) \) are excluded by the chosen window.

Predicted values can be computed in under ten minutes, but counting all Goldbach pairs up to \( n=10^8 \) required several weeks. Figure~\ref{fig:Cminmaxavg} shows scatter plots of measured values versus HL–A prediction lines.

\pgfplotstableread[col sep=comma, trim cells]{pairrangejoin-100M.csv}\pairdata

\begin{figure}[h]
\centering
\begin{tikzpicture}
\begin{axis}[
  width=\textwidth, height=0.62\textwidth,
  xmode=log,
  xlabel={\(n\) (log scale)},
  ylabel={\(C\) values},
  grid=both, tick align=outside,
  legend style={at={(0.5,1.05)}, anchor=south, legend columns=-1},
  unbounded coords=discard, filter discard warning=false,
]

% --- Scatter: measured points ---
% C_min vs n_0
\addplot+[
  only marks, mark=*, mark size=1.4pt, draw=blue!60!black, fill=blue!60!black
] table[x={n_0}, y={C_min}] {\pairdata};
\addlegendentry{\( \Cmeas_{\min} \) vs \( n_0 \) }

% C_avg vs n_geom
\addplot+[
  only marks, mark=*, mark size=1.2pt, draw=green!80!black,
  mark options={fill=green!80!black, draw=black!80!black}
] table[x={n_geom}, y={C_avg}] {\pairdata};
\addlegendentry{\( \Cmeas_{\tavg} \) vs \( \Ngeom \)}

% C_max vs n_1
\addplot+[
  only marks, mark=*, mark size=1.2pt, draw=red!80!black,
  mark options={fill=red!80!black, draw=black!80!black}
] table[x={n_1}, y={C_max}] {\pairdata};
\addlegendentry{\( \Cmeas_{\max} \) vs \( n_1 \)}

% --- Lines: predicted curves ---
% Cpred_min vs Npred_0
\addplot+[
  no markers, thick, dashed, mark size=1.2pt, draw=blue!80!black,
  mark options={fill=blue!80!black, draw=black!80!black}
] table[x={Npred_0}, y={Cpred_min}] {\pairdata};
\addlegendentry{\( \Cpred_{\min} \) vs \( \Npred_0 \)}

% Cpred_avg vs n_geom
\addplot+[
  no markers, thick, dashed, color=green!50!black
] table[x={n_geom}, y={Cpred_avg}] {\pairdata};
\addlegendentry{\( \Cpred_{\tavg} \) vs \(\Ngeom\)}

% Cpred_max vs n_1
\addplot+[
  no markers, thick, dashed, color=red!80!black
] table[x={n_1}, y={Cpred_max}] {\pairdata};
\addlegendentry{\( \Cpred_{\max} \) vs \( \Npred_1 \)}

\end{axis}
\end{tikzpicture}
\caption{Scatter plots of \( \Cmeas_{\min} \), \( \Cmeas_{\max} \), and \( \Cmeas_{\tavg} \) versus \( n \) with HL–A prediction lines.   \emph{Maxima.} The prominent peaks occur at \(n\) whose odd part is a (multiple of a) primorial, in agreement with Proposition~\ref{prop:primorial-plateau}.}
\label{fig:Cminmaxavg}
\end{figure}

\needspace{10\baselineskip}

\subsection{\(\Cmeas_{\tavg}\) Analysis}

Recall from Definition~\ref{def:lambda} that

\begin{equation}
\Lambda_{\tavg}(B)\;:=\;\;\log\frac{\Cmeas_{\tavg}(B)}{\Cpred_{\tavg}(B)}.
\end{equation}

Under HL–A, measured and predicted averages are heuristically expected to converge. For very large \( n \) this should approach \( 4 \). At \( n=10^8 \), asymmetries in the prime distribution above and below \( n \) still push the average slightly higher, but the correction factor \( \HLCorr \) accounts for this. Figure~\ref{fig:lambda-avg} shows \( \Lambda_{\tavg} \) tending toward \( 0 \).

\begin{figure}[h]
  \centering
  \FigLambdaAbsSignSplit{lambdaavg-100M.csv}{n_geom}{Lambda_avg}{\( \Ngeom \)}{\( \lvert \Lambda_{\tavg} \rvert \)}{\( |\Lambda_{\tavg}| \) vs \( \Ngeom \) }
  \caption{Scatter plot of \( \lvert \Lambda_{\tavg} \rvert \) versus \( \Ngeom \) on a log-log scale.}
  \label{fig:lambda-avg}
\end{figure}

\needspace{10\baselineskip}

Table~\ref{tab:lambda_avg_summary} summarizes per-decade values. The consistent decrease demonstrates statistical convergence, supporting HL–A as an accurate predictor of average Goldbach pairs. In the 7th decade, the second-largest \( \lvert \Lambda_{\tavg} \rvert \) is \( \xSecondLambdaAvgLimit\), so it is reasonable to expect agreement with \( \lvert \Lambda_{\tavg} \rvert < \xSecondLambdaAvgLimit \) for all \( n \ge 10^8 \).

\small
\sisetup{
  input-exponent-markers = Ee,
  round-mode = places,
  round-precision = 1, % optional; keeps the table tidy
}
\begin{longtable}{
  c
  S[table-format=1.1e-1]
  S[table-format=1.1e-1]
  S[table-format=1.1e-1]
  S[table-format=1.1e-1]
  S[table-format=1.1e-1]
  S[table-format=1.1e-1]
  S[table-format=1.1e-1]
  r
}
\caption{\( \Lambda_{\tavg} \) per-decade summary (absolute extrema)}
\label{tab:lambda_avg_summary}\\
\toprule
Dec. &
\multicolumn{1}{c}{\( \lvert \text{Max} \rvert \)} &
\multicolumn{1}{c}{\( 2^{\text{nd}}\,\lvert \text{Max} \rvert \)} &
\multicolumn{1}{c}{\( \lvert \text{Min} \rvert \)} &
\multicolumn{1}{c}{\( 2^{\text{nd}}\,\lvert \text{Min} \rvert \)} &
\multicolumn{1}{c}{\( \text{Median}_{\text{raw}} \)}  &
\multicolumn{1}{c}{\( \text{Mean}_{\text{trim}} \)}  &
\multicolumn{1}{c}{\( \text{Spread}_{\text{raw}}^{\text{IQR}} \)}  &
\shortstack{Pos-\\itive} \\
\midrule
\endfirsthead
\caption[]{\( \Lambda_{\tavg} \) per-decade summary (absolute extrema)} \\
\toprule
Dec. &
\multicolumn{1}{c}{\( \lvert \text{Max} \rvert \)} &
\multicolumn{1}{c}{\( 2^{\text{nd}}\,\lvert \text{Max} \rvert \)} &
\multicolumn{1}{c}{\( \lvert \text{Min} \rvert \)} &
\multicolumn{1}{c}{\( 2^{\text{nd}}\,\lvert \text{Min} \rvert \)} &
\multicolumn{1}{c}{\( \text{Median}_{\text{raw}} \)}  &
\multicolumn{1}{c}{\( \text{Mean}_{\text{trim}} \)}  &
\multicolumn{1}{c}{\( \text{Spread}_{\text{raw}}^{\text{IQR}} \)}  &
\shortstack{Pos-\\itive} \\
\midrule
\endhead
0 & 1.020e+00 & 4.847e-01 & 1.838e-01 & 2.806e-01 & 3.178e-01 & 3.610e-01 & 2.041e-01 & 0.0\% \\
1 & 5.911e-01 & 3.889e-01 & 3.027e-02 & 1.300e-01 & 2.864e-01 & 2.732e-01 & 1.997e-01 & 0.0\% \\
2 & 1.300e-01 & 1.206e-01 & 3.029e-02 & 3.046e-02 & 6.713e-02 & 7.116e-02 & 2.848e-02 & 0.0\% \\
3 & 3.180e-02 & 2.823e-02 & 6.772e-03 & 1.395e-02 & 1.821e-02 & 2.057e-02 & 9.870e-03 & 0.0\% \\
4 & 9.519e-03 & 7.216e-03 & 2.380e-03 & 2.910e-03 & 5.668e-03 & 5.572e-03 & 2.053e-03 & 0.0\% \\
5 & 3.158e-03 & 2.249e-03 & 8.532e-04 & 1.032e-03 & 1.707e-03 & 1.632e-03 & 6.485e-04 & 0.0\% \\
6 & 9.959e-04 & 7.160e-04 & 1.884e-04 & 3.211e-04 & 4.916e-04 & 5.033e-04 & 2.292e-04 & 0.0\% \\
7 & 3.247e-04 & 2.207e-04 & 3.572e-05 & 1.283e-04 & 1.545e-04 & 1.632e-04 & 7.347e-05 & 0.0\% \\
\end{longtable}

\needspace{10\baselineskip}

\subsection{\( \Cmeas_{\min} \) Analysis}

Recall from Definition~\ref{def:lambda}:

\begin{equation}
\Lambda_{\min}(B)\; := \;\log\frac{\Cmeas_{\min}(B)}{\Cpred_{\min}(B)}.
\end{equation}

Under HL–A, predictions and measurements converge to the same limit. For very large \( n \), minima should approach \( 2C_2 \), where \( C_2 \) is the twin prime constant. Figure~\ref{fig:lambda-min} shows \( \Lambda_{\min} \to 0 \) as \( n \)  grows.

\begin{figure}[h]
  \centering
  \FigLambdaAbsSignSplit{lambdamin-100M.csv}{n_0}{Lambda_min}{\( n_0 \)}{\( \lvert \Lambda_{\min} \rvert \)}{\( |\Lambda_{\min}| \) vs \( n_0 \) }
  \caption{Scatter plot of \( \lvert \Lambda_{\min} \rvert \) versus \( n_0 \) on a log-log scale.}
  \label{fig:lambda-min}
\end{figure}

Table~\ref{tab:lambda_min_summary} confirms per-decade convergence. In the 7th decade, the second-largest \( \lvert \Lambda_{\min} \rvert \) is \( \xSecondLambdaMinLimit \), so HL–A agrees with observed data at that tolerance for \( n \ge 10^8 \).

Thus, the statistical evidence strongly supports the Goldbach conjecture: with overwhelming certainty, there are at least \( \tfrac{2.62n}{2\log^2 n} \) Goldbach pairs \( (n-m,n+m) \) for all \( n \ge 10^8 \) and admissible \( m \).

\small
\sisetup{
  input-exponent-markers = Ee,
  round-mode = places,
  round-precision = 1, % optional; keeps the table tidy
}
\begin{longtable}{
  c
  S[table-format=1.1e-1]
  S[table-format=1.1e-1]
  S[table-format=1.1e-1]
  S[table-format=1.1e-1]
  S[table-format=1.1e-1]
  S[table-format=1.1e-1]
  S[table-format=1.1e-1]
  r
}
\caption{\( \Lambda_{\min} \) per-decade summary (absolute extrema)}
\label{tab:lambda_min_summary}\\
\toprule
Dec. &
\multicolumn{1}{c}{\( \lvert \text{Max} \rvert \)} &
\multicolumn{1}{c}{\( 2^{\text{nd}}\,\lvert \text{Max} \rvert \)} &
\multicolumn{1}{c}{\( \lvert \text{Min} \rvert \)} &
\multicolumn{1}{c}{\( 2^{\text{nd}}\,\lvert \text{Min} \rvert \)} &
\multicolumn{1}{c}{\( \text{Median}_{\text{raw}} \)}  &
\multicolumn{1}{c}{\( \text{Mean}_{\text{trim}} \)}  &
\multicolumn{1}{c}{\( \text{Spread}_{\text{raw}}^{\text{IQR}} \)}  &
\shortstack{Pos-\\itive} \\
\midrule
\endfirsthead
\caption[]{\( \Lambda_{\min} \) per-decade summary (absolute extrema)} \\
\toprule
Dec. &
\multicolumn{1}{c}{\( \lvert \text{Max} \rvert \)} &
\multicolumn{1}{c}{\( 2^{\text{nd}}\,\lvert \text{Max} \rvert \)} &
\multicolumn{1}{c}{\( \lvert \text{Min} \rvert \)} &
\multicolumn{1}{c}{\( 2^{\text{nd}}\,\lvert \text{Min} \rvert \)} &
\multicolumn{1}{c}{\( \text{Median}_{\text{raw}} \)}  &
\multicolumn{1}{c}{\( \text{Mean}_{\text{trim}} \)}  &
\multicolumn{1}{c}{\( \text{Spread}_{\text{raw}}^{\text{IQR}} \)}  &
\shortstack{Pos-\\itive} \\
\midrule
\endhead
0 & 1.067e+00 & 4.989e-01 & 1.870e-01 & 2.638e-01 & 4.011e-01 & 3.879e-01 & 2.351e-01 & 0.0\% \\
1 & 1.144e+00 & 1.031e+00 & 4.039e-01 & 5.880e-01 & 8.801e-01 & 8.111e-01 & 3.645e-01 & 0.0\% \\
2 & 9.385e-01 & 8.701e-01 & 3.031e-01 & 4.071e-01 & 4.538e-01 & 5.474e-01 & 2.704e-01 & 0.0\% \\
3 & 4.869e-01 & 4.448e-01 & 2.032e-01 & 2.104e-01 & 3.074e-01 & 3.128e-01 & 1.419e-01 & 0.0\% \\
4 & 2.246e-01 & 1.567e-01 & 8.991e-02 & 9.891e-02 & 1.271e-01 & 1.298e-01 & 3.744e-02 & 0.0\% \\
5 & 1.029e-01 & 7.705e-02 & 4.127e-02 & 4.316e-02 & 5.124e-02 & 5.568e-02 & 1.358e-02 & 0.0\% \\
6 & 4.061e-02 & 3.000e-02 & 1.619e-02 & 1.756e-02 & 2.265e-02 & 2.299e-02 & 7.621e-03 & 0.0\% \\
7 & 1.754e-02 & 1.279e-02 & 7.085e-03 & 7.110e-03 & 8.512e-03 & 9.302e-03 & 3.441e-03 & 0.0\% \\
\end{longtable}

\needspace{10\baselineskip}

\subsection{\( \Cmeas_{\max} \) Analysis}

Recall from Definition~\ref{def:lambda}:

\begin{equation}
\Lambda_{\max}(B)\;:=\;\log\frac{\Cmeas_{\max}(B)}{\Cpred_{\max}(B)}.
\end{equation}

Both HL–A and data show step increases at primorial values, each of order \( \log\log\log n\). Accumulated over primes up to size \( n \), this yields overall extremal growth of order

\begin{equation}
O!\left(\tfrac{n\log\log n}{\log^2 n}\right).
\end{equation}

Predictions corrected by \( \HLCorr \) account for asymmetry in prime distribution. Figure~\ref{fig:lambda-max} shows \( \Lambda_{\max} \to 0 \) with \( n \).

\begin{remark}[Euler-factor Step Effect]
Each primorial step corresponds to introducing a new Euler factor \( \frac{(p-1)}{(p-2)} \) in the singular series.\cite{HardyLittlewood1923,Vaughan1997} Excluding divisibility by a new prime slightly increases the expected Goldbach count, producing the \( \log\log\log n \)-sized steps.
\end{remark}

\begin{figure}[h]
  \centering
  \FigLambdaAbsSignSplit{lambdamax-100M.csv}{n_1}{Lambda_max}{\( n_1 \)}{\( \lvert \Lambda_{\max} \rvert \)}{\( |\Lambda_{\max}| \) vs \( n_1 \) }
  \caption{Scatter plot of \( \lvert \Lambda_{\max} \rvert \) versus \( n_1 \) on a log-log scale.}
  \label{fig:lambda-max}
\end{figure}

Table~\ref{tab:lambda_max_summary} shows decreasing per-decade values, again confirming convergence. In the 7th decade, the second-largest \( \lvert  \Lambda_{\max} \rvert \) is \( \xSecondLambdaMaxLimit \), supporting HL–A agreement at that level for \( n \ge 10^8 \).

\small
\sisetup{
  input-exponent-markers = Ee,
  round-mode = places,
  round-precision = 1, % optional; keeps the table tidy
}
\begin{longtable}{
  c
  S[table-format=1.1e-1]
  S[table-format=1.1e-1]
  S[table-format=1.1e-1]
  S[table-format=1.1e-1]
  S[table-format=1.1e-1]
  S[table-format=1.1e-1]
  S[table-format=1.1e-1]
  r
}
\caption{\( \Lambda_{\max} \) per-decade summary (absolute extrema)}
\label{tab:lambda_max_summary}\\
\toprule
Dec. &
\multicolumn{1}{c}{\( \lvert \text{Max} \rvert \)} &
\multicolumn{1}{c}{\( 2^{\text{nd}}\,\lvert \text{Max} \rvert \)} &
\multicolumn{1}{c}{\( \lvert \text{Min} \rvert \)} &
\multicolumn{1}{c}{\( 2^{\text{nd}}\,\lvert \text{Min} \rvert \)} &
\multicolumn{1}{c}{\( \text{Median}_{\text{raw}} \)}  &
\multicolumn{1}{c}{\( \text{Mean}_{\text{trim}} \)}  &
\multicolumn{1}{c}{\( \text{Spread}_{\text{raw}}^{\text{IQR}} \)}  &
\shortstack{Pos-\\itive} \\
\midrule
\endfirsthead
\caption[]{\( \Lambda_{\max} \) per-decade summary (absolute extrema)} \\
\toprule
Dec. &
\multicolumn{1}{c}{\( \lvert \text{Max} \rvert \)} &
\multicolumn{1}{c}{\( 2^{\text{nd}}\,\lvert \text{Max} \rvert \)} &
\multicolumn{1}{c}{\( \lvert \text{Min} \rvert \)} &
\multicolumn{1}{c}{\( 2^{\text{nd}}\,\lvert \text{Min} \rvert \)} &
\multicolumn{1}{c}{\( \text{Median}_{\text{raw}} \)}  &
\multicolumn{1}{c}{\( \text{Mean}_{\text{trim}} \)}  &
\multicolumn{1}{c}{\( \text{Spread}_{\text{raw}}^{\text{IQR}} \)}  &
\shortstack{Pos-\\itive} \\
\midrule
\endhead
0 & 1.067e+00 & 4.989e-01 & 1.870e-01 & 2.638e-01 & 4.011e-01 & 3.879e-01 & 2.351e-01 & 0.0\% \\
1 & 6.020e-01 & 3.008e-01 & 6.062e-03 & 2.827e-02 & 1.738e-01 & 1.743e-01 & 2.130e-01 & 0.0\% \\
2 & 1.252e-01 & 8.537e-02 & 8.198e-03 & 3.496e-02 & 6.642e-02 & 6.480e-02 & 3.501e-02 & 0.0\% \\
3 & 3.889e-02 & 3.720e-02 & 2.004e-03 & 4.782e-03 & 1.291e-02 & 1.499e-02 & 1.649e-02 & 44.4\% \\
4 & 1.570e-02 & 1.132e-02 & 2.782e-03 & 3.166e-03 & 6.895e-03 & 7.029e-03 & 4.725e-03 & 33.3\% \\
5 & 7.070e-03 & 5.174e-03 & 9.167e-04 & 1.240e-03 & 1.677e-03 & 2.235e-03 & 1.291e-03 & 44.4\% \\
6 & 3.258e-03 & 2.997e-03 & 4.374e-04 & 5.687e-04 & 1.054e-03 & 1.348e-03 & 1.333e-03 & 66.7\% \\
7 & 3.860e-03 & 2.135e-03 & 3.185e-04 & 6.205e-04 & 1.045e-03 & 1.295e-03 & 1.054e-03 & 0.0\% \\
\end{longtable}

Blockwise maxima align with the singular series.  Proposition~\ref{prop:primorial-plateau} predicts that on each scale \([P_y,\,p_{y+1}P_y)\) the windowed count is maximized when the odd part of \(n\) is divisible by the odd primorial \(P_y=\prod_{3\le p\le p_y}p\).

In our per–block maxima (Table~\ref{tab:normalized-status}), this is reflected by winners at 
\(15{,}015\), \(30{,}030\), \(45{,}045\), \(60{,}060\) (odd part \(=P_{13}\)), 
and later \(255{,}255\), \(510{,}510\) (odd part \(=P_{17}\)), 
\(4{,}849{,}845\), \(9{,}699{,}690\) (odd part \(=P_{19}\)).
Because the singular series ignores exponents and the prime \(2\), many nearby multiples share the same singular–series value; with per–decade decimal recording, only one such candidate appears as the block maximum.

\needspace{10\baselineskip}

\subsection{Primorial plateaus and HL–A}\label{sec:primorial-prop}

\begin{equation}\label{eq:half-primorial}
P_y \;:=\; \prod_{3\le p\le p_y} p \;=\; \frac{p_y^\#}{2}\qquad\text{(``half a primorial'')}.
\end{equation}

\begin{lemma}[Singular–series plateaus (unconditional)]\label{lem:SS-plateau}
For even $N=2n$,
\begin{equation}\label{eq:sing-series-2n}
\mathfrak S(2n)\;=\;2C_2 \prod_{\substack{p\mid n\\ p\ge 3}}\frac{p-1}{p-2}.
\end{equation}
Fix $X>0$ and consider $n\le X$. Then $\mathfrak S(2n)$ is maximized when $n$ is divisible by $P_y$ for the largest $p_y$ with $P_y\le X$; equivalently, within $[P_y,\,p_{y+1}P_y)$ the maximizers are precisely the multiples of $P_y$, and all such $n$ have the same value of $\mathfrak S(2n)$.
\end{lemma}

\begin{proof}
Write $f(p):=(p-1)/(p-2)=1+\frac{1}{p-2}$. Then
\(\mathfrak S(2n)=2C_2\prod_{p\mid n,\,p\ge 3} f(p)\).
If $q>p\ge 3$ then $f(q)<f(p)$, so among sets of distinct odd primes the product is maximized by the initial segment $\{3,5,\dots,p_y\}$. The smallest integer carrying exactly this set is $P_y$, yielding the record at $n=P_y$. On $[P_y,p_{y+1}P_y)$ no new distinct odd prime beyond $p_y$ can divide $n$, so the maximizers are exactly the multiples of $P_y$ (exponents do not affect $\mathfrak S$), and all such $n$ share the same $\mathfrak S(2n)$. 
\end{proof}

\begin{proposition}[Record and plateau maxima \emph{under HL–A}]\label{prop:primorial-plateau}
Assume the Hardy–Littlewood Conjecture~A in the form
\begin{equation}\label{eq:HLA-main-term}
\mathbb E\,R_2(2n)\ \sim\ \frac{\mathfrak S(2n)\,2n}{\log^2(2n)}\qquad(n\to\infty),
\end{equation}
uniformly on fixed-size blocks. Then, within each interval $[P_y,\,p_{y+1}P_y)$, the \emph{normalized} expected count
\begin{equation}\label{eq:normalized-expectation}
\frac{\log^2(2n)}{2n}\,\mathbb E\,R_2(2n)
\end{equation}
attains its maximum precisely at those $n$ with $P_y\mid n$ (i.e., multiples of $p_y^\#/2$). In particular,
\begin{equation}\label{eq:record-max}
\max_{\,2n\le 2P_y}\ \frac{\log^2(2n)}{2n}\,\mathbb E\,R_2(2n)
\ \text{ is attained at }\, n=P_y,
\end{equation}
so $P_y$ are record maximizers as $y$ increases.
\end{proposition}

\begin{proof}
By \eqref{eq:HLA-main-term}, the normalized expectation \eqref{eq:normalized-expectation} is asymptotic to $\mathfrak S(2n)$, while $2n/\log^2(2n)$ varies slowly across a fixed block. Therefore the maximizers of \eqref{eq:normalized-expectation} coincide with the maximizers of $\mathfrak S(2n)$, which are exactly the multiples of $P_y$ by Lemma~\ref{lem:SS-plateau}. The record statement \eqref{eq:record-max} follows likewise.
\end{proof}

\begin{remark}[Empirical alignment]
Table~\ref{tab:normalized-status} shows that blockwise maxima of the observed normalized counts occur at (or extremely near) multiples of $P_y$, matching the HL–A prediction up to sampling noise.
\end{remark}

\begin{remark}[HL–A heuristic for normalized maxima]
Under the Hardy–Littlewood baseline, the expected ordered Goldbach count satisfies
\begin{equation}\label{eq:HLA-main-term}
\mathbb E\,R_2(2n)\ \asymp\ \frac{\mathfrak S(2n)\,2n}{\log^2(2n)}.
\end{equation}
Across a fixed scale the factor \(2n/\log^2(2n)\) varies slowly, while \(\mathfrak S(2n)\) follows Proposition~\ref{prop:primorial-plateau}. Hence HL–A predicts that \emph{blockwise maxima} of normalized counts occur at \(n\) that are multiples of \(P_y=p_y^\#/2\) within each interval \([P_y,\,p_{y+1}P_y)\) (the “primorial plateaus”). Empirics in Table~\ref{tab:normalized-status} match this pattern.
\end{remark}

\needspace{10\baselineskip}

\subsection{Conclusion on Analysis}

The sieve framework was tested against HL–A for all \( n<10^8 \) . The measured values \( \Cmeas_{\min},\Cmeas_{\max},\Cmeas_{\tavg} \) asymptotically approach predictions:

\begin{equation}
\lvert \Lambda_{\min} \rvert \le \xSecondLambdaMinLimit,\qquad
\lvert \Lambda_{\max} \rvert \le \xSecondLambdaMaxLimit,\qquad
\lvert \Lambda_{\tavg} \rvert \le \xSecondLambdaAvgLimit.
\end{equation}

This is not a proof, but is a statistically robost conclusion: HL-A accurately models Goldbach pairs in the chosen window, with error bounds shrinking across decades.

\section{Sieve-Theoretic Goldbach}\label{sec:sgb}

\subsection{Sieve reduction on \( Q(n,m) \)}\label{sec:reduction}
\begin{lemma}[Analytic lower bound via certified shifted products]
\label{lem:analytic-lower-bound-reduction}
Let \( pq \) be a semiprime with distinct odd prime factors \( p,q \), where \( pq=n^2-m^2 \) with \( n>m \). 
For \( P(x):=\prod_{3\le r\le x}(1-\tfrac{1}{r-1}) \) and \( \Kem:=4\,e^{-\gamma}C_2 \) with \( C_2 \) as in Lemma~\ref{lem:shifted-enclosure}, we have
\begin{equation}
R(pq)\ \ge\ \frac{\Kem^2}{\log p\,\log q}
\left(1 \pm \delta(p,q)\right),
\end{equation}
where \( \delta(p,q) \) is an explicit decreasing function from Lemma~\ref{lem:shifted-enclosure}.
\end{lemma}

\begin{proof}
By Lemma~\ref{lem:shifted-enclosure} with \( x=\sqrt{p} \) and \( x=\sqrt{q} \),
\begin{equation}
P(\sqrt{p})\ \in\ \left[\frac{\Kem}{\log p} - \varepsilon_P(\sqrt{p}),\; \frac{\Kem}{\log p} + \varepsilon_P(\sqrt{p})\right],
\end{equation}
and similarly for \( q \). Multiplying the two intervals and expanding the error term gives the stated bound with
\begin{equation}
\delta(p,q)\ :=\ \frac{\varepsilon_P(\sqrt{p})}{\Kem/\log p} \;+\; \frac{\varepsilon_P(\sqrt{q})}{\Kem/\log q} \;+\; \frac{\varepsilon_P(\sqrt{p})\,\varepsilon_P(\sqrt{q})}{(\Kem/\log p)(\Kem/\log q)},
\end{equation}
which is explicit and decreases in both \( p \) and \( q \).
\end{proof}

\subsection{Main theorem (certified lower bound)}\label{sec:main-thm}
\begin{theorem}[Goldbach Pairs and a Double–Euler Product Sieve Bound]
\label{thm:main}
Let \(n\in\mathbb{N}\) and set \(2n\) as the even number under test.
Write \(\mathcal{G}(n)\) for the number of \emph{ordered} Goldbach pairs \((p,q)\) with \(p+q=2n\).
For each pair write \(m := \tfrac{q-p}{2}\).
Define the specific window size
\begin{equation}
M(n) \;:=\; \bigl\lfloor \tfrac{n}{2} \bigr\rfloor .
\end{equation}
Then a subset of Goldbach pairs satisfies \(1 \le |m| \le M(n)\), hence
\begin{equation}
\mathcal{G}(n;M) \;:=\; \#\{(p,q): p+q=2n,\ 1\le |m| \le M(n)\}.
\end{equation}

\begin{enumerate}
   \item \textbf{Computational Coverage (up to \(\nprodstar\)).}
   For all \(n\) with \(2n \in [4,\,2\nprodstar)\), at least one ordered Goldbach pair exists
   (verified by direct computation). A CSV listing one witness pair for each \(2n<2\nprodstar\) and the
   corresponding verification checksums are included with this submission.\footnote{This explicit verification up to \( \nprodstar \) is complementary to large-scale computational results such as Oliveira e Silva, Herzog, and Pardi [OeSHP2014], who verified Goldbach’s conjecture for all even integers up to \( 4\cdot 10^{18} \).  This approach is distinct in that it provides a certified sieve-theoretic lower bound valid for all \( n \ge \nprodstar \), thereby bridging analytic proof and computational verification.}

  \item \textbf{Certified Analytic Lower Bound (Global Ordered Pairs).}
  Define:
  \begin{equation}
    \CminusProduct(n) := \log^2{n} \;
      \prod_{\substack{p>2\\ p\in\mathcal{P}}}^{\sqrt{n}}
        \!\Bigl(1-\tfrac{1}{p-1}\Bigr)\,
      \prod_{\substack{p>2\\ p\in\mathcal{P}}}^{\sqrt{\tfrac{3n}{2}}}
        \!\Bigl(1-\tfrac{1}{p-1}\Bigr)
     \label{eq:cminus-product-definition}
   \end{equation}

  There exists a constant \( \nprodstar \) such that, for all \(n \ge \nprodstar\),
  \begin{equation}
    \mathcal{G}(n;M)\;\ge\;\frac{\CminusProduct(n)M(n)}{log^2 n},
    \qquad
    \text{with } M(n)=\bigl\lfloor \tfrac{n}{2} \bigr\rfloor
    \ \text{ and }\
    \nprodstar=\xMertens .
    \label{eq:analytic-lower-bound-global}
  \end{equation}
\end{enumerate}
\end{theorem}

\begin{remark}
Since \(\mathcal{G}(n) \ge \mathcal{G}(n;M(n))\) by construction, the bound
\eqref{eq:analytic-lower-bound-global} establishes a valid global
analytic lower bound for the ordered Goldbach count.
\end{remark}

\begin{proof}
\emph{Parity-obstruction context.}

\noindent\textbf{Establishing the Product of Two Euler Series Lower Bound.}

We show that the Eratosthenes sieve\cite{FriedlanderIwaniec2010} applied to the quadratic form
\begin{equation}
Q(n,m)=(n-m)(n+m)
\end{equation}
yields a rigorous product--of--two--Euler--series lower bound, free of the
classical parity obstruction\cite{Chen1973, IwaniecKowalski2004}, provided the separation condition holds.

With loss of generality restrict the separation regime:
\begin{equation}\label{eq:sep}
n-|m|\;>\;\sqrt{\,n+|m|\,}, \qquad (m\in I^{\mathrm{par}}).
\end{equation}
On the symmetric window \(|m|\le M(n)=\lfloor \frac{n}{2} \rfloor\), this holds whenever
\(\frac{n}{2} > \sqrt{\frac{3n}{2}}\), i.e. for all \(n\ge7\).  

Under the separation condition \eqref{eq:sep}, an Eratosthenes sieve on \(Q(n,m)\) up to
\(\sqrt{\,n+|m|\,}\) removes all composites and leaves only pairs of primes
\((n-m,n+m)\). Equivalently, sieving \(n-m\) up to \(\sqrt{n-m}\) and \(n+m\)
up to \(\sqrt{n+m}\) gives the same surviving set.  Thus, the sieve on \(Q(n,m)=(n-m)(n+m)\)
factorizes cleanly into the product of two Euler series.\cite{Vaughan1997}

For a fixed \(n\) and \(m\), and for each odd prime \(p\), let
\begin{equation}
\mathcal R_p^-:=\{\,m\bmod p:\ p\mid n-|m|\,\},\qquad
\mathcal R_p^+:=\{\,m\bmod p:\ p\mid n+|m|\,\}.
\end{equation}
Then \(|\mathcal R_p^-|=|\mathcal R_p^+|=1\) and, when both constraints are active,
the union has size at most \(2\):
\(|\mathcal R_p^-\cup\mathcal R_p^+|\le 2\).
To \emph{certify} primality of \(n\!-\!m\) it suffices to exclude
\(\mathcal R_p^-\) for all \(p\le\sqrt{n-|m|}\); similarly for \(n\!+\!m\)
exclude \(\mathcal R_p^+\) for all \(p\le\sqrt{n+|m|}\).
By the (one–sided) linear–sieve lower bound (e.g.\ \cite[Ch.~6]{IwaniecKowalski2004}),
the surviving proportion for \(n\!-\!m\) is
\begin{equation}
S_-(n,m) = \prod_{3\le p\le \sqrt{n-|m|}}\!\bigl(1-\tfrac{1}{p-1}\bigr),
\end{equation}
and for \(n\!+\!m\) is
\begin{equation}
S_+(n,m) = \prod_{3\le p\le \sqrt{n+|m|}}\!\bigl(1-\tfrac{1}{p-1}\bigr)
\end{equation}

Because the residue constraints for \(n-m\) and \(n+m\) act on \emph{disjoint} single classes modulo each odd prime \(p \), and because we take the \emph{minima} of the one–sided lower bounds before multiplying, the product \(S_-(n,m)S_+(n,m)\) is a valid conservative lower bound; no independence hypothesis is used.
\begin{equation}
S_{-}(n,m)\,S_{+}(n,m)
\ :=\
\prod_{3\le p\le \sqrt{n-m}}\!\Bigl(1-\frac{1}{p-1}\Bigr)\,
\prod_{3\le p\le \sqrt{n+m}}\!\Bigl(1-\frac{1}{p-1}\Bigr)
\end{equation}

The separation condition \eqref{eq:sep} ensures that sieving \(Q(n,m)\) up to \(\sqrt{n+|m|}\) subsumes the individual prime tests up to \(\sqrt{n\pm m}\), so the product decomposition into the two one–sided Euler factors is legitimate.  For each \(m\),
\begin{equation}
\mathbf{1}_{\{\text{\(n\pm m\) both prime}\}}\;\ge\;S_-(n,m)\,S_+(n,m).
\end{equation}

Summing over \(m\in I^{\mathrm{par}}\) gives
\begin{equation}
\mathcal G(n;I)\ \ge\ \sum_{m\in I^{\mathrm{par}}} S_-(n,m)\,S_+(n,m).
\end{equation}
Bounding by the minima,
\begin{equation}
\mathcal G(n;I)\ \ge\
M(n)\cdot\Bigl(\min_{m}S_-(n,m)\Bigr)\,\Bigl(\min_{m}S_+(n,m)\Bigr).
\end{equation}

On the symmetric window \(|m|\le M(n)=\lfloor \frac{n}{2}\rfloor\), the minima
occur at the largest cutoffs, hence
\begin{equation}\label{eq:pair-lb-product}
\mathcal G(n;I)\ \ge\
M(n)\,
\prod_{3\le p\le \sqrt n}\Bigl(1-\frac{1}{p-1}\Bigr)\,
\prod_{3\le p\le \sqrt{\frac{3n}{2}}}\Bigl(1-\frac{1}{p-1}\Bigr).
\end{equation}

\medskip
\noindent
By the Mertens--type enclosure (Lemma~\ref{lem:analytic-lower-bound-reduction}),
\begin{equation}
\prod_{p\le \sqrt x}\Bigl(1-\frac{1}{p-1}\Bigr)\ \sim\ \frac{\Kem}{\log x},
\qquad \Kem:=4e^{-\gamma}C_2,\quad (C_2 \text{ the twin prime constant}),
\end{equation}
so \eqref{eq:pair-lb-product} becomes
\begin{equation}
\mathcal G(n;I)\ \gtrsim\
\frac{\Kem^2\,M(n)}{\log n\,\log{\frac{3n}{2}}}.
\end{equation}
Equivalently, defining
\begin{equation}\label{eq:Cminus}
\CminusProduct(n)
:=\log^2 n\,
\prod_{3\le p\le \sqrt n}\Bigl(1-\frac{1}{p-1}\Bigr)\,
\prod_{3\le p\le \sqrt{\frac{3n}{2}}}\Bigl(1-\frac{1}{p-1}\Bigr),
\end{equation}
gives the analytic lower bound
\begin{equation}
\mathcal G(n;I)\ \ge\ \frac{\CminusProduct(n)}{\log^2 n}\,M(n).
\end{equation}

This exhibits the prime--pair density as the product of two Euler factors,
one attached to \(n-m\) and the other to \(n+m\), with no Hardy--Littlewood
assumptions.\cite{HardyLittlewood1923,Vaughan1997} 

\noindent\textbf{Comparison of Observed Minima.}

Figure~\ref{fig:lower-analytic-bound-comparison} displays
a numerical comparison between sieve data and this analytic bound.

\pgfplotstableread[col sep=comma,trim cells]{cpslowerbound-100M.csv}\cpsdata

\begin{figure}[H]
\centering
\begin{tikzpicture}
\begin{axis}[
  width=\textwidth, height=0.6\textwidth,
  xmode=log, % log x-axis
  xlabel={\(n_0\) (log scale)},
  ylabel={\(C\) values},
  grid=both,
  tick align=outside,
  legend style={at={(0.5,1.05)}, anchor=south, legend columns=-1},
  unbounded coords=discard,         % drop NaN/Inf quietly
  filter discard warning=false,
]

% Cmin: scatter points
\addplot+[
  only marks, mark=*, mark size=1.2pt
] table[x={n_0}, y={Cmin}] {\cpsdata};
\addlegendentry{\(\Cmeas_{\min}(n_0)\)}

% Cminus: solid line (add 'smooth' to spline)
\addplot+[
  no markers, thick
] table[x={n_0}, y={Cminus}] {\cpsdata};
\addlegendentry{\(\Cminus(n_0)\)}

% CminusAsym: dashed line (add 'smooth' to spline)
\addplot+[
  no markers, thick, dashed, smooth
] table[x={n_0}, y={CminusAsym}] {\cpsdata};
\addlegendentry{\(\CminusAsymp(n_0)\)}

\end{axis}
\end{tikzpicture}
\caption{Comparison of the observed minima \( \Cmeas_{\min}(n_0) \) (points) with the analytic lower bound \( \CminusProduct(N_0) \) (solid line)
and corresponding asymptotic proxy \( \tfrac{\Kem^2\;\log{n}}{\log{\frac{3n}{2}}} \) (dashed line), where \( \Kem \approx 1.482616 \).
For \( N_0 \ge \xMertens \), the minimal observed margin analytical margin is
\(\displaystyle 
\eta_{\tana} = \min_{N_0 \ge \xMertens} \bigl( \Cmeas_{\min}(N_0) - \CminusProduct(N_0) ) = \xDeltaGlobal,
\)
and for \( N_0 \ge n_{5\%} = 4.11 \cdot 10^{4} \), the minimal observed margin is
\(\displaystyle 
\eta = \min_{N_0 \ge n_{5\%}} \bigl( \Cmeas_{\min}(N_0) - \CminusProduct(N_0) ) = \xEtaStat,
\)
confirming that \( \Cmeas_{\min}(N_0) \ge \tfrac{\CminusProduct(N_0) M(N_0)}{\log^2{N_0}} \) throughout the verified range.}
\label{fig:lower-analytic-bound-comparison}
\end{figure}


Let \(\nprodstar\) denote the smallest \(N_0\) such that \(\Cmeas_{\min}(n) \ge \CminusProduct(n)\) for all subsequent \(n\) in our record.
From Figure~\ref{fig:lower-analytic-bound-comparison}, the last recorded minimum below \(\CminusProduct\) occurs at \(N_0 = \xPreMertens \).
The next recorded minimum is therefore taken as a
a conservative permanence threshold, \(\nprodstar = \xMertens\), with local margin
\(\zeta = \Cmeas_{\min}(\nprodstar) - \CminusProduct(\nprodstar) = \xDeltaMertens \).
Because only minima are recorded, intermediate non-minima are not observed; consequently, \(\nprodstar\) may occur slightly later than
the last crossing, and the value reported here is conservative.

\begin{remark}
Why the bound is not tight for small \(n\).
Each isolated factor (e.g.\ \(1-\tfrac{2}{7-1}\)) is an exact maximum possible removal for that prime \emph{if} it acted first.
In the sieve, earlier primes thin the set; later primes then act on an irregular remainder and their effects overlap statistically.
Thus the full product overestimates combined removal at small \(n\), and in low-statistics regimes the sieve can (and often does)
remove \(100\%\) of candidates, hence \(\Cmeas_{\min}\) may fall below the asymptotic floor until \(n\) is large enough
(around \(10^4\)) for the probabilistic model to be valid.
\end{remark}

By Appendix~\ref{app:shifted-product},
\begin{equation}
\log(\sqrt{p})\,P(\sqrt{p}) = \frac{\Kem}{\log p} \pm \varepsilon_P(\sqrt{p}),
\quad
\log(\sqrt{q})\,P(\sqrt{q}) = \frac{\Kem}{\log q} \pm \varepsilon_P(\sqrt{q}),
\end{equation}
where \( \Kem = 4 e^{-\gamma} C_2 \).
Multiplying the two factors gives the claimed bound.

\paragraph{Hard Statistical Validity Threshold.}
Define the mean lower–bound prediction
\begin{equation}
\mu(n)\;:=\;\frac{\Kem^2\,M}{\log^2 n}
\;=\;\frac{(2.1982)\,\left(\frac{n}{2}\right)}{\log^2 n}
\;=\;\frac{1.0991\,n}{\log^2 n}.
\end{equation}
The criterion for “sufficient statistics” is \(\mu(n)\ge 400\) (5\% relative statistical tolerance). Solving
\begin{equation}
\frac{1.0991\,n}{\log^2 n}\ \ge\ 400
\end{equation}
gives the explicit threshold
\begin{equation}
n_{5\%}=4.11\cdot 10^{4}.
\end{equation}

\paragraph{Monotonic Dominance Beyond the Threshold.}
For each recorded minimum \(N_0\) with \(N_0\ge n_{5\%}\), define the (dimensionless) gap
\begin{equation}
\Delta(N_0)\;:=\;\Cmeas_{\min}(N_0)\;-\;\CminusProduct(N_0)
\end{equation}
From the dataset, we observe
\begin{equation}
\eta\;:=\;\min_{N_0\ \ge\ n_{5\%}}\ \Delta(\xNzeroStat)\;=\;\xEtaStat\;>\;0,
\end{equation}
so \(\Cmeas_{\min}(N_0)\ \ge\ \CminusProduct(N_0)\) holds for all recorded minima beyond \(n_{5\%}\) with a uniform margin of \(\xEtaStat\).
Equivalently,
\begin{equation}
\min_{N_0\in[n_{5\%},\,N_{\max}]}\Bigl(\Cmeas_{\min}(N_0)-\CminusProduct(N_0)\Bigr) \;=\; \eta \;>\; 0.
\end{equation}

\noindent\emph{Notes.}
(i) Only minima is recorded, this is conservative: any unrecorded intermediate values lie \emph{above} \(\Cmeas_{\min}\).
(ii) The numerical value \(\eta=\xEtaStat\) is computed directly from the table used in Fig.~\ref{fig:lower-analytic-bound-comparison}; the first \(N_0\) attaining \(\eta\) is also recorded in the caption.

From the statistical validity criterion
\begin{equation}
\mu(n) = \frac{1.0991\,n}{\log^2 n} \ge 400,
\end{equation}
we obtain a hard threshold
\begin{equation}
n_{5\%} \;=\; 4.11\cdot 10^4,
\end{equation}
beyond which the sampling error is guaranteed to fall below \( 5\% \).

To certify that the analytic lower bound remains valid above this threshold, the dominance gap is defined:
\begin{equation}
\Delta(N_0) \;:=\; \frac{\Cmeas_{\min}(N_0)}{M} - \frac{\Kem^2}{\log^2 N_0}.
\end{equation}
Since \( \Cmeas_{\min}(N_0) \) records the empirical minimum in each interval, showing
\begin{equation}
\min_{N_0 \ge n_{5\%}} \Delta(N_0) > 0
\end{equation}
is sufficient to ensure that the analytic bound lies strictly below all observed minima for \( n \ge n_{5\%} \).

In our dataset, the smallest observed value of the dominance gap
\begin{equation}
\Delta(N_0)\;:=\;\frac{\Cmeas_{\min}(N_0)}{M}\;-\;\frac{\Kem^2}{\log^2 N_0}
\end{equation}
occurs at
\begin{equation}
N_0=\xNzeroStat,\qquad \Delta(N_0)=\xEtaStat>0.
\end{equation}
At the explicit Mertens threshold \(N_0=\xMertens\) one has
\(\Delta(\xMertens)=\xDeltaMertens>0\).
Consequently \(\Delta(N_0)>0\) for all \(N_0\ge \xMertens\), so the analytic
lower bound lies strictly below all observed minima throughout the verified range.

We record one minimum per \emph{decimal block} of the form
\([d\cdot 10^{k},(d+1)\cdot 10^{k}-1]\) for integers \(k\ge 4\) and \(1\le d\le 9\),
with the block width scaling by a factor of \(10\) when \(k\) increases (e.g.,
\(10000\text{–}19999\), \(20000\text{–}29999\), \(\dots\), then \(100000\text{–}199999\), \(\dots\)).
Consequently, from the observed minimum at \(\nprodstar\) in the block
\(6000\text{–}6999\), we can assert that no smaller \( \Delta \) occurs within that block.
In the preceding block \(5000\text{–}5999\) the recorded minimum is at \(N_0=\xPreMertens\);
since we store only one minimum per block, we cannot exclude the possibility of a
(strictly positive) smaller value at some \(N_0\in[\xPreMertens,5999]\). Thus taking
\(\nprodstar=\xMertens\) as the permanence threshold is conservative: it may occur
slightly later than the true last crossing, but it guarantees that for all
\(n\ge \nprodstar\) the empirical minima dominate the analytic bound.

\bigskip
\noindent
Thus, given the definition of \( \CminusProduct(n) \) in Equation~\ref{eq:cminus-product-definition} we conclude,

The constant \( \nprodstar \) exists such that, for all \(n \ge \nprodstar\),
  \begin{equation}
    \mathcal{G}(n;M)\;\ge\;\frac{\CminusProduct(n)M(n)}{log^2 n},
    \qquad
    \text{with } M(n)=\bigl\lfloor \tfrac{n}{2} \bigr\rfloor
    \ \text{ and }\
    \nprodstar=\xMertens .
    \label{eq:analytic-lower-bound-global}
  \end{equation}
\end{proof}

\begin{remark}[Tail thresholds: product vs.\ asymptotic]
\label{rem:tail-thresholds}
Let \(\nprodstar\) denote the product–form threshold that appears in Theorem~\ref{thm:main};
in our macros we set \(\nprodstar=\xMertens\).
Define the (slightly larger) asymptotic–surrogate threshold \(\nasymstar\) by
\begin{definition}[Asymptotic–surrogate dominance threshold (blockwise)]\label{def:nasym-block}
\begin{equation}
\nasymstar \ :=\ \min\Big\{\,N_0\in\mathcal B:\ 
\Cmeas_{\min}(N_0')\ \ge\ \CminusAsymp(N_0')\ \ \text{for all } N_0'\in\mathcal B,\ N_0'\ge N_0\Big\}.
\end{equation}
\end{definition}
In our dataset, \(\nasymstar=\xNasymStar\).
\end{remark}

\noindent\textbf{Window Scalability.}
Specializing to \(\alpha_0=\tfrac12\) above, Lemma~\ref{lem:alpha-rescale} gives the same certified lower bound for every \(\alpha\in(0,\tfrac12]\) with the natural right–edge cutoff \(\sqrt{n+\alpha n}\). 
By monotonicity in the window, Corollary~\ref{cor:alpha-superset} further implies \(\mathcal{G}(n;\alpha n)\ge \mathcal{G}(n;\tfrac12 n)\) for all \(\alpha\in[\tfrac12,1)\).

\subsection{Conclusion}\label{sec:conclusion}
This author established an explicit, certified sieve–theoretic lower bound for (windowed) Goldbach counts by applying an Eratosthenes–type sieve directly to the quadratic form \(Q(n,m)=(n-m)(n+m)\). The bound is given as a product of conservative per–prime Euler factors and holds uniformly for large \(n\) while the sieve cutoff \(z\) remains below the prime–forcing threshold \(n^{\frac{1}{2}}\), so the classical parity obstruction does not arise.

Exhaustive computation up to \(2n=2\nprodstar\) confirms that every even integer in this range is representable. Beyond that range, the certified lower bound remains strictly below the observed decade–wise minima by a uniform positive margin. Moreover, after normalization by the Hardy–Littlewood main term, the windowed counts agree with the heuristic to within \(<1\%\) throughout \(n < 10^{8}\), indicating rapid convergence and a stable singular–series normalization.

Taken together, these ingredients give a precise reduction: to push the sieve to the prime–forcing cutoff it suffices to assume a short–interval Bombieri–Vinogradov–type equidistribution for primes (as stated in the conditional corollary). Under that hypothesis one obtains a positive lower bound for all sufficiently large even integers; combined with our verification up to \(2\nprodstar\), this settles all cases.

Unconditionally, the paper contributes (i) a rigorous lower bound with explicit constants, free of tail and binning artefacts; (ii) a reproducible computation to the stated limit; and (iii) a clear reduction of the remaining analytic task to a standard short–interval distribution problem, strictly weaker than assuming the full Hardy–Littlewood asymptotic. The available data strongly support the predicted main term, and the remaining hypothesis is sharply circumscribed.

\subsection{Conditional Corollary (short-interval equidistribution)}\label{sec:SI-BV}
\begin{corollary}[Unconditional Reduction; Conditional Consequence Under Short–Interval Equidistribution]
\label{cor:SI-BV}
For \(x\ge 3\), \(q\in\mathbb{N}\), \((a,q)=1\), and \(H>0\), write
\begin{equation}\label{eq:pi-AP}
\pi(x; q,a)\ :=\ \#\{\,p\le x:\ p\ \text{prime},\ p\equiv a\!\!\pmod q\,\}.
\end{equation}
Assume the short–interval Bombieri–Vinogradov hypothesis: there exist \(\theta>\tfrac{1}{2}\) and \(\varepsilon>0\) such that, for every \(A>0\),
\begin{equation}\label{eq:SIBV}
\sum_{q\le x^{\theta}}
\ \max_{\substack{(a,q)=1}}\ \max_{x'\le x}\ \max_{H\ge x^{\frac{1}{2}+\varepsilon}}
\Bigl|\ \pi(x'+H;q,a)\ -\ \frac{H}{\varphi(q)\log x'}\ \Bigr|
\ \ \ll_{A,\varepsilon}\ \ \frac{x}{(\log x)^A}.
\end{equation}
Let \(R_2(N)\) denote the number of \emph{ordered} representations \(N=p_1+p_2\) with \(p_1,p_2\) prime, and let the (binary Goldbach) singular series be
\begin{equation}\label{eq:sing-series}
\mathfrak S(N)\ :=\ 2\prod_{p\ge 3}\Bigl(1-\frac{1}{(p-1)^2}\Bigr)\ \prod_{\substack{p\mid N\\ p\ge 3}}\frac{p-1}{p-2}
\ =\ 2C_2\ \prod_{\substack{p\mid N\\ p\ge 3}}\frac{p-1}{p-2},
\end{equation}
where \(C_2=\prod_{p\ge 3}\bigl(1-\frac{1}{(p-1)^2}\bigr)\) is the twin–prime constant. Then \(\mathfrak S(N)\ge 2C_2\) for all even \(N\).

Assuming \eqref{eq:SIBV}, there exists \(N_0\) such that every even \(N\ge N_0\) satisfies \(R_2(N)>0\). Together with our exhaustive verification up to \(2\nprodstar\), this implies that every even integer \(>2\) is a Goldbach number.
\end{corollary}

\begin{proof}
Let \(N\) be large and even. Define
\begin{equation}\label{eq:S-alpha}
e(t):=e^{2\pi i t},\qquad
S(\alpha):=\sum_{n\le N}\Lambda(n)\,e(n\alpha).
\end{equation}
Then
\begin{equation}\label{eq:R2-int-again}
R_2(N)\ =\ \int_0^1 S(\alpha)^2\,e(-N\alpha)\,d\alpha.
\end{equation}
Fix \(Q:=N^{\frac{1}{2}-\delta}\) with \(0<\delta<\theta-\tfrac12\).  Next, split the integral over the frequency variable \(\alpha\in[0,1]\) as
\([0,1]=\mathfrak M\cup\mathfrak m\) (Definition~\ref{def:S-and-arcs}): 
\(\mathfrak M\) is the union of small neighborhoods of rationals \(a/q\) with \(q\le Q\) (the \emph{major arcs}), and \(\mathfrak m\) is the complementary set (the \emph{minor arcs}).

On \(\mathfrak M\), evaluate the integral and obtain the main term \(\mathfrak S(N)N/\log^2 N\); 
and show on \(\mathfrak m\) the integral is \(o\!\big(N/\log^2 N\big)\) under \eqref{eq:SIBV}.

\medskip\noindent\emph{Major Arcs.}
Standard evaluation (see, e.g., \cite[Thm.~13.12]{MontgomeryVaughan2007}) gives
\begin{equation}\label{eq:major-arc-again}
\int_{\mathfrak M} S(\alpha)^2 e(-N\alpha)\,d\alpha
=\frac{\mathfrak S(N)\,N}{\log^2 N}\ +\ O\!\Big(\frac{N}{\log^3 N}\Big),
\end{equation}
with \(\mathfrak S(N)\) as in \eqref{eq:sing-series}. Since \(\frac{p-1}{p-2}>1\) for each odd \(p\mid N\), we have \(\mathfrak S(N)\ge 2C_2\).

\medskip\noindent\emph{Minor arcs under \eqref{eq:SIBV}.}
Applying Vaughan’s identity to \(\Lambda\) in \(S(\alpha)\) and splitting at admissible \(U,V\) (e.g.\ \(U=N^{1/3}\)), we obtain Type~I/II sums. For \(\alpha\in\mathfrak m\) with \(\bigl|\alpha-\tfrac{a}{q}\bigr|\ge (qQ)^{-1}\) (\(q\le Q\)), Cauchy–Schwarz and the large sieve bound these by mean–square discrepancies of primes in progressions over short intervals of length \(H\asymp N^{\frac{1}{2}+\varepsilon}\). The short–interval hypothesis \eqref{eq:SIBV} (with \(\theta>\tfrac12\)) then yields, for every \(A>0\),
\begin{equation}\label{eq:major-arc-again}
\int_{\mathfrak m} S(\alpha)^2 e(-N\alpha)\,d\alpha\ \ll_{A,\varepsilon}\ \frac{N}{(\log N)^A}.
\end{equation}
(See the dispersion/large–sieve treatment in \cite[Chs.~17–18]{IwaniecKowalski2004} or \cite[Chs.~17,\,28]{Harman2007}; the short–interval input replaces the classical BV step.)

\medskip
Combining \eqref{eq:R2-int-again}, \eqref{eq:major-arc-again}, and \eqref{eq:major-arc-again},
\begin{equation}
R_2(N)\ \ge\ \frac{(2C_2)\,N}{\log^2 N}\ -\ \Kminor\,\frac{N}{(\log N)^A}
\end{equation}
for some \(\Kminor=\Kminor(A,\varepsilon)\). Choosing \(A\ge 3\) and \(N_0\) so that \(\frac{2C_2}{\log^2 N_0}>\frac{\Kminor}{(\log N_0)^A}\) gives \(R_2(N)>0\) for all even \(N\ge N_0\). The exhaustive computation up to \(2\nprodstar\) covers the remaining \(N< N_0\).
\end{proof}

\begin{remark}[Weaker Sufficient Inputs for the Reduction]
\label{rem:weaker-inputs}
The corollary is proved once one has a minor--arc bound of the form
\begin{equation}\label{eq:major-arc-again}
\int_{\mathfrak m} S(\alpha)^2 e(-N\alpha)\,d\alpha\ \ll_{A,\varepsilon}\ \frac{N}{(\log N)^A}
\quad\text{for some }A>2,
\end{equation}
with \(S(\alpha)\) as in \eqref{eq:S-alpha}. The full hypothesis \eqref{eq:SIBV} is a convenient sufficient condition for \eqref{eq:major-arc-again}, but it is not necessary. Any of the following implies \eqref{eq:major-arc-again} and hence the corollary:

\begin{enumerate}
\item[\textnormal{(i)}] \emph{Short-interval BDH/\(L^2\)-type estimate.}
There exist \(\delta,\varepsilon>0\) such that, for every \(A>0\),
\begin{equation}
\sum_{q\le N^{\frac{1}{2}+\delta}}\ \sum_{\substack{a\bmod q\\(a,q)=1}}
\ \max_{H\ge N^{\frac{1}{2}+\varepsilon}}\ 
\int_{N}^{2N}\bigl|\pi(x+H;q,a)-\tfrac{H}{\varphi(q)\log x}\bigr|^2\,dx
\ \ll_{A,\varepsilon}\ \frac{N^2}{(\log N)^A}.
\end{equation}
Via Vaughan’s identity, Cauchy–Schwarz and the large sieve, this delivers \eqref{eq:major-arc-again}.

\item[\textnormal{(ii)}] \emph{Almost-everywhere short-interval equidistribution.}
For some \(\delta,\varepsilon>0\) and every \(A>0\), all but \(O\!\big(N/(\log N)^A\big)\) starting points \(x\in[N,2N]\) satisfy
\[
\max_{q\le N^{\frac{1}{2}+\delta}}\ \max_{(a,q)=1}\ \max_{H\ge N^{\frac{1}{2}+\varepsilon}}
\bigl|\pi(x+H;q,a)-\tfrac{H}{\varphi(q)\log x}\bigr|\ \ll\ \frac{H}{(\log N)^{A}}.
\]
This yields \eqref{eq:major-arc-again} after integrating over \(x\) and summing dyadically.

\item[\textnormal{(iii)}] \emph{Any stronger hypothesis implying (i) or (ii)} (e.g. a GEH/EH-type statement in short intervals, or BV in short intervals for a rich class of moduli together with a standard dispersion argument).
\end{enumerate}

Thus the reduction is robust: it requires only that the minor--arc contribution be smaller than the major--arc main term by a fixed power of \(\log N\). The full SI--BV\(_\theta\) statement \eqref{eq:SIBV} is one natural way to guarantee this, but strictly weaker inputs suffice.
\end{remark}

\begin{remark}[Scope, logical independence, and status of the reduction]
\label{rem:scope-reduction}
All bounds stated as theorems in this paper are \emph{unconditional}. In particular, the certified windowed lower bound
(Theorem~\ref{thm:main}; cf.\ \eqref{eq:analytic-lower-bound-global} with the product defined in \eqref{eq:cminus-product-definition})
is proved via a one–sided sieve on \(n\pm m\) with explicit Euler–product factors and a finite edge term; its validity is
independent of any circle–method or distributional hypothesis. The accompanying computations are exhaustive on the stated range.

Separately, Corollary~\ref{cor:SI-BV} is an \emph{unconditional reduction}: it proves the implication
\begin{equation}
\text{\eqref{eq:SIBV}}\ \Longrightarrow\ \text{Goldbach for all sufficiently large even }N,
\end{equation}
without further assumptions. The corollary is “conditional” only in the sense that the antecedent \eqref{eq:SIBV} is not established here.

Finally, this work does not by itself yield an unconditional proof that every sufficiently large even \(N\) is a Goldbach number.
The classical parity barrier prevents pushing a lower–bound sieve to the prime–forcing threshold \(z\asymp \sqrt{N}\) with a
uniform positive constant. Thus a full resolution requires additional short–interval \emph{equidistribution} input of the
SI–BV type; this author's contribution is to isolate this precise reduction while providing a certified sieve bound and comprehensive data
that are logically independent of it.
\end{remark}

\begin{remark}[How the reduction is used]
\label{rem:reduction-use}
Assume the minor–arc input \eqref{eq:major-arc-again} with some \(A>2\) (e.g. under SI--BV\(_\theta\)).

(i) \emph{Positivity.} Since \(\mathfrak S(N)\ge \Szero=2C_2\) and
\[
R_2(N)\ =\ \frac{\mathfrak S(N)\,N}{\log^2 N}\ +\ O\!\Big(\frac{N}{(\log N)^A}\Big),
\]
there exists \(N_0\) with \(R_2(N)>0\) for all even \(N\ge N_0\).

(ii) \emph{Tail from the certified \emph{product–form} bound.}
From Theorem~\ref{thm:main}, for \(n\ge \nprodstar\) and a fixed window \(M(n)\asymp n\) (e.g. \(M(n)=\lfloor n/2\rfloor\)),
\begin{equation}\label{eq:prod-tail}
\mathcal G(n;M)\ \ge\ \frac{\CminusProduct(n)}{\log^2 n}\,M(n),
\qquad
\CminusProduct(n)=\log^2 n\!\!\prod_{3\le p\le \sqrt n}\!\Bigl(1-\frac{1}{p-1}\Bigr)
\!\!\prod_{3\le p\le \sqrt{n+M(n)}}\!\Bigl(1-\frac{1}{p-1}\Bigr).
\end{equation}
Set \(\displaystyle \kappaprod:=\inf_{n\ge \nprodstar}\CminusProduct(n)>0\) (the recorded positive margin).
Then for all \(N\ge 2\nprodstar\),
\[
R_2(N)\ \ge\ \frac{c_{\mathrm{prod}}\,N}{\log^2 N},
\qquad
c_{\mathrm{prod}}=c\big(\Szero,\kappaprod\big)>0,
\]
by a dyadic decomposition and the comparison \(M(n)\asymp n\).

\smallskip
\emph{Note (asymptotic surrogate).}
If, instead of \eqref{eq:prod-tail}, you choose to work with the asymptotic surrogate
\(\CminusAsymp(n)\) (replacing the products by their Mertens asymptotics involving \(\Kem\)), use a slightly larger threshold \(\nasymstar\ge \nprodstar\) so that \(\CminusAsymp(n)\le \CminusProduct(n)\) for all \(n\ge\nasymstar\). The same conclusion then holds for all \(N\ge 2\nasymstar\) with a (possibly smaller) constant \(c_{\mathrm{asym}}>0\).
\end{remark}

\begin{definition}[Exponential sum and major/minor arcs]
\label{def:S-and-arcs}
Set \(e(t):=e^{2\pi i t}\) and
\begin{equation}\label{eq:S-alpha}
S(\alpha):=\sum_{n\le N}\Lambda(n)\,e(n\alpha),\qquad \alpha\in[0,1].
\end{equation}
Fix \(Q:=N^{1/2-\delta}\) with \(0<\delta<\theta-\tfrac12\). For each reduced fraction \(a/q\) with \(1\le q\le Q\) and \((a,q)=1\), define the major arc
\[
\mathfrak M(q,a):=\Bigl\{\alpha\in[0,1]:\ \bigl|\alpha-\tfrac{a}{q}\bigr| \le \tfrac{1}{2qQ}\Bigr\}.
\]
Let \(\mathfrak M:=\bigcup_{1\le q\le Q}\ \bigcup_{(a,q)=1}\ \mathfrak M(q,a)\) and \(\mathfrak m:=[0,1]\setminus \mathfrak M\).
\end{definition}

\appendix
\counterwithin{equation}{section}  % makes equations A.1, A.2, ...
\numberwithin{lemma}{section}
\numberwithin{conjecture}{section}
\numberwithin{corollary}{section}
\numberwithin{definition}{section}

\section{Motivating Conjecture}\label{app:motivation}

\begin{definition}[Admissible selections in the \( (n,m) \) grid]
\label{def:admissible}
Fix parameters \( \alpha,\delta\in(0,1) \). For \( N\ge 1 \) set
\begin{equation}
\mathcal{R}_N:=\{(n,m)\in\mathbb{Z}^2:\ N\le n\le (1+\delta)N,\ |m|\le \alpha n,\ m\equiv n \ (\mathrm{mod}\ 2)\}.
\end{equation}
A family \( \{\mathcal{S}_N\}_{N\ge 1} \) with \( \mathcal{S}_N\subset\mathcal{R}_N \) is \emph{admissible} if:
\begin{enumerate}
\item[(A1)] (\emph{Low complexity}) There is a fixed polynomial \( F\in\mathbb{Z}[X,Y] \) of bounded degree, independent of \( N \), such that
\( \mathcal{S}_N \subseteq \{(n,m)\in\mathcal{R}_N:\ F(n,m)=0\} \).
\item[(A2)] (\emph{Linear size}) \( \#\mathcal{S}_N \asymp N \) (i.e., \( \exists\,c_0,C_0>0 \) with \( c_0 N\le \#\mathcal{S}_N\le C_0 N \) for large \( N \)).
\item[(A3)] (\emph{Nondegenerate}) \( F(n,m)=0 \) has infinitely many integer points with \( |m|\le n \) and is not contained in \( |m|=n \).
\end{enumerate}
For such \( \mathcal{S}_N \), define the prime-pair count
\begin{equation}
\Pi(\mathcal{S}_N):=\#\{(n,m)\in\mathcal{S}_N:\ n\pm m\ \text{are both prime}\}.
\end{equation}
\end{definition}

\begin{conjecture}[Uniform Prime-Pair Density With Path–Dependent Decay (motivating observation)]\label{conj:uniform-C}
Let \( \{\mathcal{S}_N\} \) be an admissible family (low–degree algebraic path in the \( (n,m) \)–grid with \( |m|\le n \) and \( \#\mathcal{S}_N\asymp N \)).  (Definition~\ref{def:admissible})
Let \( \Bref(y) \) be the reference Brun–type product
(Definition~\ref{def:Bref}; cf.~\cite[§1.6]{HalberstamRichert1974}, \cite[Ch.~4]{Riesel1994}),
with \( y\asymp\sqrt{N} \).
Then there exist \( N_0 \), path–dependent constants \( C_{\min}(y),C_{\max}(y)>0 \), and exponents
\( k_{\min},k_{\max}\in[1,2] \) with \( k_{\max} \le k_{\min} \) such that for all \( N\ge N_0 \),
\begin{equation}
C_{\min}(y)\,\#\mathcal{S}_N\,\Bref(y)
\ \ll\
\Pi(\mathcal{S}_N)
\ \ll\
C_{\max}(y)\,\#\mathcal{S}_N\,\Bref(y),
\end{equation}
and, in particular,
\begin{equation}
\#\mathcal{S}_N\,\frac{1}{\log^{k_{\min}} N}
\ \ll\
\Pi(\mathcal{S}_N)
\ \ll\
\#\mathcal{S}_N\,\frac{1}{\log^{k_{\max}} N}.
\end{equation}
\emph{Heuristic center.}\quad
\( \Pi(\mathcal{S}_N)\approx \Cpred_\tavg(y)\,\#\mathcal{S}_N\,\Bref(y) \).
\end{conjecture}

\begin{remark}[Examples]
(i) \emph{Goldbach window} (\( F(n,m)=n-N \)): \( \#\mathcal{S}_N\asymp N \) and the bounds give \( \Pi(\mathcal{S}_N)\asymp N/\log^2 N \).\\
(ii) \emph{Fixed gap} \( g=2|m_0| \) (\( F(n,m)=m-m_0 \)): \( \#\mathcal{S}_N\asymp N \) yields the twin/cousin/etc.\ densities \( \asymp N/\log^2 N \).\\
(iii) \emph{Lines/curves} (\( F(n,m)=m-an-b \) with \( |a|<1 \), or other bounded-degree \( F \)): same conclusion.
\end{remark}

\begin{remark}[Scope]
Conjecture~\ref{conj:uniform-C} motivates the windowed sieve setup only; no theorem, lemma, or corollary in this paper depends on it. Unconditional results use sieve bounds and Euler–Mertens products; the only conditional input appears in Corollary~\ref{cor:SI-BV}.
\end{remark}

\needspace{10\baselineskip}

\section{Certified Enclosures for Euler Products}\label{app:enclosures}

\subsection{Shifted Product Enclosure}\label{app:shifted-product}

\begin{lemma}[Certified Enclosure for the Shifted Product]\label{lem:shifted-enclosure}
Define
\begin{equation}\label{eq:P-def}
P(x)\ :=\ \prod_{\substack{3\le p\le x\\ p\ \mathrm{prime}}}\Bigl(1-\frac{1}{p-1}\Bigr).
\end{equation}
There exists \(x_0\) such that for all \(x\ge x_0\),
\begin{equation}\label{eq:shifted-enclosure}
\Bigl|\,\log x\cdot P(x)\ -\ C_-^{(1)}\,\Bigr|\ \le\ \varepsilon_P(x),
\end{equation}
where \(C_-^{(1)}=e^{-\gamma}C_2\) and
\begin{equation}\label{eq:epsP-def}
\varepsilon_P(x)\ :=\ C_2\,E_M(x)\ +\ e^{-\gamma}\,T(x)\ +\ E_M(x)\,T(x).
\end{equation}
Here \(E_M(x)\) and \(T(x)\) are explicit, strictly decreasing functions given in
\eqref{eq:EM-def} and \eqref{eq:T-def} below.
\end{lemma}

\begin{proof}
For \(p\ge 3\),
\begin{equation}\label{eq:factor-identity}
1-\frac{1}{p-1}
\ =\
\Bigl(1-\frac{1}{p}\Bigr)\,\Bigl(1-\frac{1}{(p-1)^2}\Bigr),
\end{equation}
so with
\begin{equation}\label{eq:M-and-C2-partial}
M(x):=\prod_{p\le x}\Bigl(1-\frac{1}{p}\Bigr),
\qquad
C_2(x):=\prod_{\substack{3\le p\le x\\ p\ \mathrm{prime}}}\Bigl(1-\frac{1}{(p-1)^2}\Bigr),
\end{equation}
we have \(P(x)=M(x)\,C_2(x)\).

For Mertens’ product we use the explicit enclosure
\begin{equation}\label{eq:EM-def}
\Bigl|\,\log x\cdot M(x)\ -\ e^{-\gamma}\,\Bigr|\ \le\ E_M(x),
\qquad x\ge x_0,
\end{equation}
with \(E_M(x)\) strictly decreasing to \(0\).  
For the twin–prime factor we use the monotone tail bound
\begin{equation}\label{eq:T-def}
0\ <\ C_2 - C_2(x)\ \le\ T(x),
\qquad
T(x):=\sum_{p>x}\frac{1}{(p-1)^2},
\end{equation}
which is strictly decreasing in \(x\) and satisfies \(T(x)\le \sum_{n>\,x}\frac{1}{(n-1)^2}\le \frac{1}{x-1}\).

Write \(C_2(x)=C_2-\delta(x)\) with \(0\le \delta(x)\le T(x)\).
Then
\begin{equation}\label{eq:linearization-step}
\log x\cdot P(x)
=\bigl(\log x\cdot M(x)\bigr)\,C_2(x)
=\bigl(e^{-\gamma}\pm E_M(x)\bigr)\,\bigl(C_2-\delta(x)\bigr).
\end{equation}
Expanding and bounding the error terms gives
\begin{equation}\label{eq:collect-errors}
\Bigl|\,\log x\cdot P(x)-e^{-\gamma}C_2\,\Bigr|
\ \le\ C_2\,E_M(x)\ +\ e^{-\gamma}\,\delta(x)\ +\ E_M(x)\,\delta(x)
\ \le\ C_2\,E_M(x)\ +\ e^{-\gamma}\,T(x)\ +\ E_M(x)\,T(x),
\end{equation}
which is \eqref{eq:shifted-enclosure}–\eqref{eq:epsP-def}.  
The monotonicity of \(E_M,T\) makes \(\varepsilon_P\) strictly decreasing as well.
\end{proof}

\needspace{10\baselineskip}

\subsection{Mertens Product Enclosure}\label{app:mertens}

\begin{lemma}[Explicit Mertens enclosure\cite{RosserSchoenfeld1962, Dusart2010}]\label{lem:explicit-mertens}
There exists \( x_0 \) (e.g.\ \( x_0=\xMertens \)) such that for all \( x\ge x_0 \),
\begin{equation}
e^{-\gamma}\frac{1}{\log x}\!\left(1-\frac{1}{20\log^3 x}-\frac{316}{\log^4 x}\right)
\le M(x) \le
e^{-\gamma}\frac{1}{\log x}\!\left(1+\frac{1}{20\log^3 x}+\frac{3}{16\log^4 x}
+\frac{1.02}{(x-1)\log x}\right),
\end{equation}
hence
\begin{equation}
\bigl|\,\log x\cdot M(x)-e^{-\gamma}\,\bigr|\ \le\ e^{-\gamma}\,E_M(x),
\end{equation}
with
\begin{equation}
E_M(x) := \frac{1}{20\log^3 x}+\max\!\left\{\frac{316}{\log^4 x},\ \frac{3}{16\log^4 x}+\frac{1.02}{(x-1)\log x}\right\}.
\end{equation}
\end{lemma}

\subsection{\(C_2\) Tail Bound}\label{app:C2-tail}
\begin{lemma}[Certified Tail for \( C_2 \)]\label{lem:C2-tail}
For all \( x\ge 3 \),
\begin{equation}
0\ \le\ 1-\frac{C_2(x)}{C_2}\ \le\ T(x),
\qquad
T(x) := \frac{1}{x-1}+\frac{1}{3(x-1)^3},
\end{equation}
so \( |C_2(x)-C_2|\le T(x) \).
\end{lemma}

Combining the lemmas,
\begin{equation}
\bigl|\,\log x\cdot P(x) - C_-^{(1)}\,\bigr|
\le e^{-\gamma}C_2\,E_M(x) + e^{-\gamma}\,T(x) = \varepsilon_P(x),
\end{equation}
which is explicit and strictly decreasing for \( x\ge x_0 \).

\needspace{10\baselineskip}

\subsection{Application to the Lower Bound Product}
Let
\begin{equation}
F_1(n)=\log n\cdot P(\sqrt{n}),\qquad
F_2(n)=\log\!\Bigl(\tfrac{3n}{2}\Bigr)\cdot P\!\Bigl(\sqrt{\tfrac{3n}{2}}\Bigr),
\qquad
\widehat C_- := \bigl(e^{-\gamma}C_2\bigr)^2 .
\end{equation}
Let \(x_1=\sqrt{n}\) and \(x_2=\sqrt{\frac{3n}{2}}\).
By Appendix~\ref{lem:shifted-enclosure}, for \( i=1,2 \),
\begin{equation}
\bigl|\,F_i(n)-C_-^{(1)}\,\bigr|\ \le\ \varepsilon_P(x_i),
\end{equation}
and hence
\begin{equation}
\bigl|\,F_1(n)F_2(n)-\widehat C_-\,\bigr|
\ \le\ \widehat C_-\,\bigl(\varepsilon_P(x_1)+\varepsilon_P(x_2)\bigr)
\ +\ \varepsilon_P(x_1)\varepsilon_P(x_2)
\ :=\ \varepsilon(n),
\end{equation}
with \( \varepsilon(n) \) explicit and strictly decreasing in \( n \).

\section{Window rescaling}\label{app:window-rescale}

\begin{lemma}[Window Rescaling Without Re-Certification]
\label{lem:alpha-rescale}
Fix \(\alpha_0\in(0,1)\) and suppose the certified lower bound
\begin{equation}\label{eq:alpha0-certified}
\mathcal G(n;\,\alpha_0 n)\ \ge\ \frac{\CminusProductAlpha{n}{\alpha_0}}{\log^2 n}\,(\alpha_0 n)
\end{equation}
holds for all sufficiently large \(n\), where
\begin{equation}\label{eq:Cminus-alpha-def}
\CminusProductAlpha{n}{\alpha}
:=\log^2 n\,
\prod_{3\le p\le \sqrt n}\!\Bigl(1-\frac{1}{p-1}\Bigr)\,
\prod_{3\le p\le \sqrt{\,n+\alpha n\,}}\!\Bigl(1-\frac{1}{p-1}\Bigr).
\end{equation}
Then for every \(\alpha\in(0,\alpha_0]\) and all sufficiently large \(n\),
\begin{equation}\label{eq:alpha-certified}
\boxed{\ \ \mathcal G(n;\,\alpha n)\ \ge\ \frac{\CminusProductAlpha{n}{\alpha}}{\log^2 n}\,(\alpha n)\ .\ }
\end{equation}
\end{lemma}

\begin{proof}
Summing the one–sided lower bounds over \(|m|\le \alpha n\) proceeds exactly as in the \(\alpha_0\) case. Shrinking the window reduces the number of offsets linearly by \(\alpha/\alpha_0\), while
\begin{equation}\label{eq:right-edge-shift}
\sqrt{\,n+\alpha n\,}\ \le\ \sqrt{\,n+\alpha_0 n\,}
\end{equation}
tightens the right–edge cutoff in the second Euler product, which can only \emph{increase} the conservative product in \eqref{eq:Cminus-alpha-def}. Hence the same certification yields \eqref{eq:alpha-certified} for all \(\alpha\le\alpha_0\).
\end{proof}

\begin{corollary}[Monotone Extension to Larger Windows]
\label{cor:alpha-superset}
Under the hypotheses of Lemma~\ref{lem:alpha-rescale}, for every \(\alpha\in[\alpha_0,1)\) and all sufficiently large \(n\),
\begin{equation}\label{eq:alpha-superset}
\boxed{\ \ \mathcal G(n;\,\alpha n)\ \ge\ \mathcal G(n;\,\alpha_0 n)\ \ge\ \frac{\CminusProductAlpha{\alpha_0}{n}}{\log^2 n}\,(\alpha_0 n)\ .\ }
\end{equation}
\end{corollary}

\begin{proof}
Monotonicity in the window is immediate from the set inclusion
\begin{equation}\label{eq:window-inclusion}
\{\,|m|\le \alpha_0 n\,\}\ \subseteq\ \{\,|m|\le \alpha n\,\},
\end{equation}
which implies \(\mathcal G(n;\,\alpha n)\ge \mathcal G(n;\,\alpha_0 n)\). The second inequality in \eqref{eq:alpha-superset} is exactly \eqref{eq:alpha0-certified}.
\end{proof}

\begin{remark}[Uniform-in-\(\alpha\) certification]
\label{rem:alpha-uniform}
Because the one–sided sieve factors \(S_\pm(n,m)\) are pointwise in \(m\), the same argument that proves \eqref{eq:alpha0-certified} works verbatim for each fixed \(\alpha\in(0,1)\); in particular, for all sufficiently large \(n\),
\begin{equation}\label{eq:alpha-uniform}
\mathcal G(n;\,\alpha n)\ \ge\ \frac{\CminusProductAlpha{\alpha}{n}}{\log^2 n}\,(\alpha n).
\end{equation}
No re-tuning of sieve weights is required; only the right–edge cutoff \(\sqrt{\,n+\alpha n\,}\) in \(\CminusProductAlpha{\alpha}{n}\) changes.
\end{remark}

\needspace{10\baselineskip}

\clearpage

\section{Decadal Statistics for Goldbach Pair Distribution}\label{app:decadal}

\needspace{8\baselineskip}

\setlength{\LTpre}{0pt}
\setlength{\LTpost}{0pt}

\small
\sisetup{
  group-separator = {\,},
  group-minimum-digits = 4,
  round-mode=places
}
\begin{longtable}{
  S[table-format=1,round-precision=0] % Dec.
  S[table-format=7,round-precision=0] % Min At
  S[table-format=7,round-precision=0] % Min
  S[table-format=7,round-precision=0] % Max At
  S[table-format=7,round-precision=0] % Max
  S[table-format=7,round-precision=0] % <n_geom>
  S[table-format=7.1,round-precision=1] % <Count>
}
\caption[]{Per-decade Statistics for Goldbach Pair Counts for \( |m| \in [ 1, \lfloor \frac{n}{2} \rfloor ) \)}\label{tab:raw-stats} \\
\toprule
\multicolumn{1}{c}{\textbf{Dec.}} & 
\multicolumn{1}{c}{\textbf{Min At}} & 
\multicolumn{1}{c}{\textbf{Min}} & 
\multicolumn{1}{c}{\textbf{Max At}} & 
\multicolumn{1}{c}{\textbf{Max}} & 
\multicolumn{1}{c}{ \(\Ngeom \)} & 
\multicolumn{1}{c}{\( \langle \text{Count} \rangle \)} \\
\midrule
\endfirsthead

\multicolumn{7}{l}{\textit{(continued)}}\\
\toprule
\multicolumn{1}{c}{\textbf{Dec.}} & 
\multicolumn{1}{c}{\textbf{Min At}} & 
\multicolumn{1}{c}{\textbf{Min}} & 
\multicolumn{1}{c}{\textbf{Max At}} & 
\multicolumn{1}{c}{\textbf{Max}} & 
\multicolumn{1}{c}{ \(\Ngeom\)} & 
\multicolumn{1}{c}{\( \langle \text{Count} \rangle \)} \\
\midrule
\endhead
0 & 4 & 2 & 4 & 2 & 4 & 2.000000 \\
0 & 5 & 2 & 5 & 2 & 5 & 2.000000 \\
0 & 6 & 2 & 6 & 2 & 7 & 2.000000 \\
0 & 7 & 0 & 7 & 0 & 7 & 0.000000 \\
0 & 8 & 2 & 8 & 2 & 9 & 2.000000 \\
0 & 9 & 4 & 9 & 4 & 9 & 4.000000 \\
1 & 11 & 0 & 12 & 4 & 15 & 2.200000 \\
1 & 22 & 2 & 21 & 6 & 25 & 3.200000 \\
1 & 31 & 2 & 30 & 8 & 35 & 4.200000 \\
1 & 43 & 0 & 45 & 10 & 45 & 4.200000 \\
1 & 53 & 2 & 57 & 10 & 55 & 5.800000 \\
1 & 61 & 2 & 60 & 12 & 65 & 6.000000 \\
1 & 79 & 2 & 75 & 14 & 75 & 7.800000 \\
1 & 82 & 4 & 81 & 10 & 85 & 7.000000 \\
1 & 97 & 2 & 90 & 12 & 95 & 7.800000 \\
2 & 107 & 4 & 195 & 26 & 141 & 10.640000 \\
2 & 223 & 4 & 210 & 30 & 245 & 14.720000 \\
2 & 302 & 8 & 315 & 40 & 347 & 19.060000 \\
2 & 433 & 8 & 495 & 50 & 447 & 22.980000 \\
2 & 508 & 14 & 570 & 56 & 547 & 26.100000 \\
2 & 601 & 14 & 660 & 62 & 649 & 29.780000 \\
2 & 706 & 14 & 735 & 72 & 749 & 33.700000 \\
2 & 802 & 16 & 840 & 76 & 849 & 36.580000 \\
2 & 919 & 18 & 975 & 78 & 949 & 38.260000 \\
3 & 1009 & 20 & 1995 & 148 & 1415 & 54.936000 \\
3 & 2029 & 30 & 2730 & 208 & 2449 & 80.364000 \\
3 & 3076 & 44 & 3990 & 250 & 3465 & 103.686000 \\
3 & 4051 & 60 & 4830 & 310 & 4473 & 126.344000 \\
3 & 5416 & 72 & 5775 & 358 & 5477 & 146.604000 \\
3 & 6353 & 88 & 6930 & 424 & 6481 & 169.544000 \\
3 & 7219 & 94 & 7770 & 442 & 7483 & 187.038000 \\
3 & 8116 & 112 & 8925 & 520 & 8485 & 206.446000 \\
3 & 9014 & 124 & 9975 & 544 & 9487 & 225.948000 \\
4 & 10462 & 134 & 19635 & 990 & 14143 & 323.909800 \\
4 & 20023 & 234 & 28665 & 1312 & 24495 & 488.523200 \\
4 & 30332 & 332 & 39270 & 1790 & 34641 & 641.131600 \\
4 & 40597 & 416 & 49665 & 2050 & 44721 & 785.860400 \\
4 & 51826 & 516 & 58905 & 2476 & 54773 & 926.562400 \\
4 & 60413 & 604 & 69615 & 2826 & 64807 & 1064.824400 \\
4 & 71633 & 676 & 78540 & 3108 & 74833 & 1194.073200 \\
4 & 80441 & 786 & 87780 & 3374 & 84853 & 1324.849400 \\
4 & 91958 & 860 & 98175 & 3708 & 94869 & 1455.384200 \\
5 & 101467 & 948 & 195195 & 6716 & 141421 & 2117.938640 \\
5 & 204928 & 1688 & 285285 & 9808 & 244949 & 3252.344420 \\
5 & 300739 & 2396 & 390390 & 12048 & 346411 & 4319.025960 \\
5 & 401509 & 3044 & 495495 & 14828 & 447213 & 5340.328240 \\
5 & 500417 & 3742 & 570570 & 17786 & 547723 & 6334.456400 \\
5 & 603182 & 4352 & 690690 & 20546 & 648075 & 7298.350160 \\
5 & 700268 & 4948 & 765765 & 22942 & 748331 & 8241.694940 \\
5 & 804191 & 5550 & 855855 & 25114 & 848529 & 9177.052700 \\
5 & 909037 & 6154 & 990990 & 26788 & 948683 & 10089.567000 \\
6 & 1004449 & 6742 & 1996995 & 51734 & 1414213 & 14890.739498 \\
6 & 2012212 & 12360 & 2984520 & 71382 & 2449489 & 23157.936372 \\
6 & 3004042 & 17494 & 3993990 & 94150 & 3464101 & 31002.929602 \\
6 & 4015034 & 22544 & 4849845 & 118980 & 4472135 & 38562.728138 \\
6 & 5001482 & 27418 & 5870865 & 139510 & 5477225 & 45926.926894 \\
6 & 6002812 & 32242 & 6891885 & 152328 & 6480741 & 53114.608204 \\
6 & 7010638 & 36882 & 7912905 & 177818 & 7483315 & 60199.478904 \\
6 & 8007488 & 41544 & 8843835 & 195128 & 8485281 & 67166.391644 \\
6 & 9001429 & 46072 & 9699690 & 217942 & 9486833 & 74015.436638 \\
7 & 10030684 & 50364 & 19399380 & 400846 & 14142135 & 110283.272893 \\
7 & 20007184 & 93132 & 29099070 & 572870 & 24494897 & 173140.056904 \\
7 & 30032203 & 133266 & 38798760 & 738184 & 34641017 & 233156.312160 \\
7 & 40002659 & 172084 & 48498450 & 900422 & 44721359 & 291303.478127 \\
7 & 50008249 & 209830 & 58198140 & 1060096 & 54772255 & 348071.892597 \\
7 & 60010597 & 246670 & 67897830 & 1213536 & 64807407 & 403718.933389 \\
7 & 70017487 & 282866 & 77597520 & 1367996 & 74833147 & 458571.351887 \\
7 & 80015692 & 318898 & 87297210 & 1518344 & 84852813 & 512553.181301 \\
7 & 90020452 & 353874 & 99804705 & 1692366 & 94868329 & 565927.009048 \\
\bottomrule
\end{longtable}

\clearpage

\small
\sisetup{
  group-separator = {\,},
  group-minimum-digits = 4,
  round-mode=places
}
\begin{longtable}{
  S[table-format=1,round-precision=0] % Dec.
  S[table-format=7,round-precision=0] % n_0
  S[table-format=1.4,round-precision=4] % Cmeas_min
  S[table-format=7,round-precision=0] % n_1
  S[table-format=2.4,round-precision=4] % Cmeas_max
  S[table-format=7,round-precision=0] % <n_geom>
  S[table-format=1.5,round-precision=5] % Cmeas_\tavg
}
\caption[]{Normilized by \( \frac{log^2 n}{M} \) Per-Decade Statistics for Goldbach Pair Counts for \( |m| \in [ 1, \lfloor \frac{n}{2} \rfloor ] \)}\label{tab:normalized-status} \\
\toprule
\multicolumn{1}{c}{\textbf{Dec.}} &
\multicolumn{1}{c}{\textbf{\( n_0 \)}} &
\multicolumn{1}{c}{\textbf{\( \Cmeas_{\min}(n_0)\)}} & 
\multicolumn{1}{c}{\textbf{\( n_1 \)}} &
\multicolumn{1}{c}{\textbf{\(\Cmeas_{\max}(n_1)\)}} & 
\multicolumn{1}{c}{\(\Ngeom \)} &
\multicolumn{1}{c}{\(\Cmeas_\tavg\)} \\
\midrule
\endfirsthead
\multicolumn{7}{l}{\textit{(continued)}} \\
\toprule
\multicolumn{1}{c}{\textbf{Dec.}} &
\multicolumn{1}{c}{\textbf{\( n_0 \)}} &
\multicolumn{1}{c}{\textbf{\( \Cmeas_{\min}(n_0)\)}} & 
\multicolumn{1}{c}{\textbf{\( n_1 \)}} &
\multicolumn{1}{c}{\textbf{\(\Cmeas_{\max}(n_1)\)}} & 
\multicolumn{1}{c}{\(\Ngeom \)} &
\multicolumn{1}{c}{\( \Cmeas_\tavg \)} \\
\midrule
\endhead
0 & 4 & 1.921812 & 4 & 1.921812 & 4 & 1.921812 \\
0 & 5 & 2.590290 & 5 & 2.590290 & 5 & 2.590290 \\
0 & 6 & 2.140268 & 6 & 2.140268 & 6 & 2.140268 \\
0 & 7 & 0.000000 & 7 & 0.000000 & 7 & 0.000000 \\
0 & 8 & 2.162039 & 8 & 2.162039 & 8 & 2.162039 \\
0 & 9 & 4.827796 & 9 & 4.827796 & 9 & 4.827796 \\
1 & 11 & 0.000000 & 15 & 4.190592 & 15 & 2.225233 \\
1 & 28 & 1.586227 & 21 & 5.561470 & 25 & 2.767780 \\
1 & 37 & 1.448748 & 30 & 6.169677 & 35 & 3.110719 \\
1 & 43 & 0.000000 & 45 & 6.586672 & 45 & 2.746579 \\
1 & 59 & 1.146642 & 57 & 5.837951 & 55 & 3.444943 \\
1 & 64 & 1.081019 & 60 & 6.705463 & 65 & 3.262668 \\
1 & 79 & 0.979081 & 75 & 7.053239 & 75 & 3.922845 \\
1 & 89 & 1.831623 & 81 & 4.827796 & 85 & 3.287355 \\
1 & 97 & 0.871999 & 90 & 5.399543 & 95 & 3.436758 \\
2 & 199 & 1.132084 & 105 & 8.330518 & 141 & 3.583408 \\
2 & 223 & 1.053604 & 210 & 8.169017 & 245 & 3.602339 \\
2 & 379 & 1.492247 & 315 & 8.431106 & 347 & 3.754889 \\
2 & 433 & 1.364958 & 420 & 7.991882 & 447 & 3.819910 \\
2 & 569 & 1.983900 & 570 & 7.912132 & 547 & 3.788293 \\
2 & 661 & 1.788981 & 660 & 7.918936 & 649 & 3.849522 \\
2 & 706 & 1.706515 & 735 & 8.545496 & 749 & 3.944286 \\
2 & 802 & 1.784236 & 840 & 8.204146 & 849 & 3.919971 \\
2 & 967 & 1.761041 & 975 & 7.586652 & 949 & 3.790608 \\
3 & 1402 & 1.647631 & 1155 & 9.135596 & 1415 & 3.930334 \\
3 & 2029 & 1.715762 & 2730 & 9.539146 & 2449 & 3.938192 \\
3 & 3076 & 1.845344 & 3465 & 9.205098 & 3465 & 3.947789 \\
3 & 4801 & 1.856187 & 4620 & 9.432003 & 4473 & 3.975565 \\
3 & 5416 & 1.965120 & 5775 & 9.302545 & 5477 & 3.956638 \\
3 & 6353 & 2.124621 & 6930 & 9.570228 & 6481 & 4.021600 \\
3 & 7219 & 2.055910 & 7770 & 9.129686 & 7483 & 3.972489 \\
3 & 8777 & 2.179478 & 8925 & 9.643464 & 8485 & 3.976809 \\
3 & 9649 & 2.163663 & 9240 & 9.637497 & 9487 & 3.991146 \\
4 & 11272 & 2.131467 & 15015 & 10.422348 & 14143 & 4.004160 \\
4 & 20816 & 2.279925 & 21945 & 10.036306 & 24495 & 4.010742 \\
4 & 35792 & 2.297694 & 30030 & 10.293246 & 34641 & 4.011841 \\
4 & 40597 & 2.307754 & 45045 & 10.267641 & 44721 & 4.011235 \\
4 & 51826 & 2.346624 & 58905 & 10.142189 & 54773 & 4.015062 \\
4 & 67904 & 2.413565 & 60060 & 10.288579 & 64807 & 4.024570 \\
4 & 71633 & 2.358847 & 75075 & 10.186498 & 74833 & 4.012790 \\
4 & 89459 & 2.383150 & 87780 & 9.960052 & 84853 & 4.016448 \\
4 & 92357 & 2.434532 & 90090 & 10.384689 & 94869 & 4.025406 \\
5 & 116728 & 2.402456 & 150150 & 10.404410 & 141421 & 4.020835 \\
5 & 204928 & 2.464243 & 255255 & 10.988794 & 244949 & 4.022875 \\
5 & 366794 & 2.499164 & 345345 & 10.823075 & 346411 & 4.023491 \\
5 & 463549 & 2.513127 & 435435 & 10.808231 & 447213 & 4.022716 \\
5 & 548461 & 2.531963 & 510510 & 11.026899 & 547723 & 4.024812 \\
5 & 686398 & 2.527108 & 690690 & 10.755350 & 648075 & 4.023686 \\
5 & 770558 & 2.532269 & 765765 & 10.999099 & 748331 & 4.022217 \\
5 & 804191 & 2.552045 & 855855 & 10.950614 & 848529 & 4.025351 \\
5 & 915961 & 2.547074 & 930930 & 10.674738 & 948683 & 4.024420 \\
6 & 1201553 & 2.553525 & 1276275 & 11.043507 & 1414213 & 4.023671 \\
6 & 2053553 & 2.579809 & 2042040 & 11.036389 & 2449489 & 4.023685 \\
6 & 3004042 & 2.591117 & 3573570 & 11.047450 & 3464101 & 4.023936 \\
6 & 4792159 & 2.588525 & 4849845 & 11.627998 & 4472135 & 4.023153 \\
6 & 5167067 & 2.597599 & 5870865 & 11.544510 & 5477225 & 4.023416 \\
6 & 6175451 & 2.603294 & 6561555 & 11.429827 & 6480741 & 4.022315 \\
6 & 7376626 & 2.610535 & 7402395 & 11.421215 & 7483315 & 4.023270 \\
6 & 8143934 & 2.607556 & 8273265 & 11.322354 & 8485281 & 4.023595 \\
6 & 9121549 & 2.613892 & 9699690 & 11.630432 & 9486833 & 4.022609 \\
7 & 10030684 & 2.609830 & 14549535 & 11.637951 & 14142135 & 4.022242 \\
7 & 24496594 & 2.621733 & 29099070 & 11.629674 & 24494897 & 4.021850 \\
7 & 30099763 & 2.626030 & 38798760 & 11.618670 & 34641017 & 4.021567 \\
7 & 41344276 & 2.629465 & 48498450 & 11.629194 & 44721359 & 4.021290 \\
7 & 53699671 & 2.633020 & 58198140 & 11.645824 & 54772255 & 4.021122 \\
7 & 66759878 & 2.632266 & 67897830 & 11.624854 & 64807407 & 4.020612 \\
7 & 78822322 & 2.634340 & 77597520 & 11.636855 & 74833147 & 4.021107 \\
7 & 82476448 & 2.635786 & 82447365 & 11.630461 & 84852813 & 4.020558 \\
7 & 96281998 & 2.635636 & 96996900 & 11.629483 & 94868329 & 4.020444 \\
\bottomrule
\end{longtable}
 
\begin{remark}
Primorials consistently correspond to maxima. Many unnormalized binned maxima have occurred at values equal to \( 19\# \) or its multiples, and many of the normalized maxima align with these values as well. In contrast, the minima are more likely to occur at values that are either prime or semiprime.
\end{remark}
 
\clearpage

\small
\sisetup{
  group-separator = {\,},
  group-minimum-digits = 4,
  round-mode=places
}
\begin{longtable}{
  S[table-format=1,round-precision=0] % Dec.
  S[table-format=7,round-precision=0] % n_0
  S[table-format=1.4,round-precision=4] % Cpred_min
  S[table-format=7,round-precision=0] % n_1
  S[table-format=2.4,round-precision=4] % Cpred_max
  S[table-format=7,round-precision=0] % <n_geom>
  S[table-format=1.5,round-precision=5] % Cpred_\tavg
}
\caption[]{Normalized by \( \frac{log^2 n}{M} \) Per-Decade HL-A Predictions for Goldbach Pair Counts for \( |m| \in [ 1, \lfloor \frac{n}{2} \rfloor ] \)}\label{tab:normalized-status} \\
\toprule
\multicolumn{1}{c}{\textbf{Dec.}} &
\multicolumn{1}{c}{\textbf{\(\Npred_0\)}} &
\multicolumn{1}{c}{\textbf{\(\Cpred_{\min}(n_0)\)}} & 
\multicolumn{1}{c}{\textbf{\(\Npred_1\)}} &
\multicolumn{1}{c}{\textbf{\(\Cpred_{\max}(n_1)\)}} & 
\multicolumn{1}{c}{\(\Ngeom \)} &
\multicolumn{1}{c}{\(\Cpred_\tavg(\Ngeom) \)} \\
\midrule
\endfirsthead
\multicolumn{7}{l}{\textit{(continued)}} \\
\toprule
\multicolumn{1}{c}{\textbf{Dec.}} &
\multicolumn{1}{c}{\textbf{\(\Npred_0\)}} &
\multicolumn{1}{c}{\textbf{\(\Cpred_{\min}(n_0)\)}} & 
\multicolumn{1}{c}{\textbf{\(\Npred_1\)}} &
\multicolumn{1}{c}{\textbf{\(\Cpred_{\max}(n_1)\)}} & 
\multicolumn{1}{c}{\(\Ngeom \)} &
\multicolumn{1}{c}{\(\Cpred_\tavg(\Ngeom) \)} \\
\midrule
\endhead
0 & 4 & 2.87013699 & 4 & 2.87013699 & 4 & 2.64064726 \\
0 & 5 & 4.26609329 & 5 & 4.26609329 & 5 & 4.20600483 \\
0 & 6 & 6.21887455 & 6 & 6.21887455 & 6 & 5.64216820 \\
0 & 7 & 3.39303648 & 7 & 3.39303648 & 7 & 3.38530092 \\
0 & 8 & 2.81463781 & 8 & 2.81463781 & 8 & 2.90085628 \\
0 & 9 & 5.82036707 & 9 & 5.82036707 & 9 & 5.80171256 \\
1 & 16 & 2.76514172 & 15 & 7.65091857 & 15 & 4.01876698 \\
1 & 29 & 2.85574939 & 21 & 7.06549955 & 25 & 4.02053563 \\
1 & 32 & 2.73462427 & 30 & 7.34072410 & 35 & 4.33099119 \\
1 & 47 & 2.78303527 & 45 & 7.28666194 & 45 & 4.05231999 \\
1 & 59 & 2.76459736 & 51 & 5.87344572 & 55 & 3.92309015 \\
1 & 64 & 2.71538521 & 60 & 7.27786003 & 65 & 4.34450112 \\
1 & 79 & 2.74592873 & 75 & 7.25547171 & 75 & 4.04341550 \\
1 & 89 & 2.74316855 & 84 & 6.52215624 & 85 & 3.91074574 \\
1 & 97 & 2.73859883 & 90 & 7.25157226 & 95 & 4.32055410 \\
2 & 128 & 2.70246576 & 105 & 8.80126716 & 141 & 4.08097788 \\
2 & 256 & 2.69327548 & 210 & 8.73000537 & 245 & 4.06423727 \\
2 & 397 & 2.69605382 & 315 & 8.73109115 & 347 & 4.09735539 \\
2 & 499 & 2.69203486 & 420 & 8.67805011 & 447 & 4.05744364 \\
2 & 512 & 2.68642910 & 525 & 8.58966504 & 547 & 4.05134592 \\
2 & 691 & 2.68786367 & 630 & 8.62469707 & 649 & 4.08269353 \\
2 & 797 & 2.68650813 & 735 & 8.61583623 & 749 & 4.06558233 \\
2 & 887 & 2.68515443 & 840 & 8.60202236 & 849 & 4.04121796 \\
2 & 997 & 2.68419046 & 945 & 8.59894371 & 949 & 4.07948187 \\
3 & 1024 & 2.68114071 & 1155 & 9.49783347 & 1415 & 4.05732740 \\
3 & 2048 & 2.67693692 & 2310 & 9.48618550 & 2449 & 4.05096314 \\
3 & 3989 & 2.67434162 & 3465 & 9.55395942 & 3465 & 4.05381398 \\
3 & 4096 & 2.67351721 & 4620 & 9.47721540 & 4473 & 4.04863303 \\
3 & 5987 & 2.67234740 & 5775 & 9.50999421 & 5477 & 4.04552966 \\
3 & 6997 & 2.67168396 & 6930 & 9.50035916 & 6481 & 4.04892526 \\
3 & 7993 & 2.67112453 & 7140 & 9.14799904 & 7483 & 4.04549973 \\
3 & 8192 & 2.67068212 & 8085 & 9.49920205 & 8485 & 4.04350777 \\
3 & 9973 & 2.67024401 & 9240 & 9.51384520 & 9487 & 4.04719853 \\
4 & 16384 & 2.66829428 & 15015 & 10.37436085 & 14143 & 4.04245695 \\
4 & 29989 & 2.66659241 & 21945 & 10.15052858 & 24495 & 4.03978826 \\
4 & 32768 & 2.66625607 & 30030 & 10.36446377 & 34641 & 4.03877239 \\
4 & 49999 & 2.66520706 & 45045 & 10.36398816 & 44721 & 4.03747564 \\
4 & 59999 & 2.66475120 & 58905 & 10.11012965 & 54773 & 4.03663285 \\
4 & 65536 & 2.66449625 & 60060 & 10.35598050 & 64807 & 4.03629909 \\
4 & 79999 & 2.66406689 & 75075 & 10.34769829 & 74833 & 4.03559927 \\
4 & 89989 & 2.66379965 & 87780 & 10.03346629 & 84853 & 4.03511638 \\
4 & 99991 & 2.66356349 & 90090 & 10.35584357 & 94869 & 4.03499768 \\
5 & 131072 & 2.66296164 & 105105 & 10.38895826 & 141421 & 4.03355119 \\
5 & 262144 & 2.66161176 & 255255 & 11.01757874 & 244949 & 4.03193098 \\
5 & 399989 & 2.66087269 & 345345 & 10.84610338 & 346411 & 4.03102055 \\
5 & 499979 & 2.66049818 & 435435 & 10.73208779 & 447213 & 4.03033245 \\
5 & 524288 & 2.66041527 & 510510 & 11.01229674 & 547723 & 4.02981837 \\
5 & 699967 & 2.65995946 & 690690 & 10.81114647 & 648075 & 4.02945078 \\
5 & 799999 & 2.65975367 & 765765 & 11.01274606 & 748331 & 4.02908983 \\
5 & 899981 & 2.65957595 & 855855 & 10.93227057 & 848529 & 4.02878707 \\
5 & 999983 & 2.65941957 & 930930 & 10.68452845 & 948683 & 4.02857525 \\
6 & 1048576 & 2.65934749 & 1021020 & 11.00758663 & 1414213 & 4.02768014 \\
6 & 2097152 & 2.65838879 & 2042040 & 11.00335893 & 2449489 & 4.02656689 \\
6 & 3999971 & 2.65758050 & 3063060 & 11.05712271 & 3464101 & 4.02591467 \\
6 & 4194304 & 2.65752329 & 4849845 & 11.62138674 & 4472135 & 4.02544734 \\
6 & 5999993 & 2.65710885 & 5870865 & 11.52057514 & 5477225 & 4.02509003 \\
6 & 6999997 & 2.65693635 & 6561555 & 11.44271137 & 6480741 & 4.02480524 \\
6 & 7999993 & 2.65678982 & 7402395 & 11.41274135 & 7483315 & 4.02456221 \\
6 & 8388608 & 2.65673807 & 8273265 & 11.31740297 & 8485281 & 4.02435326 \\
6 & 9999991 & 2.65655073 & 9699690 & 11.64269129 & 9486833 & 4.02417674 \\
7 & 16777216 & 2.65602249 & 14549535 & 11.66159797 & 14142135 & 4.02354807 \\
7 & 29999999 & 2.65546976 & 24249225 & 11.67464602 & 24494897 & 4.02273770 \\
7 & 33554432 & 2.65536769 & 33948915 & 11.63081684 & 34641017 & 4.02225484 \\
7 & 49999991 & 2.65501557 & 43648605 & 11.65404706 & 44721359 & 4.02191131 \\
7 & 59999999 & 2.65486011 & 53348295 & 11.64953326 & 54772255 & 4.02164612 \\
7 & 67108864 & 2.65476625 & 63047985 & 11.63930860 & 64807407 & 4.02143150 \\
7 & 79999987 & 2.65462154 & 72747675 & 11.64407803 & 74833147 & 4.02125065 \\
7 & 89999999 & 2.65452615 & 82447365 & 11.64227746 & 84852813 & 4.02109518 \\
7 & 99999989 & 2.65444192 & 92147055 & 11.64083198 & 94868329 & 4.02095965 \\
\bottomrule
\end{longtable}
 
\clearpage
\small
\sisetup{
  group-separator = {\,},
  group-minimum-digits = 4,
  round-mode=places
}
\begin{longtable}{
  S[table-format=1,round-precision=0] % Dec.
  S[table-format=7,round-precision=0] % n_0
  S[table-format=1.3,round-precision=3] % Cmeas_min
  S[table-format=1.3,round-precision=3] % C_-
  S[table-format=1.3,round-precision=3] % Delta
  S[table-format=1.3,round-precision=3] % C_-
  S[table-format=1.3,round-precision=3] % Delta
}
\caption[]{\( \Lambda \) Calculations for Euler Product Series Products}\label{tab:lambda-values} \\
\toprule
\multicolumn{1}{c}{\textbf{Dec.}} &
\multicolumn{1}{c}{\(\Nmeas_0\)} &
\multicolumn{1}{c}{\(\Cmeas_{\min}\)} &
\multicolumn{1}{c}{\(\CminusProduct\)} &
\multicolumn{1}{c}{\(\Cmeas_{\min} - \CminusProduct\)} &
\multicolumn{1}{c}{\(\CminusAsymp\)} &
\multicolumn{1}{c}{\(\Cmeas_{\min} - \CminusAsymp\)} \\
\midrule
\endfirsthead
\multicolumn{7}{l}{\textit{(continued)}} \\
\toprule
\multicolumn{1}{c}{\textbf{Dec.}} &
\multicolumn{1}{c}{\(\Nmeas_0\)} &
\multicolumn{1}{c}{\(\Cmeas_{\min}(\Nmeas_0)\)} &
\multicolumn{1}{c}{\(\CminusProduct(\Nmeas_0)\)} &
\multicolumn{1}{c}{\(\Cmeas_{\min}(\Nmeas_0) - \CminusProduct(\Nmeas_0)\)} &
\multicolumn{1}{c}{\(\CminusAsymp(\Nmeas_0)\)} &
\multicolumn{1}{c}{\(\Cmeas_{\min}(\Nmeas_0) - \CminusAsymp(\Nmeas_0)\)} \\
\midrule
\endhead
0 & 4 & 1.9218 & 0.9609 & 0.9609 & 1.7007 & 0.2211 \\
0 & 5 & 2.5903 & 1.2951 & 1.2951 & 1.7558 & 0.8345 \\
0 & 6 & 2.1403 & 1.6052 & 0.5351 & 1.7925 & 0.3478 \\
0 & 7 & 0.0000 & 1.8933 & -1.8933 & 1.8191 & -1.8191 \\
0 & 8 & 2.1620 & 2.1620 & 0.0000 & 1.8395 & 0.3226 \\
0 & 9 & 4.8278 & 1.2069 & 3.6208 & 1.8557 & 2.9721 \\
1 & 11 & 0.0000 & 1.4375 & -1.4375 & 1.8802 & -1.8802 \\
1 & 28 & 1.5862 & 1.5614 & 0.0248 & 1.9597 & -0.3735 \\
1 & 37 & 1.4487 & 1.5280 & -0.0792 & 1.9762 & -0.5275 \\
1 & 43 & 0.0000 & 1.6578 & -1.6578 & 1.9842 & -1.9842 \\
1 & 59 & 1.1466 & 1.6237 & -0.4770 & 1.9993 & -0.8527 \\
1 & 64 & 1.0810 & 1.6891 & -0.6081 & 2.0029 & -0.9219 \\
1 & 79 & 0.9791 & 1.8645 & -0.8854 & 2.0115 & -1.0324 \\
1 & 89 & 1.8316 & 1.7708 & 0.0608 & 2.0160 & -0.1844 \\
1 & 97 & 0.8720 & 1.8394 & -0.9674 & 2.0192 & -1.1472 \\
2 & 199 & 1.1321 & 1.7460 & -0.6139 & 2.0418 & -0.9097 \\
2 & 223 & 1.0536 & 1.8219 & -0.7683 & 2.0448 & -0.9912 \\
2 & 379 & 1.4922 & 1.7535 & -0.2613 & 2.0576 & -0.5654 \\
2 & 433 & 1.3650 & 1.8331 & -0.4681 & 2.0605 & -0.6956 \\
2 & 569 & 1.9839 & 1.8425 & 0.1414 & 2.0661 & -0.0822 \\
2 & 661 & 1.7890 & 1.8663 & -0.0773 & 2.0690 & -0.2800 \\
2 & 706 & 1.7065 & 1.9043 & -0.1978 & 2.0702 & -0.3637 \\
2 & 802 & 1.7842 & 1.9790 & -0.1948 & 2.0725 & -0.2883 \\
2 & 967 & 1.7610 & 1.8953 & -0.1342 & 2.0757 & -0.3147 \\
3 & 1402 & 1.6476 & 1.9484 & -0.3008 & 2.0817 & -0.4340 \\
3 & 2029 & 1.7158 & 1.9655 & -0.2497 & 2.0870 & -0.3713 \\
3 & 3076 & 1.8453 & 1.9962 & -0.1509 & 2.0925 & -0.2472 \\
3 & 4801 & 1.8562 & 2.0060 & -0.1498 & 2.0978 & -0.2416 \\
3 & 5416 & 1.9651 & 1.9829 & -0.0178 & 2.0991 & -0.1340 \\
3 & 6353 & 2.1246 & 2.0097 & 0.1149 & 2.1009 & 0.0237 \\
3 & 7219 & 2.0559 & 2.0033 & 0.0526 & 2.1022 & -0.0463 \\
3 & 8777 & 2.1795 & 2.0120 & 0.1675 & 2.1042 & 0.0753 \\
3 & 9649 & 2.1637 & 2.0328 & 0.1309 & 2.1051 & 0.0585 \\
4 & 11272 & 2.1315 & 2.0445 & 0.0870 & 2.1066 & 0.0249 \\
4 & 20816 & 2.2799 & 2.0628 & 0.2171 & 2.1120 & 0.1679 \\
4 & 35792 & 2.2977 & 2.0772 & 0.2204 & 2.1163 & 0.1814 \\
4 & 40597 & 2.3078 & 2.0576 & 0.2501 & 2.1173 & 0.1905 \\
4 & 51826 & 2.3466 & 2.0761 & 0.2705 & 2.1190 & 0.2276 \\
4 & 67904 & 2.4136 & 2.0776 & 0.3360 & 2.1209 & 0.2927 \\
4 & 71633 & 2.3588 & 2.0896 & 0.2693 & 2.1212 & 0.2376 \\
4 & 89459 & 2.3832 & 2.0902 & 0.2929 & 2.1227 & 0.2605 \\
4 & 92357 & 2.4345 & 2.0962 & 0.3384 & 2.1229 & 0.3117 \\
5 & 116728 & 2.4025 & 2.1038 & 0.2987 & 2.1243 & 0.2781 \\
5 & 204928 & 2.4642 & 2.1064 & 0.3579 & 2.1276 & 0.3366 \\
5 & 366794 & 2.4992 & 2.1103 & 0.3888 & 2.1307 & 0.3684 \\
5 & 463549 & 2.5131 & 2.1063 & 0.4068 & 2.1319 & 0.3812 \\
5 & 548461 & 2.5320 & 2.1123 & 0.4196 & 2.1327 & 0.3992 \\
5 & 686398 & 2.5271 & 2.1154 & 0.4117 & 2.1338 & 0.3933 \\
5 & 770558 & 2.5323 & 2.1140 & 0.4183 & 2.1343 & 0.3980 \\
5 & 804191 & 2.5520 & 2.1124 & 0.4397 & 2.1345 & 0.4175 \\
5 & 915961 & 2.5471 & 2.1177 & 0.4294 & 2.1351 & 0.4120 \\
6 & 1201553 & 2.5535 & 2.1168 & 0.4368 & 2.1363 & 0.4172 \\
6 & 2053553 & 2.5798 & 2.1248 & 0.4550 & 2.1385 & 0.4413 \\
6 & 3004042 & 2.5911 & 2.1261 & 0.4650 & 2.1400 & 0.4511 \\
6 & 4792159 & 2.5885 & 2.1336 & 0.4549 & 2.1417 & 0.4468 \\
6 & 5167067 & 2.5976 & 2.1314 & 0.4662 & 2.1420 & 0.4556 \\
6 & 6175451 & 2.6033 & 2.1308 & 0.4725 & 2.1426 & 0.4607 \\
6 & 7376626 & 2.6105 & 2.1341 & 0.4765 & 2.1432 & 0.4673 \\
6 & 8143934 & 2.6076 & 2.1343 & 0.4732 & 2.1435 & 0.4640 \\
6 & 9121549 & 2.6139 & 2.1342 & 0.4797 & 2.1439 & 0.4700 \\
7 & 10030684 & 2.6098 & 2.1372 & 0.4726 & 2.1442 & 0.4656 \\
7 & 24496594 & 2.6217 & 2.1420 & 0.4798 & 2.1470 & 0.4747 \\
7 & 30099763 & 2.6260 & 2.1414 & 0.4847 & 2.1476 & 0.4784 \\
7 & 41344276 & 2.6295 & 2.1423 & 0.4872 & 2.1485 & 0.4810 \\
7 & 53699671 & 2.6330 & 2.1444 & 0.4886 & 2.1492 & 0.4838 \\
7 & 66759878 & 2.6323 & 2.1455 & 0.4868 & 2.1498 & 0.4825 \\
7 & 78822322 & 2.6343 & 2.1453 & 0.4890 & 2.1502 & 0.4841 \\
7 & 82476448 & 2.6358 & 2.1459 & 0.4899 & 2.1503 & 0.4855 \\
7 & 96281998 & 2.6356 & 2.1463 & 0.4893 & 2.1507 & 0.4849 \\
\bottomrule
\end{longtable}

\section{Reproducibility}

All source code, certification tools, and datasets used in this work are permanently archived on Zenodo.~\cite{Riemers2025SieveGoldbach}  
The repository includes build scripts, certification outputs, and checksums to ensure bitwise reproducibility of all results.

\clearpage

\bibliographystyle{plain}   % or 'alpha', 'abbrv', etc.
\bibliography{sieve_goldbach}
\end{document}
\typeout{get arXiv to do 4 passes: Label(s) may have changed. Rerun}

