% Copyright (C) 2025 Bill C. Riemers
% SPDX-License-Identifier: CC-BY-4.0
%
% Licensed under the Creative Commons Attribution 4.0 International License.
% You may obtain a copy of the License at:
%     https://creativecommons.org/licenses/by/4.0/
%
% You are free to:
%   - Share --- copy and redistribute the material in any medium or format
%   - Adapt --- remix, transform, and build upon the material for any purpose
% Under the following terms:
%   - Attribution --- You must give appropriate credit, provide a link to the license,
%     and indicate if changes were made.

\documentclass[11pt]{article}
\usepackage{amsmath,amssymb,amsthm,fullpage,mathtools,microtype}
\setlength{\parskip}{0.8em}
\setlength{\parindent}{0pt}
\usepackage[utf8]{inputenc}
\usepackage{geometry}
\usepackage{fullpage}
\usepackage{hyperref}
\usepackage{graphicx}
\usepackage{float}
\usepackage{needspace}
\usepackage{mathtools}
\usepackage{url}
\usepackage{longtable,booktabs}
\usepackage[T1]{fontenc}
\usepackage{lmodern}
\renewcommand{\ttdefault}{lmtt}
\usepackage{microtype}
\usepackage{mathtools}
\usepackage{siunitx}
\usepackage{pgfplots}
\pgfplotsset{compat=1.18} % or at least 1.11+
\pgfplotsset{set layers} % enables axis background/foreground layers
\usepgfplotslibrary{fillbetween} % <-- this is the key line
\usetikzlibrary{fillbetween}
\usepackage{pgfplotstable}
\usepackage{xparse}
\usepackage[section]{placeins} % floats can't cross section boundaries
\usepackage{etoolbox}
\usepackage{xcolor}
\usepackage[skip=2pt]{caption}

% Define stable color names:
\colorlet{bandA}{blue}
\colorlet{bandB}{red}
\colorlet{bandC}{teal}
\colorlet{bandD}{orange}
\colorlet{bandE}{violet}
\colorlet{bandF}{green!60!black}
\colorlet{bandG}{magenta}
\colorlet{bandH}{cyan!60!black}
\colorlet{bandI}{brown}
\colorlet{bandJ}{gray}
\colorlet{bandK}{lime!60!black}

\pgfplotstableset{empty cells with={nan}} % treat empty CSV cells as NaN

% When a point becomes NaN, don't try to draw through it
\pgfplotsset{unbounded coords=jump}

\title{\textbf{Empirical Validation and Certification of Prime Curvature Geometry}}
\author{Bill C. Riemers}
\date{\today}

\newcommand{\BandColorName}[1]{bandcolor#1}
% Make sure you have:
% \usepgfplotslibrary{fillbetween}
% \pgfplotsset{unbounded coords=jump}

\newcommand{\PlotBand}[3]{% #1 = alpha literal, #2 = color, #3 = alpha in caption form

  % low curve
  \addplot[
    name path=low#2,
    draw=#2,
    thin,
    forget plot,
    x filter/.code={%
      \pgfmathparse{abs((\thisrow{alpha}+0)-(#1)) < 1e-12 ? \pgfmathresult : nan}%
    },
  ] table[
    x=n_min_lambda, 
    y expr=abs(\thisrow{min_lambda}),
  ]{\banddata};

  % high curve
  \addplot[
    name path=high#2,
    draw=#2,
    thin,
    forget plot,
    x filter/.code={%
      \pgfmathparse{abs((\thisrow{alpha}+0)-(#1)) < 1e-12 ? \pgfmathresult : nan}%
    },
  ] table[
    x=n_max_lambda,
    y expr=abs(\thisrow{max_lambda}),
  ]{\banddata};

  % fill between
  \addplot[
    draw=none,
    fill=#2,
    fill opacity=0.18,
  ] fill between[of=low#2 and high#2];

  \addlegendentry{\(\alpha=#3\)}
}

\DeclarePairedDelimiter{\card}{\lvert}{\rvert}

\newtheoremstyle{inline}% for remarks/notes
  {}{}{\normalfont}{}{\itshape}{.}{ }{}
  
\newtheoremstyle{break}  % Name
  {1ex}                  % Space above
  {1ex}                  % Space below
  {\normalfont}          % Body font
  {}                     % Indent
  {\bfseries}            % Theorem head font
  {.}                    % Punctuation after theorem head
  {\newline}             % Space after theorem head (THIS FORCES LINE BREAK)
  {}                     % Theorem head spec

\theoremstyle{inline}
\newtheorem*{remark}{Remark}

\theoremstyle{inline}
\newtheorem*{convention}{Convention}

\theoremstyle{break}
\newtheorem{lemma}{Lemma}

\makeatletter
\renewenvironment{proof}[1][\proofname]{%
  \par\pushQED{\qed}%
  \normalfont \topsep6\p@\@plus6\p@\relax
  \trivlist
  \item[\hskip\labelsep
        \itshape
    #1\@addpunct{.}]\mbox{}\\  % This line forces the break
}{%
  \popQED\endtrivlist\@endpefalse
}
\makeatother

\theoremstyle{break}
\newtheorem*{conclusion}{Conclusion}

\theoremstyle{break}
\newtheorem{theorem}{Theorem}

\theoremstyle{break}
\newtheorem{proposition}{Proposition}

\theoremstyle{break}
\newtheorem{conjecture}{Conjecture}

\theoremstyle{break}
\newtheorem{corollary}{Corollary}

\theoremstyle{break}
\newtheorem{definition}{Definition}

\theoremstyle{break}
\newtheorem{hypothesis}{Hypothesis}

\theoremstyle{inline}
\newtheorem*{note}{Note}

% --- Base (unchanged) ---
\newcommand{\Cmeas}{C}
\newcommand{\Cpred}{\mathring{C}}

\newcommand{\Gmeas}{G}
\newcommand{\Gpred}{\mathring{G}}
\newcommand{\Gproxy}{\widehat{G}}

% tiny tags
\newcommand{\talign}{{\scriptscriptstyle\mathrm{align}}}
\newcommand{\thead}{{\scriptscriptstyle\mathrm{head}}}
\newcommand{\ttail}{{\scriptscriptstyle\mathrm{tail}}}
\newcommand{\tavg}{{\scriptscriptstyle\mathrm{avg}}}
\newcommand{\tlow}{{\scriptscriptstyle\mathrm{low}}}
\newcommand{\thigh}{{\scriptscriptstyle\mathrm{high}}}
\newcommand{\textrema}{{\scriptscriptstyle\mathrm{extrema}}}
\newcommand{\tbound}{{\scriptscriptstyle\mathrm{bound}}}
\newcommand{\tdensity}{{\scriptscriptstyle\mathrm{density}}}
\newcommand{\tenv}{{\scriptscriptstyle\mathrm{env}}}
\newcommand{\tpairs}{{\scriptscriptstyle\mathrm{pairs}}}
\newcommand{\ttrivial}{{\scriptscriptstyle\mathrm{trivial}}}
\newcommand{\tsem}{{\scriptscriptstyle\mathrm{sem}}}
\newcommand{\tsieve}{{\scriptscriptstyle\mathrm{sieve}}}
\newcommand{\twin}{{\scriptscriptstyle\mathrm{win}}}
\newcommand{\tref}{{\scriptscriptstyle\mathrm{ref}}}
\newcommand{\tana}{{\scriptscriptstyle\mathrm{analytical}}}
\newcommand{\tgb}{{\scriptscriptstyle\mathrm{GB}}}
\newcommand{\thl}{{\scriptscriptstyle\mathrm{HL}}}
\newcommand{\teff}{{\scriptscriptstyle\mathrm{eff}}}
\DeclareMathOperator{\SquareCap}{\mathsf{Sq}}
\DeclareMathOperator{\EffLocModCap}{\mathsf{\EffLocMod}}

\newcommand{\etaHL}{\eta^{\mathrm{HL}}}
\newcommand{\Ecap}{\mathrm{Ecap}}
\newcommand{\HLCorr}{\mathcal{H}}
\newcommand{\pmin}{\mathrm{p_{\min}}}
\newcommand{\ecap}[1]{\left\langle #1 \right\rangle_{\mathrm{ec}}}
\newcommand{\GHL}{\Gpred^{\thl}}
\newcommand{\GHLproxy}{\Gproxy^{\thl}}
\newcommand{\OmegaPrime}{\Omega_{\mathrm{prime}}}
\newcommand{\OmegaPrimeNorm}{\widehat{\Omega}_{\mathrm{prime}}}
\newcommand{\Ipar}{I^{\mathrm{par}}}
\newcommand{\EffLocMod}{\mathcal{Q}}
\newcommand{\Peff}{\mathbb{P}_{\teff}}
\newcommand{\Podd}{{\mathbb{P}\setminus\{2\}}}
\newcommand{\RHLbound}{\widehat{R}^{\thl}}
\newcommand{\RHLboundEnv}{\RHLbound_{\tenv}}
\newcommand{\Ssem}{\mathfrak{S}}
\newcommand{\SsemHead}{\Ssem_{\thead}}
\newcommand{\SsemTail}{\Ssem_{\ttail}}
\newcommand{\SGB}{\mathfrak{S_{\tgb}}}
\newcommand{\HHL}{H^{\thl}}
\newcommand{\THL}{T^{\thl}}
\newcommand{\RHL}{R^{\thl}}
\newcommand{\QpleL}{\EffLocMod_p(n) \le L}
\newcommand{\QpProduct}{\prod_{\substack{p\in\Peff(n) \\ \QpleL}}}
\newcommand{\pZeroProduct}{\prod_{\substack{p\in\Peff(n) \\ p \le P_0}}}

\begin{document}

\maketitle

\begin{abstract}
This paper presents an empirical certification of the \textbf{Prime Curvature Geometry
Conjecture for Goldbach (PCGC--Goldbach)}, as formulated in~\cite{Riemers2026-PCGH}.
All tested instances with \(4 \le 2n \le 23\# \approx 2.5852\times 10^{8}\)
are certified to satisfy the geometric curvature bounds predicted by
PCGC--Goldbach.
The certification spans a discrete family of fixed-scale Goldbach windows and
covers more than 16~billion non-trivial test scenarios, with no violations
observed across the entire tested range.

Beyond aggregate verification, the paper provides detailed extremal and
pointwise analysis of the certified bounds together with a fully reproducible
computational pipeline.
All certified claims are grounded in explicit enumeration rather than
probabilistic heuristics, and the associated data products and verification
artifacts are archived to permit independent audit and falsification.
\end{abstract}

\clearpage
\tableofcontents


\clearpage
\preto\section{\FloatBarrier}
\section{Introduction}

\preto\subsection{\FloatBarrier}
\subsection{Background}

The Prime Curvature Geometry Conjecture for Goldbach (PCGC--Goldbach) was introduced
in~\cite{Riemers2026-PCGH} as a geometric framework for bounding deviations in
Goldbach pair counts.
The present paper reports an empirical certification of that framework over a
large, explicitly enumerated domain.

Crucially, the original intent of the computational programme was \emph{not} to
verify pointwise validity of PCGC--Goldbach.
On the contrary, the working assumption motivating the software design was that
prime distributions exhibit pseudo-random behaviour at sufficiently fine scales,
and that any non-aggregated formulation of the conjecture would almost certainly
admit local violations.

Under this assumption, one expects bounds derived from a geometric approximation
to hold only after aggregation over intervals large enough to be statistically
representative, and to fail when tested pointwise in \(n\).
Accordingly, the computational framework was engineered to probe a wide range of
window sizes and aggregation regimes, with the expectation that sufficiently fine
partitions would expose outliers.

Only after observing \emph{zero} violations across all tested cross-intervals
\(\Delta_\alpha(n)\), spanning 81 distinct curvature parameters \(\alpha\), did it
become clear that this expectation of statistical failures was incorrect.
Rather than exhibiting sporadic pointwise escapes, the measured pair counts remain
uniformly confined within the predicted curvature envelopes.

To verify that the absence of violations was not an artifact of the computational
pipeline, controlled perturbation tests were performed designed to force failures.
Specifically, key structural components of the PCGC--Goldbach framework were
deliberately replaced with known-inadequate surrogates.
In particular, the effective local modulus \(\EffLocMod_p(n)\) was replaced by the
corresponding primorial cutoff \(p\#\), thereby discarding the adaptive
localisation that encodes small-prime congruence structure.

Under these substitutions, violations of the predicted bounds appeared
immediately and systematically, both in aggregated extrema and at the pointwise
level.
The resulting failure patterns were consistent with theoretical expectations and
confirm that the software reliably detects genuine departures from admissible
curvature.

These perturbation tests demonstrate that the observed absence of violations
under the full PCGC--Goldbach formulation is not a consequence of numerical
smoothing, windowing artefacts, or implicit aggregation, but reflects a
structural property of the conjectured geometry itself.
Thus, the empirical results do not arise from tuning the experiment to confirm a
conjecture, but instead from an outcome that directly contradicts the original
working hypothesis built into the experimental design.

\preto\subsection{\FloatBarrier}
\subsection{Computational Framework Overview}

The software used in this study is an extension of the framework introduced
in~\cite{Riemers2025-SieveGoldbach}.
As before, binning is performed using primorial-based windows, preserving
compatibility with Hardy--Littlewood--style local corrections.

Two operational modes are supported:
\begin{enumerate}
\item \emph{Aggregated mode}, in which Hardy--Littlewood windowed predictions and
measured pair counts are generated in separate runs.
In this configuration, pointwise verification is not possible, but global extrema
can be rigorously compared.
\item \emph{Combined mode}, in which predicted bounds and measured values are
evaluated jointly, enabling pointwise certification.
\end{enumerate}

In aggregated mode, prediction runs complete in approximately half a day on an
Apple M2 processor, while full pair-count runs require approximately six days.
Combined mode additionally requires evaluating high-precision logarithmic
products for each admissible odd-pair candidate and consequently incurs a
computational cost approximately an order of magnitude greater than standard
Goldbach pair counting.

Across all tested primorial bins, no violations of the predicted bounds were
observed.
Moreover, the extremal values of the normalised log-ratio
\begin{equation}
\log(\Gmeas/\Gpred)
\end{equation}
exhibit local fluctuations but show an overall tightening trend toward zero
across the tested range.

\preto\subsection{\FloatBarrier}
\subsection{Contributions}

The principal contributions of this paper are as follows:

\begin{itemize}
\item Provide an empirical certification of the Prime Curvature Geometry
Conjecture for Goldbach over the range \(4 \le 2n \le 23\#\), covering a discrete
family of 81 fixed-scale window parameters and more than 16~billion non-trivial
test scenarios.

\item Demonstrate that the predicted geometric curvature bounds hold not only
in aggregated regimes but also pointwise, with no observed violations across the
entire certified domain.

\item Introduce a reproducible certification methodology that treats geometric
bounds as falsifiable numerical claims, supported by explicit enumeration,
archived data products, and independently auditable verification logs.

\item Validate the sensitivity of the computational pipeline through
controlled perturbation tests, confirming that violations are reliably detected
when essential geometric structure is removed.

\item Clearly delineate the scope of the certification, distinguishing the
fixed-scale regime studied here from geometrically distinct short-interval
problems that require separate analytical and experimental treatment.
\end{itemize}

\preto\section{\FloatBarrier}
\section{Fundamental Definitions and Identities}
\label{sec:fundamental-definitions}

The definitions and identities collected in this section are adapted from
\cite{Riemers2025-Framework,Riemers2026-PCGH}.
They are presented here in a condensed form sufficient for the present work;
for full derivations, motivation, and extended discussion, the reader is
referred to the cited papers.

\subsection{Admissible Parity}

The parity restriction was introduced in \cite{Riemers2025-Framework}.
Here it is used purely as an admissibility convention for windowed sums
and products.

\begin{definition}[Parity--Admissible Window Index Set]
\label{def:parity-admissible}
Let \(n\in\mathbb{N}\) and let \(M\in[0,n]\).
Define the symmetric window
\begin{equation}
I_M
:=
\{\, m\in\mathbb{Z} : 0<|m|\le M \text{ and } 3\le n-|m| \,\}.
\end{equation}
The \emph{parity--admissible index set} is
\begin{equation}
\Ipar(n;M)
:=
\{\, m\in I_M : n+m \equiv 1 \pmod{2} \,\}.
\end{equation}
Equivalently, \(\Ipar(n;M)\) consists of those shifts \(m\) with
\(0<|m|\le M\) for which both \(n-m\) and \(n+m\) are odd integers at least \(3\).
Unless stated otherwise, all summations over window variables \(m\) in
this paper are implicitly restricted to \(m\in\Ipar(n;M)\).
\end{definition}

\begin{definition}[Euler--Cap Operator]
\label{def:euler-cap-operator}
The PCGC--Goldbach conjecture is formulated only for
\emph{Euler--cap admissible} window radii.
To allow formulas to be written uniformly in terms of an unconstrained
window parameter, the following coercion operator is introduced.

Fix \(n\in\mathbb{N}\).  For any \(M\ge 0\) measured in the same units as the
window radius, define
\begin{equation}
\label{eq:ecap-operator}
\ecap{M}
\;:=\;
\left\lfloor
\min\!\left(
M,\;
\Ecap(n)\,n
\right)
\right\rfloor
\;=\;
\left\lfloor
\min\!\left(
M,\;
\frac{(2n+1)-\sqrt{8n+1}}{2}
\right)
\right\rfloor.
\end{equation}

Thus \(\ecap{M}\) denotes the largest Euler--cap admissible window radius
not exceeding \(M\).
\end{definition}

\medskip
\noindent
\textbf{Convention.}
The Euler--cap operator \(\ecap{\cdot}\) enforces admissibility of window
parameters.  This paper assumes the Euler cap is applied globally and
will therefore not be explicitly specified.

\subsection{Effective Local Moduli}

\begin{definition}[Effective Local Modulus]
\label{def:Qp}
Let \(n\in\mathbb{N}\) and let \(p\) be an odd prime.
Define the minimum contributing prime by
\begin{equation}
p_{\min}(n)
   :=
\begin{cases}
3, & 3 \mid n, \\[4pt]
5, & 3 \nmid n.
\end{cases}
\end{equation}

For each \(n\ge2\), let
\begin{equation}
\Peff(n):=\{\,p\in\mathbb{P} : p\ge \pmin(n)\,\}.
\end{equation}
We regard \(\EffLocMod_p(n)\) as defined only for \(p\in\Peff(n)\).

For any odd prime \(q_{\min}\), define the partial Euler product
\begin{equation}
\EffLocMod_p^{(q_{\min})}
   :=
   \prod_{\substack{
      q \in \mathbb{P} \\
      q_{\min} \le q \le p
   }}
   (q-1).
\end{equation}
The \emph{effective local modulus} at \(p\) for the even integer \(2n\) is
then
\begin{equation}
\EffLocMod_p(n)
   := \EffLocMod_p^{\bigl(p_{\min}(n)\bigr)}.
\end{equation}
\end{definition}
Thus \(\EffLocMod_p(n)\) encodes the cumulative residue structure imposed by the
odd primes below \(p\), with the only \(n\)-dependence arising from the
choice of base prime \(p_{\min}(n)\) according to whether \(3\mid n\).
This convention is tailored to the singular--series geometry underlying
Goldbach--type problems.

\begin{definition}[Effective Moduli Interval Max]
\begin{align}
\EffLocMod(2n;L) 
&:= \EffLocMod_{P_0(2n;L)}(n), \\[3pt]
 P_0(2n;L)
 &:= \max\{\,p\in\Peff(n) :\ p\mid n, \QpleL\,\},
\\ 
& \text{with } P_0(2n;L):=\pmin(n)\text{ if the set is empty.}
\end{align}
\end{definition}

\subsection{Prime Curvature Constants}

The prime curvature constants encode the cumulative
medium-- and large--prime contribution to the geometric remainder envelope.

\begin{definition}[Prime Curvature Constants]
Fix the base prime \(q_{\min}=5\), and recall the auxiliary effective moduli
\(\EffLocMod_p^{(5)}\) from Definition~\ref{def:Qp}.  
Define the \emph{prime curvature constant} by the convergent Euler product
\begin{equation}
\label{eq:def-OmegaPrime}
\OmegaPrime
   :=
   \prod_{\substack{p\in\Peff(5)}}
   (p-2)^{1/\EffLocMod_p^{(5)}}
\end{equation}

For each even integer \(2n\) and Euler--cap admissible window scale \(L\),
define the \emph{renormalized prime curvature factor} by
\begin{equation}
\label{eq:def-OmegaPrimeNorm}
\OmegaPrimeNorm(2n;L)
   :=
   \OmegaPrime^{\kappa(n)}
   \prod_{\substack{p\in\Peff(n)\\ \EffLocMod_p(n)\le L}}
      (p-2)^{-1/\EffLocMod_p(n)}
\end{equation}
where the exponent \(\kappa(n)\in\{\tfrac12,1\}\) compensates for the
choice of base prime \(\pmin(n)\in\{3,5\}\) in Definition~\ref{def:Qp}.

Specifically, when \(3\nmid n\) one has \(\pmin(n)=5\) and \(\kappa(n)=1\).
When \(3\mid n\), the extra contribution from the prime
\(3\) is absorbed into the exponent \(\kappa(n)=\tfrac12\).
\end{definition}


\begin{definition}[Bounding Envelope Constants]
Then the envelope for \( L \ge \pmin(n) \) constant is given explicitly by
\begin{equation}
\label{eq:c-explicit}
c(2n;L)=c\!\bigl(2n;\EffLocMod_p(n)\bigr)
      \;=\;
      2\prod_{\substack{q \in \Peff(n) \\ \EffLocMod_q(n) > \EffLocMod_p(n)}} (q-2)^{\,\EffLocMod_p(n)/\EffLocMod_q(n)}.
\end{equation}
In particular, \(c(2n;L)\) is constant on each envelope interval
\([\EffLocMod_p(n),\,\EffLocMod_{p^+}(n))\).
\end{definition}

\subsection{Goldbach Singular Series Factors}

\begin{definition}[Goldbach Singular Series Factors]
\label{def:SGB-Ssem}
For an even integer \(2n\), the \emph{local semiprime correction
factor}~\cite{HardyLittlewood1923} is defined as
\begin{equation}
\Ssem(2n)
   :=
   \prod_{\substack{p\mid n \\ p>2}}
      \frac{p-1}{p-2}.
\end{equation}
Define the \emph{prime--pair constant}~\cite{HardyLittlewood1923} by
\begin{equation}
C_2
   :=
   \prod_{p>2}\left(1-\frac{1}{(p-1)^2}\right).
\end{equation}
The corresponding \emph{Goldbach singular series factor}~\cite{HardyLittlewood1923} is
\begin{equation}
\SGB(2n)
   :=
   2\,C_2\,\Ssem(2n).
\end{equation}
\end{definition}

\begin{definition}[Hardy-Littlewood Circle Method Correction Factor]
For \( n \ge 2 \) and \( n > M \ge 1 \), and a weight function \( \frac{1}{\log{n}} \), define,
 \begin{equation}
 \HLCorr(2n;M) 
 := \frac{1}{|\Ipar(n,M)|} \sum_{m \in \Ipar(n,M)} \frac{\log^2{n}}{\log{(n-m)}\log{(n+m)}}.
\end{equation}
\label{def:HLCorr}
\end{definition}

\begin{definition}[Measured Goldbach Count]
The measured number of Goldbach pairs in the window is
\begin{equation}
\Gmeas(2n;M)
:=
\sum_{m\in\Ipar(n,M)}
1_{\mathrm{prime}}(n-m)\,1_{\mathrm{prime}}(n+m).
\end{equation}
\end{definition}

\begin{definition}[Hardy--Littlewood Window Predictor (Hl--Windowed)]
\label{def:GHL}
Assume the standard prime weight
(e.g.\ \(\omega(x)=1/\log x\)).
Define
\begin{equation}
\GHL(2n;M)
:=
2\,C_2\,\Ssem(2n)\frac{2M}{\log^2{n}}\,\HLCorr(2n;M).
\end{equation}
where \(\Ssem(2n)\) is the classical Hardy--Littlewood semiprime correction.
\end{definition}

\subsection{Complementary and Full Euler--Type Products}

In addition to the local semiprime correction factor \(\Ssem(2n)\) defined
above, it is occasionally convenient to refer to the complementary and full
Euler--type products obtained by modifying the divisibility condition on the
prime index.

\begin{definition}[Complementary and Full Local Semiprime Products]
\label{def:Ssem-complement-full}
Define the \emph{complementary} local semiprime product by
\begin{equation}
\Ssem^{\complement}(2n)
   :=
   \prod_{\substack{p\in\Podd \\ p\nmid n}}
      \frac{p-1}{p-2},
\end{equation}
and the corresponding \emph{full} product by
\begin{equation}
\Ssem^{\bullet}(2n)
   :=
   \prod_{\substack{p\in\Podd}}
      \frac{p-1}{p-2}.
\end{equation}
\end{definition}

\begin{convention}[Notation]
A superscript \( \complement \) indicates replacement of the divisibility
condition \(p\mid n\) by \(p\nmid n\), while a superscript \( \bullet \)
indicates removal of the divisibility condition entirely. No additional
structure is implied.
\end{convention}

\subsection{Additional Series Operators}

The classical Hardy--Littlewood singular series \(\Ssem(2n)\) is naturally an
asymptotic object.  For finite--window analysis it is convenient to
introduce auxiliary operators that partition the series into base and
tail components relative to a square cutoff.

\begin{definition}[Cutoff Operators]
The square cutoff operator is defined as
\begin{equation}
\SquareCap(x):=x^2.
\end{equation}
The function \( \EffLocModCap(\cdot) \) may also be used as a cutoff operator.
\end{definition}

The tags \( \thead \) and \( \ttail \) denote the base (small--prime)
and tail (large--prime) components relative to a given cutoff.

\begin{definition}[Cutoff Components of \( \Ssem \)]
Let \( M>0 \) and \( n\ge 2 \). Then, cutoff terms are defined as
\begin{equation}
\SsemHead^{\SquareCap}(2n;M)
:= \prod_{\substack{p\in\Podd \\ p\mid n \\ p^2\le M}}
\frac{p-1}{p-2} , \qquad
\SsemTail^{\SquareCap}(2n;M)
:= \prod_{\substack{p\in\Podd \\ p\mid n \\ p^2> M}}
\frac{p-1}{p-2}.
\end{equation}
\begin{equation}
 \SsemHead^{\EffLocModCap}(2n;L)
:= \prod_{\substack{p\in\Peff(n) \\ p\mid n\\ \QpleL}}
\frac{p-1}{p-2}, \qquad
\SsemTail^{\EffLocModCap}(2n;L)
:= \prod_{\substack{p\in\Peff(n) \\ p\mid n \\ \EffLocMod_p(n) > L}}
\frac{p-1}{p-2}.
\end{equation}
\begin{equation}
\Ssem^{\SquareCap}(2n)
:= \Ssem(2n), \qquad
\Ssem^{\EffLocMod}(2n)
:= \prod_{\substack{p\in\Peff(n)\\ p\mid n}}
\frac{p-1}{p-2}.
\end{equation}

Empty products are interpreted as \(1\).
\end{definition}

\subsection{Specialized Remainder Decomposition}

This leads up to the PCGC--Goldbach Remainder.

\begin{definition}[PCGC--Goldbach Remainder]\label{def:RHL}
Define for \( M > 0 \),
\begin{align}
\RHL(2n;L)
&:=
2\,\frac{\EffLocMod(2n;L)}{\SsemHead^{\EffLocModCap,\complement}(2n;L)}\,
\bigl(\OmegaPrimeNorm(2n;L)\bigr)^L,
\qquad L := \sqrt{2M} \\[3pt]
&=
2\,\frac{\EffLocMod(2n;L)}{\SsemHead^{\EffLocModCap,\complement}(2n;L)}\,\Xi(2n;L),
\qquad \Xi(2n;L):=\bigl(\OmegaPrimeNorm(2n;L)\bigr)^L.
\end{align}
\end{definition}

\noindent
The associated bounding envelope is defined as:

\begin{definition}[PCGC--Goldbach Bound Envelope]\label{def:RHLbound}
Define for \( 2M \ge (\pmin(n)-1)^2 \),
\begin{equation}
\RHLbound(2n;L)
:=
\frac{c(2n;L)\,L}{\SsemHead^{\EffLocModCap,\complement}(2n;L)},
\qquad L := \sqrt{2M}.
\end{equation}
\end{definition}

\subsubsection*{Role of the Remainder Bound}

Lemma~(Overall Bounding Envelope) of \cite{Riemers2025-Framework} shows that
\begin{equation}
\RHL(2n;L)\le \RHLbound(2n;L)
\end{equation}
for all admissible \(L\).  The purpose of \(\RHLbound\) is therefore
\emph{pointwise control} (a certified upper bound), not approximation.
In particular, \(\RHLbound\) does not satisfy
\(\RHLbound(2n;L)/\RHL(2n;L)\to 1\), and will overestimate some values of \(\RHL(2n;L)\)
by a scale-dependent factor.

\subsection{Empirical Hl-Normalized Measurements}

\begin{definition}[Empirical Hl-Normalized Measurements (from Semiprime Survivors)]
\label{semiprime-survivors}

Defining the measured (pairs-scale) normalization factor
\begin{equation}
\Cmeas(2n;n\alpha)\ :=\ \frac{\log^2 n}{n\alpha}\,\GHL(2n;n\alpha),
\end{equation}
where \(n\alpha\) denotes the Goldbach window half-width used in the measurement.

\smallskip
\noindent
For \textbf{primorial-based binning}, small values are treated separately:
for \(n<15\), bins are singletons \([k,k+1)\), \(k=1,\dots,14\);
for \(15\le n<30\), a single block \([15,30)\) is used.

For \(n\ge30\), let \(p<q\) be consecutive primes with corresponding primorials
\(p\#\) and \(q\#\).
Each primorial plateau \([p\#,q\#)\) is partitioned into bins of width
\begin{equation}
w_p := \max\!\Bigl(15,\frac{p\#}{2p}\Bigr), \qquad
L_p := q\# - p\#, \qquad
K_p := \Bigl\lceil \frac{L_p}{w_p} \Bigr\rceil .
\end{equation}
Then nodes are defined as
\begin{equation}
a_j := p\# + \min(j\,w_p,\,L_p), \qquad j=0,1,\dots,K_p ,
\end{equation}
so that \(a_0=p\#\) and \(a_{K_p}=q\#\).
The indexed primorial bins are
\begin{equation}
B_{j,p\#} := [\,a_j,\ a_{j+1}\,), \qquad j=0,1,\dots,K_p-1,
\end{equation}
forming consecutive intervals of width \(w_p\), with the final bin truncated so
that its right endpoint is exactly \(q\#\).

Within each primorial bin \(B_{j,p\#}\), with \(n\) ranging over integers in the bin,
define
\begin{equation}
N_0:=\arg\min_{n\in B_{j,p\#}}\Cmeas(2n;n\alpha),\qquad
N_1:=\arg\max_{n\in B_{j,p\#}}\Cmeas(2n;n\alpha),
\end{equation}
and
\begin{align}
& \Cmeas_{\min}(j,p\#):=\Cmeas(2N_0;n\alpha),\\
& \Cmeas_{\max}(j,p\#):=\Cmeas(2N_1;n\alpha),\\
& \Cmeas_{\tavg}(j,p\#):=
\frac{1}{\card{B_{j,p\#}}}
\sum_{n\in B_{j,p\#}}\Cmeas(2n;n\alpha).
\end{align}
\end{definition}

\section{Main Conjecture}
\label{sec:main-conjecture}

\subsection{PCGC--Goldbach}

This work is to verify the PCGC--Goldbach, introduced in
\cite{Riemers2026-PCGH} over a finite range of data.

\begin{conjecture}[Prime Curvature Conjecture for Goldbach (Pcgc--Goldbach)]
\label{con:PCGC-Goldbach}
For every even integer \(2n \ge 4\) and every admissible window size
Euler Capped \( M \), the Goldbach pair counts satisfy
\begin{equation}
  \bigl|
    \Gmeas(2n;M) - \GHL(2n;M)
  \bigr|
  \le
  \RHL(2n;L),
  \qquad L := \sqrt{2M},
\end{equation}
where the remainder term \(\RHL(2n;L)\) is defined in
Definition~\ref{def:RHL}.
\end{conjecture}

\preto\section{\FloatBarrier}
\section{Aggregated Pair Count Extrema Analysis}

The analysis sections that follow describe empirical structure observed in 
the data, including higher-order regularities and qualitative trends.
These observations are intended to be heuristic and descriptive, and 
are not required for the certification results presented later, but are inherently
easier to independently verify.  Only the bounds and non-crossing conditions 
established in the Certification section are used to support formal claims.

Global extrema of Goldbach pair counts aggregated over each primorial 
window are analysed.  For each bin, we compare measured maxima and 
minima are compared against the corresponding PCGC--Goldbach bounds.

Across the full tested range \(2n < 23\# \approx 2.5852 \times 10^8\), no 
violations occur. Moreover, the extrema trace smooth curvature envelopes 
consistent with the theoretical predictions, providing strong evidence that 
the geometric model accurately captures aggregate behaviour.

Figures~\ref{fig:CminmaxavgPrim1},~\ref{fig:CminmaxavgPrim}, and
\ref{fig:CminmaxavgPrim015625} are plots of the measured values 
compared to the HL--Windowed prediction lines and PCGC--Goldbach 
bounds for \( \alpha = 1, 0.5, \) and \( 0.015625 \), respectively.

\pgfplotstableread[col sep=comma, trim cells]{pairrangejoin-23PR.5-1-v0.2.0.csv}\pairdata

\begin{figure}[ht]
\centering
\begin{tikzpicture}

\pgfplotsset{every axis/.append style={title style={at={(0.98,0.97)},anchor=north east}}}

\begin{axis}[
  title={\(\,\alpha=1\)},
  width=\textwidth, height=0.42\textwidth,
  xmode=log,
  xlabel={\(n\) (log scale)},
  ylabel={\(C\) values},
  xmin=10,
  xmax=1.2e8,
  ymin=0,
  ymax=15,
  grid=both, tick align=outside,
  legend style={at={(0.5,1.05)}, anchor=south, legend columns=-1},
  unbounded coords=discard, filter discard warning=false,
  set layers,
  axis on top=false,  % so the band stays behind gridlines
]

% ----- styles -----
\pgfplotsset{
  myband/.style   ={fill=orange!50, fill opacity=0.30, draw=none, on layer=axis background},
  mypred/.style   ={no markers, thick, dotted},
  myavg/.style    ={no markers, semithick, color=black!70},
  mydots/.style   ={only marks, mark=*, mark size=0.7pt, opacity=0.30},
  guide/.style    ={gray!45, very thin, on layer=axis foreground}, % draw on top
}

% ----- band (draw first) -----
\addplot[name path=Cminpath, draw=orange!30!orange, opacity=0.5, forget plot]
  table[x={n_0}, y={C_min}] {\pairdata};
\addplot[name path=Cmaxpath, draw=orange!30!orange, opacity=0.5, forget plot]
  table[x={n_1}, y={C_max}] {\pairdata};
\addplot[myband, forget plot] fill between[of=Cminpath and Cmaxpath];

% ----- main lines (legend entries) -----
\addplot[myavg]
  table[x={n_geom}, y={C_avg}] {\pairdata};
\addlegendentry{\( \Cmeas_{\tavg} \) (bin line)}

\addplot[mypred, thick, purple!70!black, dotted, sharp plot]
  table[x={n_a}, y={Cbound_min}] {\pairdata};
\addlegendentry{\( \Cpred_{\min}^{\tbound} \)}

\addplot[mypred, thick, purple!70!black, dotted, sharp plot]
  table[x={n_b}, y={Cbound_max}] {\pairdata};
\addlegendentry{\( \Cpred_{\max}^{\tbound} \)}

\addplot[mypred, thick, blue!70!black, dashed, sharp plot]
  table[x={Npred_0}, y={Cpred_min}] {\pairdata};
\addlegendentry{\( \Cpred_{\min} \)}

\addplot[mypred, thick, green!50!black, dashed, sharp plot]
  table[x={n_geom}, y={Cpred_avg}] {\pairdata};
\addlegendentry{\( \Cpred_{\tavg} \)}

\addplot[mypred, red!70!black, dashed, sharp plot]
  table[x={n_1}, y={Cpred_max}] {\pairdata};
\addlegendentry{\( \Cpred_{\max} \)}

% style
\pgfplotsset{primorial/.style={color=violet!70!black, densely dashdotted, line width=0.6pt, opacity=0.9}}

% draw them last, and don't add to legend
\foreach \x in {15,105,1155,15015,255255,4849845,111546435}{
  \addplot[primorial, forget plot] coordinates {(\x,0) (\x,14)};
}

\end{axis}
\end{tikzpicture}
\caption{Scatter plots of \( \Cmeas_{\min} \), \( \Cmeas_{\max} \), and \( \Cmeas_{\tavg} \) versus \( n \) with HL-A prediction lines for \( \alpha = 1 \).   The \( \Cpred_{\min}^{\tbound} \) and \( \Cpred_{\max}^{\tbound} \) curve represents the maximum amount of cancellation that can result from small primes from PCGC--Goldbach.}
\label{fig:CminmaxavgPrim1}
\end{figure}

\pgfplotstableread[col sep=comma, trim cells]{pairrangejoin-23PR.5-0.5-v0.2.0.csv}\pairdata

\begin{figure}[ht]
\centering
\begin{tikzpicture}
\pgfplotsset{every axis/.append style={title style={at={(0.98,0.97)},anchor=north east}}}

\begin{axis}[
  title={\(\,\alpha=0.5\)},
  width=\textwidth, height=0.42\textwidth,
  xmode=log,
  xlabel={\(n\) (log scale)},
  ylabel={\(C\) values},
  xmin=10,
  xmax=1.2e8,
  ymin=0,
  ymax=15,
  grid=both, tick align=outside,
  legend style={at={(0.5,1.05)}, anchor=south, legend columns=-1},
  unbounded coords=discard, filter discard warning=false,
  set layers,
  axis on top=false,  % so the band stays behind gridlines
]

% ----- styles -----
\pgfplotsset{
  myband/.style   ={fill=orange!50, fill opacity=0.30, draw=none, on layer=axis background},
  mypred/.style   ={no markers, thick, dotted},
  myavg/.style    ={no markers, semithick, color=black!70},
  mydots/.style   ={only marks, mark=*, mark size=0.7pt, opacity=0.30},
  guide/.style    ={gray!45, very thin, on layer=axis foreground}, % draw on top
}

% ----- band (draw first) -----
\addplot[name path=Cminpath, draw=orange!30!orange, opacity=0.5, forget plot]
  table[x={n_0}, y={C_min}] {\pairdata};
\addplot[name path=Cmaxpath, draw=orange!30!orange, opacity=0.5, forget plot]
  table[x={n_1}, y={C_max}] {\pairdata};
\addplot[myband, forget plot] fill between[of=Cminpath and Cmaxpath];

% ----- main lines (legend entries) -----
\addplot[myavg]
  table[x={n_geom}, y={C_avg}] {\pairdata};
\addlegendentry{\( \Cmeas_{\tavg} \) (bin line)}

\addplot[mypred, thick, purple!70!black, dotted, sharp plot]
  table[x={n_a}, y={Cbound_min}] {\pairdata};
\addlegendentry{\( \Cpred_{\min}^{\tbound} \)}

\addplot[mypred, thick, purple!70!black, dotted, sharp plot]
  table[x={n_b}, y={Cbound_max}] {\pairdata};
\addlegendentry{\( \Cpred_{\max}^{\tbound} \)}

\addplot[mypred, thick, blue!70!black, dashed, sharp plot]
  table[x={Npred_0}, y={Cpred_min}] {\pairdata};
\addlegendentry{\( \Cpred_{\min} \)}

\addplot[mypred, thick, green!50!black, dashed, sharp plot]
  table[x={n_geom}, y={Cpred_avg}] {\pairdata};
\addlegendentry{\( \Cpred_{\tavg} \)}

\addplot[mypred, red!70!black, dashed, sharp plot]
  table[x={n_1}, y={Cpred_max}] {\pairdata};
\addlegendentry{\( \Cpred_{\max} \)}

% style
\pgfplotsset{primorial/.style={color=violet!70!black, densely dashdotted, line width=0.6pt, opacity=0.9}}

% draw them last, and don't add to legend
\foreach \x in {15,105,1155,15015,255255,4849845,111546435}{
  \addplot[primorial, forget plot] coordinates {(\x,0) (\x,14)};
}

\end{axis}
\end{tikzpicture}
\caption{Scatter plots of \( \Cmeas_{\min} \), \( \Cmeas_{\max} \), and \( \Cmeas_{\tavg} \) versus \( n \) with HL-A prediction lines for \( \alpha = 0.5 \).   The \( \Cpred_{\min}^{\tbound} \) and \( \Cpred_{\max}^{\tbound} \) curve represents the maximum amount of cancellation that can result from small primes from PCGC--Goldbach.}
\label{fig:CminmaxavgPrim}
\end{figure}

\pgfplotstableread[col sep=comma, trim cells]{pairrangejoin-23PR.5-0.015625-v0.2.0.csv}\pairdata

\begin{figure}[ht]
\centering
\begin{tikzpicture}

\pgfplotsset{every axis/.append style={title style={at={(0.98,0.97)},anchor=north east}}}

\begin{axis}[
  title={\(\,\alpha=0.015625\)},
  width=\textwidth, height=0.42\textwidth,
  xmode=log,
  xlabel={\(n\) (log scale)},
  ylabel={\(C\) values},
  xmin=10000,
  xmax=1.2e8,
  ymin=0,
  ymax=15,
  grid=both, tick align=outside,
  legend style={at={(0.5,1.05)}, anchor=south, legend columns=-1},
  unbounded coords=discard, filter discard warning=false,
  set layers,
  axis on top=false,  % so the band stays behind gridlines
]

% ----- styles -----
\pgfplotsset{
  myband/.style   ={fill=orange!50, fill opacity=0.30, draw=none, on layer=axis background},
  mypred/.style   ={no markers, thick, dotted},
  myavg/.style    ={no markers, semithick, color=black!70},
  mydots/.style   ={only marks, mark=*, mark size=0.7pt, opacity=0.30},
  guide/.style    ={gray!45, very thin, on layer=axis foreground}, % draw on top
}

% ----- band (draw first) -----
\addplot[name path=Cminpath, draw=orange!30!orange, opacity=0.5, forget plot]
  table[x={n_0}, y={C_min}] {\pairdata};
\addplot[name path=Cmaxpath, draw=orange!30!orange, opacity=0.5, forget plot]
  table[x={n_1}, y={C_max}] {\pairdata};
\addplot[myband, forget plot] fill between[of=Cminpath and Cmaxpath];

% ----- main lines (legend entries) -----
\addplot[myavg]
  table[x={n_geom}, y={C_avg}] {\pairdata};
\addlegendentry{\( \Cmeas_{\tavg} \) (bin line)}

\addplot[mypred, thick, purple!70!black, dotted, sharp plot]
  table[x={n_a}, y={Cbound_min}] {\pairdata};
\addlegendentry{\( \Cpred_{\min}^{\tbound} \)}

\addplot[mypred, thick, purple!70!black, dotted, sharp plot]
  table[x={n_b}, y={Cbound_max}] {\pairdata};
\addlegendentry{\( \Cpred_{\max}^{\tbound} \)}

\addplot[mypred, thick, blue!70!black, dashed, sharp plot]
  table[x={Npred_0}, y={Cpred_min}] {\pairdata};
\addlegendentry{\( \Cpred_{\min} \)}

\addplot[mypred, thick, green!50!black, dashed, sharp plot]
  table[x={n_geom}, y={Cpred_avg}] {\pairdata};
\addlegendentry{\( \Cpred_{\tavg} \)}

\addplot[mypred, red!70!black, dashed, sharp plot]
  table[x={n_1}, y={Cpred_max}] {\pairdata};
\addlegendentry{\( \Cpred_{\max} \)}

% style
\pgfplotsset{primorial/.style={color=violet!70!black, densely dashdotted, line width=0.6pt, opacity=0.9}}

% draw them last, and don't add to legend
\foreach \x in {15,105,1155,15015,255255,4849845,111546435}{
  \addplot[primorial, forget plot] coordinates {(\x,0) (\x,24)};
}


\end{axis}
\end{tikzpicture}
\caption{Scatter plots of \( \Cmeas_{\min} \), \( \Cmeas_{\max} \), and \( \Cmeas_{\tavg} \) versus \( n \) with HL-A prediction lines for \( \alpha = 0.015625 \).  The \( \Cpred_{\min}^{\tbound} \) and \( \Cpred_{\max}^{\tbound} \) curve represents the maximum amount of cancellation that can result from small primes from PCGC--Goldbach.}
\label{fig:CminmaxavgPrim015625}
\end{figure}

Readers who find these plots visually underwhelming have not missed the point.
The absence of dramatic features is itself informative.

A sharp change in curvature is clearly visible in the theoretical lower bounding curve
whenever a new small prime enters the head term.
This transition reflects a discrete structural update in the bound construction.
Notably, no corresponding feature appears in the measured data, which remain smooth across
the same transition.

Heuristic or near-fitting bounds are easy to construct, particularly when parameters are freely adjusted.
By contrast, parameter-free bounds that are intentionally extremely loose at small and moderate values of \(n\),
yet still tighten and converge asymptotically, are substantially more difficult to obtain.

This behaviour is essential to their validity.  When Goldbach pair counts exhibit their largest relative variability, 
the bounds impose the weakest constraints.   Conversely, as \(n\) increases and the intrinsic variability 
decreases, the bounds tighten asymptotically.

For example, with \(\alpha = 0.5\), the theoretical onset scale \(n_0\) for the bound
is approximately \(2100\),
whereas the measured data exhibit stable behaviour more than two orders of magnitude earlier.
This substantial buffer renders the lower bound effectively trivial over most of the plotted
range and ensures that early irregularities cannot threaten certification.

As a consequence, the asymptotic regime in which the geometric behaviour becomes apparent
occupies only the final decades of the plots.
Earlier regions are dominated by deliberate slack in the bounds rather than by meaningful
geometric constraint.

The \(\Lambda\)-analysis of the following sections reveals the strength of the geometric model in a way 
that extrema plots alone cannot.

\needspace{10\baselineskip}

\preto\subsection{\FloatBarrier}
\subsection{Definition of the Empirical Range Statistic}

For each primorial bin \(B_{j,p\#}\), the Hardy--Littlewood--normalized
Goldbach counts are considered.  Within each bin \(B_{j,p\#}\), let
\begin{equation}
n_{\min}^{j,p\#,\alpha} \in \arg\min_{n \in B_{j,p\#}} C(2n; n\alpha),
\qquad
n_{\max}^{j,p\#,\alpha} \in \arg\max_{n \in B_{j,p\#}} C(2n; n\alpha),
\end{equation}
denote values of \(n\) at which the normalized count attains its minimum and
maximum, respectively.
The corresponding measured extrema measurements are defined as
\begin{align}
\Cmeas_{\min}^{j,p\#}(\alpha)
&:= \frac{\Gmeas(2n_{\min}^{j,p\#,\alpha};n\alpha)\log^2 n_{\min}^{j,p\#,\alpha}}{n\alpha}, \\
\Cmeas_{\max}^{j,p\#}(\alpha)
&:= \frac{\Gmeas(2n_{\max}^{j,p\#,\alpha};n\alpha)\log^2 n_{\max}^{j,p\#,\alpha}}{n\alpha}.
\end{align}

The associated bound predictions, evaluated over the same bin, are defined by
\begin{align}
\Cpred_{\min}^{\tbound,j,p\#}(\alpha)
&:= \min_{n \in B_{j,p\#}}
\left(\GHL(2n;n\alpha)-\RHL(2n;\sqrt{2n\alpha})\right)\frac{\log^2 n}{n\alpha}, \\
\Cpred_{\max}^{\tbound,j,p\#}(\alpha)
&:= \max_{n \in B_{j,p\#}}
\left(\GHL(2n;n\alpha)+\RHL(2n;\sqrt{2n\alpha})\right)\frac{\log^2 n}{n\alpha}.
\end{align}

then the logarithmic deviations are defined as
\begin{equation}
\Lambda_{\min}^{\tbound,j,p\#}(\alpha)
:= \log\!\left(
\frac{\Cmeas_{\min}^{j,p\#}(\alpha)}{\Cpred_{\min}^{\tbound,j,p\#}(\alpha)}
\right),
\qquad
\Lambda_{\max}^{\tbound,j,p\#}(\alpha)
:= \log\!\left(
\frac{\Cmeas_{\max}^{j,p\#}(\alpha)}{\Cpred_{\max}^{\tbound,j,p\#}(\alpha)}
\right).
\end{equation}

To visualize finite-scale variability while suppressing bin-to-bin noise,
we enumerate all primorial bins \(B_{j,p\#}\) in increasing order of their
left endpoints and denote the resulting ordered sequence by \(\{B_k\}_{k\ge0}\).
This ordering may cross primorial boundaries.
For each \(k\), the quantities
\(\Lambda_{\min}^{\tbound,k}(\alpha)\) and
\(\Lambda_{\max}^{\tbound,k}(\alpha)\) are defined to be the corresponding
values associated with the bin \(B_k\).

Then group consecutive bins into blocks of fixed size.
For each block index \(\ell \ge 0\), the interval is defined as
\begin{equation}
\label{eq:K-ell-def}
\mathcal{K}_\ell := \{\, k : 12\ell \le k < 12(\ell+1) \,\}.
\end{equation}
The range extrema over each block are defined as
\begin{align}
\Lambda_{\min}^{\tlow,\ell}(\alpha)
&:= \min_{k \in \mathcal{K}_\ell} \Lambda_{\min}^{\tbound,k}(\alpha),
&
\Lambda_{\min}^{\thigh,\ell}(\alpha)
&:= \max_{k \in \mathcal{K}_\ell} \Lambda_{\min}^{\tbound,k}(\alpha), \\
\Lambda_{\max}^{\tlow,\ell}(\alpha)
&:= \min_{k \in \mathcal{K}_\ell} \Lambda_{\max}^{\tbound,k}(\alpha),
&
\Lambda_{\max}^{\thigh,\ell}(\alpha)
&:= \max_{k \in \mathcal{K}_\ell} \Lambda_{\max}^{\tbound,k}(\alpha).
\end{align}

The corresponding \(n\)-values plotted on the horizontal axis are taken to be
the values \(n_{\min}^k\) and \(n_{\max}^k\) at which these extremal deviations
occur within each block.
Thus, each range block summarizes the spread of the deviations
\(\Lambda_{\min}^{\tbound,k}\) and \(\Lambda_{\max}^{\tbound,k}\)
over a fixed collection of consecutive primorial bins, rather than
representing a pointwise interval at a single value of \(n\).

\begin{remark}[Choice of Block Size]
The block size of \(12\) consecutive primorial bins was chosen for pragmatic
and statistical reasons.
First, sample sizes on the order of ten are commonly regarded as a minimal
scale at which dispersion statistics become stable, even in non-Gaussian
settings.
Second, grouping bins at this scale provides sufficient data reduction to
allow multiple range bands to be displayed simultaneously without obscuring
large-scale structure.
Empirically, smaller block sizes lead to visually noisy envelopes, while
larger block sizes suppress finite-scale features that remain relevant to
the interpretation of the data.
\end{remark}

\preto\subsection{\FloatBarrier}
\subsection{Minimum Geometric Bounds}

Figure~\ref{fig:min-bounds-envelope} plots eleven representative summary ranges
for \(\Lambda_{\min}^{\tbound,k}\).
PCGC--Goldbach predicts that these curves approach zero asymptotically, but it
does not impose strong restrictions on their detailed shape beyond the
requirement that the corresponding ratios remain strictly non-negative.
The observed overall decreasing trend is consistent with the predictions of
PCGC--Goldbach.

The constants defining \(\RHLbound\) are chosen so that
\begin{equation}
\RHLbound(2n;\EffLocMod_p(n))
=
\RHL(2n;\EffLocMod_p(n))
\qquad
\forall\, n \in \mathbb{N}, \quad \forall\, p \in \Peff(n).
\end{equation}

Since \(\RHLbound(2n;L)\) decreases monotonically with \(n\), for all
81 values of \(\alpha\) we plot \(\Lambda_{\min}^{\tbound}\) using
\begin{equation}
L \approx \EffLocMod_p(n),
\qquad
3 \mid n,
\qquad
p \in \{11,13\}.
\end{equation}
To avoid division by zero and undefined values of \(\log(0)\), such cases are
treated as unplottable.
The approximation is taken to be the closest available value to equality.
If no plottable value exists on both sides of the target
\(\EffLocMod_p(n)\), the corresponding point is also considered unplottable.

Figure~\ref{fig:lambdabound-cert-min} plots all 81 values of \(\alpha\).
The smallest \(\alpha\)-values due to low statistics exhibit excessive 
finite-scale variability (jitter), making it difficult to confirm a monotonic trend.
For the remaining values of \(\alpha\), the curves are clearly decreasing.
Table~\ref{tab:lambdabound-audit-min} lists representative values for twelve
selected \(\alpha\)-levels.  The table includes all \(\alpha\) values that are powers of \(2\)
(i.e., \(2^{-n}\) for integer \(n\)), together with the second lowest \(\alpha\) value
to illustrate the change in standard deviation.

Since \(\RHL \ge \lvert \varepsilon \rvert\), and the plotted quantities are only loosely tied
to the scale \(\lvert \varepsilon \rvert\), strict inequalities and asymptotic convergence are
required only once squeezed sufficiently by the upper and lower bounds.
Nevertheless, the observed decreasing trend provides additional empirical
support for the conjecture.

Including unplottable data, the inequality \(\RHL \ge \lvert \varepsilon \rvert\) holds
without exception across the entire computed range.

\pgfplotstableread[col sep=comma, trim cells]{lambdaboundmin-summary-23PR.5-v0.2.0.csv}\banddata

\begin{figure}[ht]
\centering
\begin{tikzpicture}
\begin{axis}[
  width=\linewidth,
  height=0.63\linewidth,
  xlabel={\(n\)},
  ylabel={\(\Lambda_{\min}^{\tbound}\)},
  xmin=100,
  xmax=1.2e8,
  ymax=10,
  ymin=1e-3,
  xmode=log,
  ymode=log,
  grid=both,
  legend cell align=left,
  legend style={
    at={(0.02,0.02)},
    anchor=south west,
    draw=none,
    fill=white,
    fill opacity=0.85,
    text opacity=1,
    font=\small,
    /tikz/align=left,
  }
]

\PlotBand{1.0}{bandA}{1}
%\PlotBand{0.5}{bandB}{1/2}
\PlotBand{0.25}{bandC}{1/4}
%\PlotBand{0.125}{bandD}{1/8}
\PlotBand{0.0625}{bandE}{1/16}
%\PlotBand{0.03125}{bandF}{1/32}
\PlotBand{0.015625}{bandG}{1/64}
%\PlotBand{0.0078125}{bandH}{1/128}
\PlotBand{0.00390625}{bandI}{1/256}
%\PlotBand{0.001953125}{bandJ}{1/512}
\PlotBand{0.0009765625}{bandK}{1/1024}

\end{axis}
\end{tikzpicture}
\caption{Smooth envelope bands for \( \Lambda_{\min}^{\tbound} \) by \( \alpha \).}
\label{fig:min-bounds-envelope}
\end{figure}


\pgfplotstableread[col sep=comma]{lambdabound-cert-23PR.5-v0.2.0.csv}\eventdataA

\begin{figure}[ht]
\centering
\begin{tikzpicture}
\begin{axis}[
  width=\linewidth,
  height=0.30\linewidth,
  xlabel={\(\alpha\)},
  ylabel={\(\Lambda_{\min}^{\tbound}\)},
  xmode=log,
  ymode=log,
  xmin=0.0009765625,
  xmax=1.0,
  ymin=1e-3,
  ymax=0.5,
  grid=both,
  legend style={
    at={(0.02,0.02)},
    anchor=south west,
    draw=none,
    fill=none,
    font=\small,
    cells={anchor=west},
  },
  legend cell align=left,
]

% ---- styles ----
\pgfplotsset{
  finalLine/.style={thick, solid, opacity=0.75, mark=none},
  peleven/.style={only marks, mark=triangle*, mark size=1.7pt},
  pthirteen/.style={only marks, mark=square*,   mark size=1.6pt},
}

% Drop empty y-cells (treat as NaN so pgfplots discards the coordinate)
\pgfplotsset{
  y drop empty/.style={
    y filter/.code={
      \edef\temp{\pgfmathresult}%
      \ifx\temp\empty
        \def\pgfmathresult{nan}%
      \fi
    }
  }
}

% ---- p = 11 : markers only ----
\addplot[peleven, y drop empty]
  table[x=alpha, y=L11_lo] {\eventdataA};
\addlegendentry{\(p = 11\)}

% ---- p = 13 : markers only ----
\addplot[pthirteen, y drop empty]
  table[x=alpha, y=L13_lo] {\eventdataA};
\addlegendentry{\(p = 13\)}

% ---- final : line only ----
\addplot[finalLine, y drop empty]
  table[x=alpha, y=Lfinal_lo] {\eventdataA};
\addlegendentry{final}
\end{axis}
\end{tikzpicture}
\caption{All $\alpha$ values showing event and final envelope $\Lambda_{\min}^{\tbound}$ statistics.}
\label{fig:lambdabound-cert-min}
\end{figure}

\pgfplotstableread[col sep=comma]{lambdabound-audit-23PR.5-v0.2.0.csv}\auditdataA

\begin{table}[t]
\centering
\caption{Selected $\alpha$ values showing event and final envelope $\Lambda_{\min}^{\tbound}$  statistics.}
\label{tab:lambdabound-audit-min}
\pgfplotstabletypeset[
  empty cells with={---},
  columns={alpha,L11_lo,L13_lo,Lfinal_lo,Lfinal_lo_std},
  columns/alpha/.style={column name={$\alpha$}, fixed, precision=12,column type=l},
  every head row/.style={before row=\toprule, after row=\midrule},
  every last row/.style={after row=\bottomrule},
  columns/L11_lo/.style={column name={$p{=}11$}, sci, sci zerofill, precision=2},
  columns/L13_lo/.style={column name={$p{=}13$}, sci, sci zerofill, precision=2},
  columns/Lfinal_lo/.style={column name={final}, sci, sci zerofill, precision=2,column type={r@{$\,\pm\,$}}},
  columns/Lfinal_lo_std/.style={column name={std.dev.}, sci, sci zerofill, precision=2, column type=l},
]{\auditdataA}
\end{table}

\preto\subsection{\FloatBarrier}
\subsection{Maximum Geometric Bounds}

The complementary analysis for the upper geometric bounds is done next.
All definitions, binning conventions, and range constructions are exactly as
in the preceding subsection, with \(\Lambda_{\max}^{\tbound,k}\) replacing
\(\Lambda_{\min}^{\tbound,k}\).
Accordingly, only the behaviour of the resulting range summaries
and their consistency with the predictions of PCGC--Goldbach are considered.

Figure~\ref{fig:max-bounds-envelope} plots eleven representative summary ranges
for \(\Lambda_{\max}^{\tbound,k}\).
As in the lower-bound case, PCGC--Goldbach predicts eventual convergence
toward zero, while imposing no detailed constraints on the intermediate
shape of the curves beyond positivity of the underlying ratios.
The observed envelopes exhibit the same overall decreasing trend, consistent
with this prediction.

The plotting procedure mirrors that used for the minimum bounds.
Since \(\RHLbound(2n;L)\) is monotone in \(n\), the approximation
\(L \approx \EffLocMod_p(n)\) is again employed, with the same restrictions on
admissible values of \(n\) and the same criteria for excluding unplottable
points.
Figure~\ref{fig:lambdabound-cert-max} displays the full set of 81
\(\alpha\)-values.
As before, the smallest values of \(\alpha\) are dominated by finite-scale
variability, while the remaining curves show a clear decreasing trend.
Table~\ref{tab:lambdabound-audit-max} lists representative values for twelve
selected \(\alpha\)-levels.

Including unplottable data, no violations of the inequality
\(\RHL \ge \lvert \varepsilon \rvert\) are observed across the entire computed range.

\pgfplotstableread[col sep=comma, trim cells]{lambdaboundmax-summary-23PR.5-v0.2.0.csv}\banddata

\begin{figure}[ht]
\centering
\begin{tikzpicture}
\begin{axis}[
  width=\linewidth,
  height=0.63\linewidth,
  xlabel={\(n\)},
  ylabel={\(\Lambda_{\max}^{\tbound}\)},
  xmin=7,
  xmax=1.2e8,
  ymax=3,
  ymin=1e-3,
  xmode=log,
  ymode=log,
  grid=both,
  legend cell align=left,
  legend style={
    at={(0.02,0.02)},
    anchor=south west,
    draw=none,
    fill=white,
    fill opacity=0.85,
    text opacity=1,
    font=\small,
    /tikz/align=left,
  }
]

\PlotBand{1.0}{bandA}{1}
%\PlotBand{0.5}{bandB}{1/2}
\PlotBand{0.25}{bandC}{1/4}
%\PlotBand{0.125}{bandD}{1/8}
\PlotBand{0.0625}{bandE}{1/16}
%\PlotBand{0.03125}{bandF}{1/32}
\PlotBand{0.015625}{bandG}{1/64}
%\PlotBand{0.0078125}{bandH}{1/128}
\PlotBand{0.00390625}{bandI}{1/256}
%\PlotBand{0.001953125}{bandJ}{1/512}
\PlotBand{0.0009765625}{bandK}{1/1024}

\end{axis}
\end{tikzpicture}
\caption{Smooth envelope bands for \( \Lambda_{\max}^{\tbound} \) by \( \alpha \).}
\label{fig:max-bounds-envelope}
\end{figure}


\begin{figure}[ht]
\centering
\begin{tikzpicture}
\begin{axis}[
  width=\linewidth,
  height=0.30\linewidth,
  xlabel={\(\alpha\)},
  ylabel={\(-\Lambda_{\max}^{\tbound}\)},
  xmode=log,
  ymode=log,
  xmin=0.0009765625,
  xmax=1.0,
  ymin=1e-3,
  ymax=0.5,
  grid=both,
  unbounded coords=discard,
  filter discard warning=false,
  legend style={
    at={(0.02,0.02)},
    anchor=south west,
    draw=none,
    fill=none,
    font=\small,
    cells={anchor=west},
  },
  legend cell align=left,
]

% ---------- styles ----------
\pgfplotsset{
  final/.style={thick, mark=none},
  p11/.style={only marks, mark=triangle*, mark size=1.8pt},
  p13/.style={only marks, mark=square*,   mark size=1.8pt},
}

% ---------- p = 11 : triangles only ----------
\addplot[
  p11,
  x filter/.code={%
    \edef\temp{\thisrow{L11_hi}}%
    \ifx\temp\empty
      \def\pgfmathresult{nan}%
    \fi
  },
] table[
  x=alpha,
  y expr=abs(\thisrow{L11_hi}),
]{\eventdataA};
\addlegendentry{\( p=11 \)}

% ---------- p = 13 : squares only ----------
\addplot[
  p13,
  x filter/.code={%
    \edef\temp{\thisrow{L13_hi}}%
    \ifx\temp\empty
      \def\pgfmathresult{nan}%
    \fi
  },
] table[
  x=alpha,
  y expr=abs(\thisrow{L13_hi}),
]{\eventdataA};
\addlegendentry{\( p=13 \)}

% ---------- final : solid line only ----------
\addplot[
  final,
  x filter/.code={%
    \edef\temp{\thisrow{Lfinal_hi}}%
    \ifx\temp\empty
      \def\pgfmathresult{nan}%
    \fi
  },
] table[
  x=alpha,
  y expr=abs(\thisrow{Lfinal_hi}),
]{\eventdataA};
\addlegendentry{final}

\end{axis}
\end{tikzpicture}
\caption{All $\alpha$ values showing event and final envelope $\Lambda_{\max}^{\tbound}$ statistics.}
\label{fig:lambdabound-cert-max}
\end{figure}

\begin{table}[t]
\centering
\caption{Selected $\alpha$ values showing event and final envelope $\Lambda_{\max}^{\tbound}$ statistics.}
\label{tab:lambdabound-audit-max}
\pgfplotstabletypeset[
  empty cells with={---},
  columns={alpha,L11_hi,L13_hi,Lfinal_hi,Lfinal_hi_std},
  columns/alpha/.style={column name={$\alpha$}, fixed, precision=12,column type=l},
  every head row/.style={before row=\toprule, after row=\midrule},
  every last row/.style={after row=\bottomrule},
  columns/L11_hi/.style={column name={$p{=}11$}, sci, sci zerofill, precision=2},
  columns/L13_hi/.style={column name={$p{=}13$}, sci, sci zerofill, precision=2},
  columns/Lfinal_hi/.style={column name={final}, sci, sci zerofill, precision=2,column type={r@{$\,\pm\,$}}},
  columns/Lfinal_hi_std/.style={column name={std.dev.}, sci, sci zerofill, precision=2, column type=l},
]{\auditdataA}
\end{table}

\preto\subsection{\FloatBarrier}
\subsection{Origin of the Structured Envelope Curves}
The non-monotone structure visible in the range envelopes is not an artifact
of the aggregation procedure, but a genuine feature of the prediction
geometry.
Its origin may be understood heuristically from the residue-based formulation
introduced in \cite{Riemers2026-PCGH}.

The prediction function is ultimately governed by residue classes modulo
small primes, with the underlying density model justified by Dirichlet's
Theorem on arithmetic progressions~\cite{Apostol1976}.
In the asymptotic limit, all admissible residue classes are expected to occur
with equal frequency, yielding a smooth limiting contribution to the
Hardy--Littlewood correction factors.

To model finite-scale behaviour, a Chinese Remainder Theorem-like threshold is introduced~\cite{NivenZuckermanMontgomery1991} to separate
primes whose residue distributions may be regarded as effectively uniform from
those whose contributions have not yet stabilized.
Primes below this threshold are treated as fully equidistributed and contribute
to a smooth base term.
Primes above the threshold are grouped into an exponential tail, whose net
effect decays rapidly as the scale increases.

As the scale parameter grows, this exponential tail contracts, causing the
predicted bounds to tighten exponentially.
However, whenever a new prime crosses below the CRT-like threshold, the
structure of the residue product changes discretely.
This transition resets the tail decomposition and produces a localized
relaxation in the prediction envelope.
The resulting alternation between exponential tightening and discrete
re-normalization gives rise to the observed periodic modulation in the
prediction curves.

The measured values do not exhibit the same sharp periodic structure.
This is expected: higher-order correlations beyond the leading residue model,
together with the blockwise aggregation used to construct the range envelopes,
act to smooth the empirical curves.
Consequently, the prediction envelopes display clear modular interference
patterns, while the measured ranges converge more uniformly toward the
asymptotic limit.

\preto\section{\FloatBarrier}
\section{Pointwise Pair Count Analysis}

Pointwise analysis for the purpose of certifying the conjecture across all validation checks follows.

For each admissible \(n\), we evaluate 81 distinct windowed pair counts,
corresponding to the discrete scale parameters
\begin{equation}
\alpha_i := 2^{-i/8}, \qquad i = 0,1,\dots,80.
\end{equation}
this set of scale parameters is set as
\begin{equation}
\label{eq:A-def}
\mathcal{A} := \{\, \alpha_i : i = 0,1,\dots,80 \,\}.
\end{equation}
For each \(\alpha\in\mathcal{A}\), both the upper and lower bounds are tested
independently.

Within each primorial bin, the measured value that approaches its bound most
closely is identified.
In all cases, no crossings are observed.
Rather than exhibiting monotonic convergence, the extremal deviations display
bounded local fluctuations while remaining confined within the predicted
envelope and trending toward the asymptotic \(\lambda\) values.
This behaviour supports the interpretation of the bounds as genuine geometric
constraints rather than statistical envelopes.

Due to storage limitations, predicted bounds and measured counts must be generated jointly in this mode.
The combined computation requires approximately 60 days on an Apple M2 processor, with up to 12 concurrent processes running continuously.

\preto\subsection{\FloatBarrier}
\subsection{Definition of the Empirical Range Statistic}

Define the logarithmic deviations
\begin{align}
& \lambda_{\min}^{j,p\#}(\alpha)
:= \min_{n \in B_{j,p\#}}
\log\!\left(
\frac{\Gmeas(2n;n\alpha)}{\GHL(2n;\sqrt{2n\alpha})-\RHL(2n;\sqrt{2n\alpha})}
\right), \\
& \lambda_{\max}^{j,p\#}(\alpha)
:= \max_{n \in B_{j,p\#}}
\log\!\left(
\frac{\Gmeas(2n;n\alpha)}{\GHL(2n;\sqrt{2n\alpha})+\RHL(2n;\sqrt{2n\alpha})}
\right).
\end{align}

To visualize finite-scale variability while suppressing bin-to-bin noise,
all primorial bins \(B_{j,p\#}\) in increasing order of their
left endpoints and denote the resulting ordered sequence by \(\{B_k\}_{k\ge0}\)
are enumerated.
This ordering may cross primorial boundaries.
For each \(k\), the quantities
\(\lambda_{\min}^{k}(\alpha)\) and
\(\lambda_{\max}^{k}(\alpha)\) are defined to be the corresponding
values associated with the bin \(B_k\).

Then consecutive bins into blocks are aggregated in fixed sized groups.
For each block index \(\ell \ge 0\), intervals are defined as
\begin{equation}
\mathcal{K}_\ell := \{\, k : 12\ell \le k < 12(\ell+1) \,\}.
\end{equation}
Then, the range extrema over each block are defined as
\begin{align}
\lambda_{\min}^{\tlow,\ell}(\alpha)
&:= \min_{k \in \mathcal{K}_\ell} \lambda_{\min}^{k}(\alpha),
&
\lambda_{\min}^{\thigh,\ell}(\alpha)
&:= \max_{k \in \mathcal{K}_\ell} \lambda_{\min}^{k}(\alpha), \\
\lambda_{\max}^{\tlow,\ell}(\alpha)
&:= \min_{k \in \mathcal{K}_\ell} \lambda_{\max}^{k}(\alpha),
&
\lambda_{\max}^{\thigh,\ell}(\alpha)
&:= \max_{k \in \mathcal{K}_\ell} \lambda_{\max}^{k}(\alpha).
\end{align}

The corresponding \(n\)-values plotted on the horizontal axis are taken to be
the values \(n_{\min}^k\) and \(n_{\max}^k\) at which these extremal deviations
occur within each block.
Thus, each range block summarizes the spread of the pointwise deviations
\(\lambda_{\min}^{k}\) and \(\lambda_{\max}^{k}\)
over a fixed collection of consecutive primorial bins, rather than
representing an interval at a single value of \(n\).

\preto\subsection{\FloatBarrier}
\subsection{Pointwise Minimum Bounds}

The pointwise minimum deviations
\(\lambda_{\min}^{k}(\alpha)\), aggregated into range summaries as defined
in the previous subsection are examined.
In contrast to the geometric bounds, these quantities reflect local
(pointwise) behaviour prior to any block-level aggregation.

Figure~\ref{fig:pointwise-min-envelope} plots representative range envelopes
for \(\lambda_{\min}^{k}(\alpha)\).
As predicted by PCGC--Goldbach, these values remain non-negative and exhibit
an overall downward trend.
Notably, the fluctuations are less pronounced than in the geometric case.
This reflects the fact that the geometric aggregation procedure inflates the
effective bounds and, with them, the apparent variability.

The plotting conventions, admissibility criteria, and handling of unplottable
points are identical to those used previously.
Figure~\ref{fig:pointwise-cert-min} displays all 81 \(\alpha\)-values.
As before, the smallest values of \(\alpha\) are dominated by finite-scale
effects, while the remaining curves show clear decay.
Table~\ref{tab:pointwise-audit-min} lists representative values for
twelve selected \(\alpha\)-levels.

Including unplottable data, no violations of the inequality
\(\RHL \ge \lvert \varepsilon \rvert \) are observed across the entire computed range.

\pgfplotstableread[col sep=comma, trim cells]{boundratiomin-summary-23PR.5-v0.2.0.csv}\banddata

\begin{figure}[ht]
\centering
\begin{tikzpicture}
\begin{axis}[
  width=\linewidth,
  height=0.63\linewidth,
  xlabel={\(n\)},
  ylabel={\(\lambda_{\min}\)},
  xmin=100,
  xmax=1.2e8,
  ymax=1.1,
  ymin=1e-3,
  xmode=log,
  ymode=log,
  grid=both,
  legend cell align=left,
  legend style={
    at={(0.02,0.02)},
    anchor=south west,
    draw=none,
    fill=white,
    fill opacity=0.85,
    text opacity=1,
    font=\small,
    /tikz/align=left,
  }
]

\PlotBand{1.0}{bandA}{1}
%\PlotBand{0.5}{bandB}{1/2}
\PlotBand{0.25}{bandC}{1/4}
%\PlotBand{0.125}{bandD}{1/8}
\PlotBand{0.0625}{bandE}{1/16}
%\PlotBand{0.03125}{bandF}{1/32}
\PlotBand{0.015625}{bandG}{1/64}
%\PlotBand{0.0078125}{bandH}{1/128}
\PlotBand{0.00390625}{bandI}{1/256}
%\PlotBand{0.001953125}{bandJ}{1/512}
\PlotBand{0.0009765625}{bandK}{1/1024}

\end{axis}
\end{tikzpicture}
\caption{Smooth envelope bands for \( \lambda_{\min} \) by \( \alpha \).}
\label{fig:pointwise-min-envelope}
\end{figure}
\pgfplotstableread[col sep=comma]{bounds-cert-23PR.5-v0.2.0.csv}\eventdata

\begin{figure}[ht]
\centering
\begin{tikzpicture}
\begin{axis}[
  width=\linewidth,
  height=0.30\linewidth,
  xlabel={\(\alpha\)},
  ylabel={\(\lambda_{\min}\)},
  xmode=log,
  ymode=log,
  xmin=0.0009765625,
  xmax=1.0,
  ymin=1e-3,
  ymax=0.5,
  grid=both,
  legend style={
    at={(0.02,0.02)},
    anchor=south west,
    draw=none,
    fill=none,
    font=\small,
    cells={anchor=west},
  },
  legend cell align=left,
]

% ---- styles ----
\pgfplotsset{
  finalLine/.style={thick, solid, opacity=0.75, mark=none},
  peleven/.style={only marks, mark=triangle*, mark size=1.7pt},
  pthirteen/.style={only marks, mark=square*,   mark size=1.6pt},
}

% Drop empty y-cells (treat as NaN so pgfplots discards the coordinate)
\pgfplotsset{
  y drop empty/.style={
    y filter/.code={
      \edef\temp{\pgfmathresult}%
      \ifx\temp\empty
        \def\pgfmathresult{nan}%
      \fi
    }
  }
}

% ---- p = 11 : markers only ----
\addplot[peleven, y drop empty]
  table[x=alpha, y=L11_lo] {\eventdata};
\addlegendentry{\(p = 11\)}

% ---- p = 13 : markers only ----
\addplot[pthirteen, y drop empty]
  table[x=alpha, y=L13_lo] {\eventdata};
\addlegendentry{\(p = 13\)}

% ---- final : line only ----
\addplot[finalLine, y drop empty]
  table[x=alpha, y=Lfinal_lo] {\eventdata};
\addlegendentry{final}
\end{axis}
\end{tikzpicture}
\caption{All \( \alpha \) values showing event and final envelope \( \lambda_{\min} \) lower statistics.}
\label{fig:pointwise-cert-min}
\end{figure}

\pgfplotstableread[col sep=comma]{bounds-audit-23PR.5-v0.2.0.csv}\auditdata

\begin{table}[ht]
\centering
\caption{Selected \( \alpha \) values showing event and final envelope \( \lambda_{\min} \) statistics.}
\label{tab:pointwise-audit-min}
\pgfplotstabletypeset[
  empty cells with={---},
  columns={alpha,L11_lo,L13_lo,Lfinal_lo,Lfinal_lo_std},
  columns/alpha/.style={column name={$\alpha$}, fixed, precision=12,column type=l},
  every head row/.style={before row=\toprule, after row=\midrule},
  every last row/.style={after row=\bottomrule},
  columns/L11_lo/.style={column name={$p{=}11$}, sci, sci zerofill, precision=2},
  columns/L13_lo/.style={column name={$p{=}13$}, sci, sci zerofill, precision=2},
  columns/Lfinal_lo/.style={column name={final}, sci, sci zerofill, precision=2,column type={r@{$\,\pm\,$}}},
  columns/Lfinal_lo_std/.style={column name={std.dev.}, sci, sci zerofill, precision=2, column type=l},
]{\auditdata}
\end{table}

\preto\subsection{\FloatBarrier}
\subsection{Pointwise Maximum Bounds}

Consider the complementary pointwise maximum deviations
\(\lambda_{\max}^{k}(\alpha)\).
These quantify local excursions above the Hardy--Littlewood prediction and
provide a direct pointwise analogue of the maximum geometric bounds.

Figure~\ref{fig:pointwise-max-envelope} shows representative range envelopes
for \(\lambda_{\max}^{k}(\alpha)\).
As in the minimum case, PCGC--Goldbach predicts asymptotic convergence toward
zero without imposing detailed constraints on the intermediate shape.
Here again, the observed fluctuations are smaller than those seen in the
geometric bounds, consistent with the absence of aggregation-induced
inflation.

The full collection of pointwise maximum bounds for all 81 values of
\(\alpha\) are shown in Figure~\ref{fig:pointwise-cert-max}.
Finite-scale effects dominate only at the smallest values of \(\alpha\);
for the remaining curves a clear decreasing trend is visible.
Representative values are summarized in
Table~\ref{tab:pointwise-audit-max}.

Including unplottable data, no violations of the inequality
\(\RHL \ge \lvert \varepsilon \rvert \) are observed across the entire computed range.

\pgfplotstableread[col sep=comma, trim cells]{boundratiomax-summary-23PR.5-v0.2.0.csv}\banddata

\begin{figure}[ht]
\centering
\begin{tikzpicture}
\begin{axis}[
  width=\linewidth,
  height=0.63\linewidth,
  xlabel={\(n\)},
  ylabel={\(\lambda_{\max}\)},
  xmin=7,
  xmax=1.2e8,
  ymax=3,
  ymin=1e-3,
  xmode=log,
  ymode=log,
  grid=both,
  legend cell align=left,
  legend style={
    at={(0.02,0.02)},
    anchor=south west,
    draw=none,
    fill=white,
    fill opacity=0.85,
    text opacity=1,
    font=\small,
    /tikz/align=left,
  }
]

\PlotBand{1.0}{bandA}{1}
%\PlotBand{0.5}{bandB}{1/2}
\PlotBand{0.25}{bandC}{1/4}
%\PlotBand{0.125}{bandD}{1/8}
\PlotBand{0.0625}{bandE}{1/16}
%\PlotBand{0.03125}{bandF}{1/32}
\PlotBand{0.015625}{bandG}{1/64}
%\PlotBand{0.0078125}{bandH}{1/128}
\PlotBand{0.00390625}{bandI}{1/256}
%\PlotBand{0.001953125}{bandJ}{1/512}
\PlotBand{0.0009765625}{bandK}{1/1024}

\end{axis}
\end{tikzpicture}
\caption{Smooth envelope bands for \( \lambda_{\max} \) by \( \alpha \).}
\label{fig:pointwise-max-envelope}
\end{figure}

\begin{figure}[ht]
\centering
\begin{tikzpicture}
\begin{axis}[
  width=\linewidth,
  height=0.30\linewidth,
  xlabel={\(\alpha\)},
  ylabel={\(-\lambda_{\max}\)},
  xmode=log,
  ymode=log,
  xmin=0.0009765625,
  xmax=1.0,
  ymin=1e-3,
  ymax=0.5,
  grid=both,
  unbounded coords=discard,
  filter discard warning=false,
  legend style={
    at={(0.02,0.02)},
    anchor=south west,
    draw=none,
    fill=none,
    font=\small,
    cells={anchor=west},
  },
  legend cell align=left,
]

% ---------- styles ----------
\pgfplotsset{
  final/.style={thick, mark=none},
  p11/.style={only marks, mark=triangle*, mark size=1.8pt},
  p13/.style={only marks, mark=square*,   mark size=1.8pt},
}

% ---------- p = 11 : triangles only ----------
\addplot[
  p11,
  x filter/.code={%
    \edef\temp{\thisrow{L11_hi}}%
    \ifx\temp\empty
      \def\pgfmathresult{nan}%
    \fi
  },
] table[
  x=alpha,
  y expr=abs(\thisrow{L11_hi}),
]{\eventdata};
\addlegendentry{\( p=11 \)}

% ---------- p = 13 : squares only ----------
\addplot[
  p13,
  x filter/.code={%
    \edef\temp{\thisrow{L13_hi}}%
    \ifx\temp\empty
      \def\pgfmathresult{nan}%
    \fi
  },
] table[
  x=alpha,
  y expr=abs(\thisrow{L13_hi}),
]{\eventdata};
\addlegendentry{\( p=13 \)}

% ---------- final : solid line only ----------
\addplot[
  final,
  x filter/.code={%
    \edef\temp{\thisrow{Lfinal_hi}}%
    \ifx\temp\empty
      \def\pgfmathresult{nan}%
    \fi
  },
] table[
  x=alpha,
  y expr=abs(\thisrow{Lfinal_hi}),
]{\eventdata};
\addlegendentry{final}

\end{axis}
\end{tikzpicture}
\caption{All  \( \alpha \) values showing event and final envelope \( \lambda_{\max} \) statistics.}
\label{fig:pointwise-cert-max}
\end{figure}

\begin{table}[ht]
\centering
\caption{Selected \( \alpha \) values showing event and final envelope \(\lambda_{\max}\) upper statistics.}
\label{tab:pointwise-audit-max}
\pgfplotstabletypeset[
  empty cells with={---},
  columns={alpha,L11_hi,L13_hi,Lfinal_hi,Lfinal_hi_std},
  columns/alpha/.style={column name={$\alpha$}, fixed, precision=12,column type=l},
  every head row/.style={before row=\toprule, after row=\midrule},
  every last row/.style={after row=\bottomrule},
  columns/L11_hi/.style={column name={$p{=}11$}, sci, sci zerofill, precision=2},
  columns/L13_hi/.style={column name={$p{=}13$}, sci, sci zerofill, precision=2},
  columns/Lfinal_hi/.style={column name={final}, sci, sci zerofill, precision=2,column type={r@{$\,\pm\,$}}},
  columns/Lfinal_hi_std/.style={column name={std.dev.}, sci, sci zerofill, precision=2, column type=l},
]{\auditdata}
\end{table}

\preto\section{\FloatBarrier}
\section{Certification}
\label{sec:certification}

\begin{theorem}[Empirical Certification of PCGC--Goldbach Bounds Up To \(23\#/2\)]
\label{thm:cert-23primorial}

Let \(\mathcal{A}\) denote the discrete set of window parameters used in the certification, as defined in equation~\eqref{eq:A-def}.
For each \(\alpha\in\mathcal{A}\) and each integer \(2n\ge 4\), let
\(\Gmeas(2n;n\alpha)\) denote the measured Goldbach pair count in the window
of half-width \(n\alpha\), and let \(\GHL(2n;n\alpha)\) be the corresponding
Hardy--Littlewood geometric main term.
As in \cite{Riemers2026-PCGH}, write
\begin{equation}
\Gmeas(2n;n\alpha)
=
\GHL(2n;n\alpha) + \varepsilon(2n;L),
\end{equation}
where \(\varepsilon(2n;L)\) is the exact remainder term and \(L\) denotes the
associated effective modulus cutoff.

Fix an admissible effective modulus cutoff \(L=\EffLocMod(2n;\alpha)\).
Let \(\RHL(2n;L)\) denote the exact geometric remainder envelope, and let
\(\RHLbound(2n;L)\) denote the associated explicit bounding envelope,
as in Definitions~\ref{def:RHL} and~\ref{def:RHLbound}, respectively.

\smallskip
\noindent
\textbf{(A) Certifiability of the envelope.}
For all admissible \((n,L)\), the envelope \(\RHLbound(2n;L)\) is nonnegative,
conservative, and asymptotically sharp in the sense that
\begin{equation}
\label{eq:certifiable-envelope}
|\varepsilon(2n;L)| \le \RHL(2n;L) \le \RHLbound(2n;L),
\end{equation}
and moreover \(\RHLbound(2n;L)=\RHL(2n;L)\bigl(1+o_L(1)\bigr)\) uniformly over the
admissible parameter ranges.

\smallskip
\noindent
\textbf{(B) Empirical certification on the computed range.}
For every \(\alpha\in\mathcal{A}\) and every integer \(n\) with
\begin{equation}
4 \le 2n \le 23\#,
\end{equation}
the measured counts satisfy the certified two-sided geometric bounds

\begin{equation}
\label{eq:certified-two-sided-RHL}
\GHL(2n; n\alpha) - \RHL(2n;\sqrt{2n\alpha})
\;\le\;
\Gmeas(2n; n\alpha)
\;\le\;
\GHL(2n; n\alpha) + \RHL(2n;\sqrt{2n\alpha}).
\end{equation}

Since \(\RHLbound(2n;L)\ge \RHL(2n;L)\) for all admissible \(L\),
by the \emph{Overall Bounding Envelope Lemma} of \cite{Riemers2026-PCGH},
which establishes uniform dominance of \(\RHLbound\) over \(\RHL\),
the same certified inequalities hold \emph{a fortiori} with
\(\RHLbound\) in place of \(\RHL\).

On the full computed range, certification is expressed by the two-sided
inequality~\eqref{eq:certified-two-sided-RHL}, which is checked directly for
every \((n,\alpha)\) in the grid.
For visualization, logarithmic ratio diagnostics on the
\emph{plottable domain} where the relevant denominators are positive (so that
division and \(\log\) are defined) are plotted as well.
On this domain, the lower inequality implies
\begin{equation}
\lambda_{\min}(n;\alpha)
:=
\log\!\left(
\frac{\Gmeas(2n;n\alpha)}{\GHL(2n;n\alpha)-\RHL(2n;L)}
\right)\ge 0,
\end{equation}
while the upper inequality implies, equivalently,
\begin{equation}
\lambda_{\max}(n;\alpha)
:=
\log\!\left(
\frac{\Gmeas(2n;n\alpha)}{\GHL(2n;n\alpha)+\RHL(2n;L)}
\right)\le 0.
\end{equation}
Points excluded from plots only to avoid division by \(0\) or \(\log(0)\) are
treated as unplottable; these cases are still included in the direct
verification of~\eqref{eq:certified-two-sided-RHL}.

\end{theorem}

\begin{proof}
Part~(A) follows directly from the envelope framework developed in
\cite{Riemers2026-PCGH}.
In particular, the \emph{Overall Bounding Envelope Lemma} of \cite{Riemers2025-Framework}
establishes that the exact remainder term satisfies
\begin{equation}
|\varepsilon(2n;L)| \le \RHL(2n;L),
\end{equation}
and that the explicit envelope \(\RHLbound(2n;L)\) uniformly dominates
\(\RHL(2n;L)\) for all admissible \(L\).
The non-asymptotic-proxy nature of \(\RHLbound\) is likewise a consequence
of this lemma.

For Part~(B), we certify the inequalities~\eqref{eq:certified-two-sided-RHL} by
direct computation over the full grid
\(\{(\alpha,n): \alpha\in\mathcal{A},\; 4\le n\le 23\#/2\}\).
For each pair \((\alpha,n)\), the computational pipeline evaluates
\(\Gmeas(2n;n\alpha)\), \(\GHL(2n;n\alpha)\), and
\(\RHLbound(2n;\sqrt{2n\alpha})\), and explicitly checks the two inequalities
in~\eqref{eq:certified-two-sided-RHL}.
All checks pass with no exceptions on the computed range.
The stated nonnegativity of the plotted logarithmic deviations is equivalent to
the same inequalities wherever the denominators are positive; points excluded
from plots for numerical reasons were nevertheless checked against
\eqref{eq:certified-two-sided-RHL} and likewise show no violations.

The certified data products, executable pipeline, and verification logs used in
this computation are archived and made available via the accompanying software
package~\cite{Riemers2025-SieveGoldbach}, enabling independent reproduction and
audit of all certified claims.
\end{proof}

\preto\section{\FloatBarrier}
\section{Conclusion}

This work establishes an empirical certification of geometric confinement for
Goldbach pair counts over the range \(4 \le 2n \le 23\#\), evaluated across
the discrete family of fixed-scale windows
\(\mathcal{A}=\{2^{-i/8}: i=0,1,\dots,80\}\).
Within this domain, no violations of the certified two-sided inequalities occur,
either in plotted data or in the full set of values checked but excluded from
visualization for numerical reasons.
In this sense, the tested range admits no empirical escape route: every measured
value lies within the geometric envelopes predicted by the Prime Curvature
Geometry Conjecture---Goldbach case.

A central distinction of this work is the separation between probabilistic
heuristics and geometric confinement.
While classical Hardy--Littlewood heuristics describe average behaviour,
the present certification concerns pointwise and blockwise control by explicit
geometric envelopes.
The absence of violations is therefore not an assertion of likelihood or
statistical typicality, but a falsifiable geometric statement verified over a
finite, explicitly enumerated domain.

The certification framework employed here is deliberately reproducible.
All claims are grounded in direct computation over a fully specified grid of
parameters, with archived data products, executable pipelines, and verification
logs enabling independent audit.
In this respect, certification is treated not as an informal observational
claim, but as a concrete experimental standard: any counterexample within the
tested regime would be detectable, reproducible, and decisive.

It is important to delineate the scope of this result.
The present analysis is restricted to fixed-scale Goldbach windows indexed by
constant \(\alpha\), organized using primorial-based binning.
No claims are made regarding short-interval regimes in the offset variable \(m\),
including windows of fixed or slowly growing size.
Such regimes are geometrically distinct and would require a different
experimental design, coverage strategy, and certification criterion than those
used here.

Although one could, in principle, project certified points from the fixed-scale
\(\alpha_i\) lines onto shorter interval regimes, such projections are inherently
loose and do not constitute meaningful short interval certification.
At best, they provide consistency checks rather than independent validation.
A serious short-interval analysis would instead require tracing a family of
scale functions \(\alpha(n)\) across the full data range, with bounds evaluated
pointwise along these dynamically varying trajectories.
That problem lies outside the scope of the present in this paper.

Within its stated domain, however, the conclusion is unambiguous:
across all tested scales and throughout the full certified range,
the Goldbach pair counts exhibit no empirical deviations from the geometric
constraints predicted by the Prime Curvature Geometry Conjecture.

\preto\subsection{\FloatBarrier}
\subsection{Outlook and Open Directions}

The present work establishes certified geometric bounds for primorial-based
window families over a fixed-scale parameter grid.
Several natural extensions suggest themselves, particularly in regimes that
are not amenable to the fixed-scale framework employed here.

One such direction is the development of a dedicated certification program for
short-interval behaviour in the offset variable \(m\).
While projections from the fixed-scale \(\alpha_i\) lines are possible, such
projections are inherently loose and do not support meaningful certification.
A genuine short-interval analysis would instead require tracing dynamically
varying scale functions \(\alpha(n)\) across the full data range, with bounds
certified pointwise along these trajectories.

This regime constitutes a geometrically distinct problem and would necessitate
different experimental design principles and coverage strategies than those
used in the present study.
This is a natural and well-defined direction for future investigation.

\preto\section{\FloatBarrier}
\section{Reproducibility}

All empirical results and certifications reported in this paper are derived
from deterministic computations over explicitly enumerated parameter grids.
The full computational pipeline---including executable code, configuration files,
and verification scripts---has been archived to enable independent reproduction
and audit of all certified claims.

The certification procedure does not rely on stochastic sampling, heuristic
filtering, or post hoc selection.
Every admissible pair \((\alpha,n)\) in the certified domain is evaluated
explicitly, and all inequalities are checked directly.
Any counterexample within the tested range would therefore be detectable and
reproducible using the archived materials.

The software package associated with this work also records version
information, build parameters, and platform details sufficient to reproduce
the numerical results across standard computing environments.\cite{Riemers2025-SieveGoldbach}

\preto\subsection{\FloatBarrier}
\subsection{Data Availability}

The certified data products generated for this study, including all evaluated
Goldbach pair counts, geometric envelope values, and verification logs, are
archived alongside the computational pipeline.
Data are provided in machine-readable formats with documented field
specifications and fixed ordering conventions.

Checksum files are included to verify data integrity, and all datasets are
labelled by the corresponding parameter values \((\alpha,n)\) to permit
independent validation of individual certified inequalities.
The complete archive is available via the software repository 
\cite{Riemers2025-SieveGoldbach} , and contains all information necessary to
reproduce the figures, tables, and certification results reported in this paper.

\clearpage
\bibliographystyle{plain}
\bibliography{pcgc-goldbach-empirical}

\end{document}

