\documentclass[11pt]{article}
\usepackage{amsmath, amssymb, amsthm}
\usepackage{fullpage}
\usepackage{hyperref}
\usepackage{mathtools}
\usepackage{booktabs}
\usepackage{url}

\title{\textbf{Reductions to Prime Curvature Geometry: Conditional Theorems for Goldbach, Hardy--Littlewood A, and Short--Interval Problems}}
\author{Bill C. Riemers}
\date{\today}

\newtheoremstyle{inline}% for remarks/notes
  {}{}{\normalfont}{}{\itshape}{.}{ }{}
  
\newtheoremstyle{break}  % Name
  {1ex}                  % Space above
  {1ex}                  % Space below
  {\normalfont}          % Body font
  {}                     % Indent
  {\bfseries}            % Theorem head font
  {.}                    % Punctuation after theorem head
  {\newline}             % Space after theorem head (THIS FORCES LINE BREAK)
  {}                     % Theorem head spec

\theoremstyle{inline}
\newtheorem*{remark}{Remark}

\theoremstyle{inline}
\newtheorem*{convention}{Convention}

\theoremstyle{break}
\newtheorem{lemma}{Lemma}

\makeatletter
\renewenvironment{proof}[1][\proofname]{%
  \par\pushQED{\qed}%
  \normalfont \topsep6\p@\@plus6\p@\relax
  \trivlist
  \item[\hskip\labelsep
        \itshape
    #1\@addpunct{.}]\mbox{}\\  % This line forces the break
}{%
  \popQED\endtrivlist\@endpefalse
}
\makeatother

\theoremstyle{break}
\newtheorem*{conclusion}{Conclusion}

\theoremstyle{break}
\newtheorem{theorem}{Theorem}

\theoremstyle{break}
\newtheorem{proposition}{Proposition}

\theoremstyle{break}
\newtheorem{conjecture}{Conjecture}

\theoremstyle{break}
\newtheorem{corollary}{Corollary}

\theoremstyle{break}
\newtheorem{definition}{Definition}

\theoremstyle{break}
\newtheorem{hypothesis}{Hypothesis}

\theoremstyle{inline}
\newtheorem*{note}{Note}

% constants
\newcommand{\xTwinPrimeConstant}{0.66016181584686957392\dots}

\newcommand{\xEffLocModDivA}{2}
\newcommand{\xEffLocModDivB}{8}
\newcommand{\xEffLocModDivC}{48}
\newcommand{\xEffLocModDivD}{480}
\newcommand{\xEffLocModDivE}{5760}
\newcommand{\xEffLocModDivF}{92160}
\newcommand{\xEffLocModDivG}{1658880}
\newcommand{\xEffLocModDivH}{36495360}
\newcommand{\xEffLocModDivI}{1021870080}
\newcommand{\xEffLocModNDivA}{1}
\newcommand{\xEffLocModNDivB}{4}
\newcommand{\xEffLocModNDivC}{24}
\newcommand{\xEffLocModNDivD}{240}
\newcommand{\xEffLocModNDivE}{2880}
\newcommand{\xEffLocModNDivF}{46080}
\newcommand{\xEffLocModNDivG}{829440}
\newcommand{\xEffLocModNDivH}{18247680}
\newcommand{\xEffLocModNDivI}{510935040}
\newcommand{\xEffLocModNDivJ}{15328051200}
\newcommand{\xEffLocModNDivK}{551809843200}
\newcommand{\xEffLocModNDivL}{22072393728000}
\newcommand{\xEffLocModDivJ}{30656102400}
\newcommand{\xEffLocModDivK}{1103619686400}
\newcommand{\xEffLocModDivL}{44144787456000}
\newcommand{\xOmegaPrimeNormASqrt}{1.19229679166961633687\dots}
\newcommand{\xOmegaPrimeNormBSqrt}{1.03930787611075825952\dots}
\newcommand{\xOmegaPrimeNormCSqrt}{1.00503767800313569410\dots}
\newcommand{\xOmegaPrimeNormDSqrt}{1.00044758029006635718\dots}
\newcommand{\xOmegaPrimeNormESqrt}{1.00003117937931407649\dots}
\newcommand{\xOmegaPrimeNormFSqrt}{1.00000179466958537388\dots}
\newcommand{\xOmegaPrimeNormGSqrt}{1.00000008676062914913\dots}
\newcommand{\xOmegaPrimeNormHSqrt}{1.00000000333844685314\dots}
\newcommand{\xOmegaPrimeNormISqrt}{1.00000000011314753555\dots}
\newcommand{\xOmegaPrimeNormA}{1.42157163942566050077\dots}
\newcommand{\xOmegaPrimeNormB}{1.08016086134585523892\dots}
\newcommand{\xOmegaPrimeNormC}{1.01010073420593466543\dots}
\newcommand{\xOmegaPrimeNormD}{1.00089536090824877026\dots}
\newcommand{\xOmegaPrimeNormE}{1.00006235973078184739\dots}
\newcommand{\xOmegaPrimeNormF}{1.00000358934239158669\dots}
\newcommand{\xOmegaPrimeNormG}{1.00000017352126582567\dots}
\newcommand{\xOmegaPrimeNormH}{1.00000000667689371743\dots}
\newcommand{\xOmegaPrimeNormI}{1.00000000022629507111\dots}
\newcommand{\xOmegaPrimeNormJ}{1.00000000000661314375\dots}
\newcommand{\xOmegaPrimeNormK}{1.00000000000017007626\dots}
\newcommand{\xOmegaPrimeNormL}{1.00000000000000409690\dots}
\newcommand{\xOmegaPrimeNormJSqrt}{1.00000000000330657187\dots}
\newcommand{\xOmegaPrimeNormKSqrt}{1.00000000000008503813\dots}
\newcommand{\xOmegaPrimeNormLSqrt}{1.00000000000000204845\dots}
\newcommand{\xOmegaPrime}{\xOmegaPrimeNormA}
\newcommand{\xOmegaPrimeSqrt}{\xOmegaPrimeNormASqrt}
\newcommand{\xCvalueA}{2.84314327885132100154\dots}
\newcommand{\xCvalueB}{2.72259939396405723999\dots}
\newcommand{\xCvalueC}{2.54555496154236393982\dots}
\newcommand{\xCvalueD}{2.47920428379799626238\dots}
\newcommand{\xCvalueE}{2.39345422669887085237\dots}
\newcommand{\xCvalueF}{2.35972191892736156441\dots}
\newcommand{\xCvalueG}{2.30959606775400312716\dots}
\newcommand{\xCvalueH}{2.25914178520133225614\dots}
\newcommand{\xCvalueI}{2.24514310212975360197\dots}
\newcommand{\xCvalueJ}{2.21336456753910061273\dots}
\newcommand{\xCvalueK}{2.19678940939142586685\dots}
\newcommand{\xCvalueL}{2.18928655289998458252\dots}

\newcommand{\xCratioDivA}{\xOmegaPrimeNormA}
\newcommand{\xCratioDivB}{1.02097477273652146499\dots}
\newcommand{\xCratioDivC}{0.79548592548198873119\dots}
\newcommand{\xCratioDivD}{0.69727620481818644879\dots}
\newcommand{\xCratioDivE}{0.61706241782080264162\dots}
\newcommand{\xCratioDivF}{0.57034294427199412811\dots}
\newcommand{\xCratioDivG}{0.52721492269286936618\dots}
\newcommand{\xCratioDivH}{0.49225684650688208950\dots}
\newcommand{\xCratioDivI}{0.47173493219492656096\dots}
\newcommand{\xCratioNDivA}{2.84314327885132100154\dots}
\newcommand{\xCratioNDivB}{2.04194954547304292999\dots}
\newcommand{\xCratioNDivC}{1.59097185096397746238\dots}
\newcommand{\xCratioNDivD}{1.39455240963637289758\dots}
\newcommand{\xCratioNDivE}{1.23412483564160528325\dots}
\newcommand{\xCratioNDivF}{1.14068588854398825623\dots}
\newcommand{\xCratioNDivG}{1.05442984538573873237\dots}
\newcommand{\xCratioNDivH}{0.98451369301376417900\dots}
\newcommand{\xCratioNDivI}{0.94346986438985312192\dots}
\newcommand{\xCratioNDivJ}{0.89911180903665064257\dots}
\newcommand{\xCratioNDivK}{0.86759036084859588621\dots}
\newcommand{\xCratioNDivL}{0.84301153526760871403\dots}
\newcommand{\xCratioDivJ}{0.44955590451832532128\dots}
\newcommand{\xCratioDivK}{0.43379518042429794310\dots}
\newcommand{\xCratioDivL}{0.42150576763380435701\dots}
\newcommand{\xGoldbachN}{10805}

% --- Base (unchanged) ---
\newcommand{\Gmeas}{G}
\newcommand{\Gpred}{\mathring{G}}
\newcommand{\Gproxy}{\widehat{G}}

% tiny tags
\newcommand{\talign}{{\scriptscriptstyle\mathrm{align}}}
\newcommand{\thead}{{\scriptscriptstyle\mathrm{head}}}
\newcommand{\ttail}{{\scriptscriptstyle\mathrm{tail}}}
\newcommand{\tavg}{{\scriptscriptstyle\mathrm{avg}}}
\newcommand{\tbound}{{\scriptscriptstyle\mathrm{bound}}}
\newcommand{\tdensity}{{\scriptscriptstyle\mathrm{density}}}
\newcommand{\tenv}{{\scriptscriptstyle\mathrm{env}}}
\newcommand{\tpairs}{{\scriptscriptstyle\mathrm{pairs}}}
\newcommand{\ttrivial}{{\scriptscriptstyle\mathrm{trivial}}}
\newcommand{\tsem}{{\scriptscriptstyle\mathrm{sem}}}
\newcommand{\tsieve}{{\scriptscriptstyle\mathrm{sieve}}}
\newcommand{\twin}{{\scriptscriptstyle\mathrm{win}}}
\newcommand{\tref}{{\scriptscriptstyle\mathrm{ref}}}
\newcommand{\tana}{{\scriptscriptstyle\mathrm{analytical}}}
\newcommand{\tgb}{{\scriptscriptstyle\mathrm{GB}}}
\newcommand{\thl}{{\scriptscriptstyle\mathrm{HL}}}
\newcommand{\teff}{{\scriptscriptstyle\mathrm{eff}}}
\DeclareMathOperator{\SquareCap}{\mathsf{Sq}}
\DeclareMathOperator{\EffLocModCap}{\mathsf{\EffLocMod}}

\newcommand{\etaHL}{\eta^{\mathrm{HL}}}
\newcommand{\Ecap}{\mathrm{Ecap}}
\newcommand{\HLCorr}{\mathcal{H}}
\newcommand{\pmin}{\mathrm{p_{\min}}}
\newcommand{\ecap}[1]{\left\langle #1 \right\rangle_{\mathrm{ec}}}
\newcommand{\GHL}{\Gpred^{\thl}}
\newcommand{\GHLproxy}{\Gproxy^{\thl}}
\newcommand{\OmegaPrime}{\Omega_{\mathrm{prime}}}
\newcommand{\OmegaPrimeNorm}{\widehat{\Omega}_{\mathrm{prime}}}
\newcommand{\Ipar}{I^{\mathrm{par}}}
\newcommand{\EffLocMod}{\mathcal{Q}}
\newcommand{\Peff}{\mathbb{P}_{\teff}}
\newcommand{\Podd}{{\mathbb{P}\setminus\{2\}}}
\newcommand{\RHLbound}{\widehat{R}^{\thl}}
\newcommand{\RHLboundEnv}{\RHLbound_{\tenv}}
\newcommand{\Ssem}{\mathfrak{S}}
\newcommand{\SsemHead}{\Ssem_{\thead}}
\newcommand{\SsemTail}{\Ssem_{\ttail}}
\newcommand{\SGB}{\mathfrak{S_{\tgb}}}
\newcommand{\HHL}{H^{\thl}}
\newcommand{\THL}{T^{\thl}}
\newcommand{\RHL}{R^{\thl}}
\newcommand{\QpleL}{\EffLocMod_p(n) \le L}
\newcommand{\QpProduct}{\prod_{\substack{p\in\Peff(n) \\ \QpleL}}}
\newcommand{\pZeroProduct}{\prod_{\substack{p\in\Peff(n) \\ p \le P_0}}}

\begin{document}

\maketitle

\begin{abstract}
The \textbf{Prime Curvature Geometry Conjecture for Goldbach (PCGC--Goldbach)}
introduces a geometric framework for prime pair counting in which remainder
terms are controlled by an explicit bounding envelope with intrinsic exponential
curvature.
This paper develops and analyzes the \emph{conditional consequences} of this
framework, assuming either PCGC--Goldbach or its weaker variant,
\textbf{PCGC--Goldbach Bounds}.

Under this assumption, a collection of reductions is established showing that
the geometric framework provides sufficient quantitative control to imply several
classical results in analytic number theory.
First, PCGC--Goldbach Bounds implies Goldbach's conjecture for all even integers
\(2n \ge 4\), with explicit verification required only up to a computable finite
threshold.
Second, the classical Hardy--Littlewood asymptotic formula for Goldbach pair
counts (Conjecture~A, Goldbach form) emerges as an asymptotic consequence of the
geometric bounds.
Third, the framework yields relative agreement between measured and predicted
Goldbach counts in the short--interval scaling regime
\(M \ge (2n)^{\frac{1}{2}+\varepsilon}\), corresponding to the window sizes
associated with Bombieri--Vinogradov--type phenomena.

These results show that a single geometric hypothesis suffices to organize and
connect several problems that have traditionally required distinct analytic
techniques.
All statements in this paper are explicitly conditional.
Rather than claiming unconditional progress on Goldbach's conjecture, the paper
demonstrates that PCGC--Goldbach, if validated, provides a unified geometric
foundation with explicit bounding structure for classical Goldbach asymptotics
and short--interval behavior.
\end{abstract}

\clearpage
\tableofcontents

\clearpage
\section{Introduction}

\subsection{Background and Motivation}

The \emph{Prime Curvature Geometry Conjecture for Goldbach} (PCGC--Goldbach) was introduced
in a previous paper in this series \cite{Riemers2026-PCGH} as a geometric framework for organizing remainder
terms in Goldbach pair counting.
Rather than treating error terms as purely analytic artifacts, PCGC--Goldbach
models them through an explicit bounding envelope whose structure exhibits
intrinsic exponential curvature.
This geometric perspective yields uniform, scale--sensitive control over
deviations between measured Goldbach counts and Hardy--Littlewood--type
predictors.

Extensive computational validation of this framework, carried out up to
\(2n = 23\#\), has been presented in a supportive paper in this series \cite{Riemers2026-PCGC-Empirical}.
These computations demonstrate strong empirical agreement between observed
Goldbach counts and the predicted geometric bounds, across a wide range of
window sizes and residue configurations.

The purpose of the present paper is not to further validate PCGC--Goldbach
computationally, but rather to examine its \emph{conditional consequences}.
Specifically, this paper investigates which classical conjectures and asymptotic
predictions in analytic number theory follow if PCGC--Goldbach (or its weaker
bounding variant, PCGC--Goldbach Bounds) is assumed to hold.
In doing so, it is shown that a single geometric hypothesis subsumes several results
that have historically required distinct analytic techniques.

\subsection{Main Results}

Assuming PCGC--Goldbach or its weaker form, \emph{PCGC--Goldbach Bounds}, three principal reductions are established.

\begin{enumerate}
\item \textbf{Goldbach's Conjecture.}
It is shown that PCGC--Goldbach Bounds implies Goldbach's conjecture for all even
integers \(2n \ge 4\).
The reduction yields an explicit finite verification threshold, with direct
computation required only up to
\begin{equation}
n_0 = \xGoldbachN.
\end{equation}
Beyond this range, the geometric remainder bounds guarantee the existence of at
least one Goldbach representation for every even integer.

\item \textbf{Hardy--Littlewood Conjecture A (Goldbach form).}
The classical Hardy--Littlewood asymptotic formula for Goldbach
representations is implied as a direct consequence of the PCGC--Goldbach Bounds.
In particular, the framework yields
\begin{equation}
\Gmeas(2n) \sim
2C_2\,\Ssem(2n)\,\frac{2n}{\log^2(2n)},
\qquad (n\to\infty),
\end{equation}
showing that the geometric model is not merely compatible with
Hardy--Littlewood heuristics, but sufficient---under its stated assumptions---to
imply their standard asymptotic prediction.

\item \textbf{Short--Interval Bombieri--Vinogradov--Type Agreement.}
It is shown that PCGC--Goldbach Bounds implies relative agreement between measured and
predicted Goldbach counts in the short--interval scaling regime
\begin{equation}
M \ge (2n)^{\frac{1}{2}+\varepsilon}.
\end{equation}
This corresponds precisely to the window sizes at which
Bombieri--Vinogradov--type hypotheses are traditionally formulated.
The result demonstrates that the geometric framework provides asymptotic
control in regimes where classical sieve methods encounter fundamental
limitations.
\end{enumerate}

\subsection{Structure and Approach}

All reductions in this paper follow a common strategy.
Starting from the geometric bounding structure supplied by PCGC--Goldbach (or
PCGC--Goldbach Bounds), explicit quantitative inequalities are derived that imply
the corresponding classical results.
The central technical object is the remainder envelope
\(\RHLbound(2n;L)\), which provides uniform control over deviations across all
admissible window scales.

Section~\ref{sec:fundamental-definitions} collects the fundamental definitions,
operators, and identities required for the reductions, adapted from the
geometric framework developed in \cite{Riemers2026-PCGH}.
Section~\ref{sec:main-conjectures} formally states PCGC--Goldbach and
PCGC--Goldbach Bounds, establishing the precise hypotheses under which all
subsequent results are derived.
Section~\ref{sec:geometric-framework} presents the main reduction theorems,
showing how the geometric bounds imply Goldbach's conjecture and recover
Hardy--Littlewood asymptotics.
Section~\ref{sec:short-intervals} examines the interaction between
PCGC--Goldbach Bounds and short--interval distribution problems, establishing
relative agreement results in the Bombieri--Vinogradov scaling regime.

All results are stated conditionally.
The paper shows that, assuming the validity of PCGC--Goldbach (or the weaker PCGC--Goldbach Bounds),
a broad class of classical conjectures and asymptotic phenomena
follows naturally from a single geometric principle.

\section{Fundamental Definitions and Identities}
\label{sec:fundamental-definitions}

The definitions and identities collected in this section are adapted from
the first two papers of this series \cite{Riemers2025,Riemers2026-PCGH}.
They are presented here in a condensed form sufficient for the present work;
for full derivations, motivation, and extended discussion, the reader is
referred to the cited papers.

\subsection{Admissible Parity}

The parity restriction was introduced in \cite{Riemers2025}.
Here it is used purely as an admissibility convention for windowed sums
and products.

\begin{definition}[Parity--admissible window index set]
\label{def:parity-admissible}
Let \(n\in\mathbb{N}\) and let \(M\in[0,n]\).
The symmetric window is defined as
\begin{equation}
I_M
:=
\{\, m\in\mathbb{Z} : 0<|m|\le M \text{ and } 3\le n-|m| \,\}.
\end{equation}
The \emph{parity--admissible index set} is
\begin{equation}
\Ipar(n;M)
:=
\{\, m\in I_M : n+m \equiv 1 \pmod{2} \,\}.
\end{equation}
Equivalently, \(\Ipar(n;M)\) consists of those shifts \(m\) with
\(0<|m|\le M\) for which both \(n-m\) and \(n+m\) are odd integers at least \(3\).
Unless stated otherwise, all summations over window variables \(m\) in
this paper are implicitly restricted to \(m\in\Ipar(n;M)\).
\end{definition}

\begin{definition}[Euler--cap operator]
\label{def:euler-cap-operator}
The PCGC--Goldbach conjecture is formulated only for
\emph{Euler--cap admissible} window radii.
To allow formulas to be written uniformly in terms of an unconstrained
window parameter, the following coercion operator is introduced.

Fixing \(n\in\mathbb{N}\), for any \(M\ge 0\) measured in the same units as the
window radius, the following is defined:
\begin{equation}
\label{eq:ecap-operator}
\ecap{M}
\;:=\;
\left\lfloor
\min\!\left(
M,\;
\Ecap(n)\,n
\right)
\right\rfloor
\;=\;
\left\lfloor
\min\!\left(
M,\;
\frac{(2n+1)-\sqrt{8n+1}}{2}
\right)
\right\rfloor.
\end{equation}

Thus \(\ecap{M}\) denotes the largest Euler--cap admissible window radius
not exceeding \(M\).
\end{definition}

\medskip
\noindent
\textbf{Convention.}
The Euler--cap operator \(\ecap{\cdot}\) enforces admissibility of window
parameters and is applied only when explicitly written.
This allows the geometric and analytic lemmas of this paper to remain valid
independently of the particular saturation mechanism assumed.
When PCGC--Goldbach is invoked, windowed quantities are evaluated at the
Euler--capped scale, e.g.

\begin{equation}
\GHL(2n;\ecap{M}), \qquad \Gmeas(2n;\ecap{M}).
\end{equation}

\subsection{Effective Local Moduli}

\begin{definition}[Effective local modulus]
\label{def:Qp}
Let \(n\in\mathbb{N}\) and let \(p\) be an odd prime.
The minimum contributing prime is defined as
\begin{equation}
p_{\min}(n)
   :=
\begin{cases}
3, & 3 \mid n, \\[4pt]
5, & 3 \nmid n.
\end{cases}
\end{equation}

For each \(n\ge2\), let
\begin{equation}
\Peff(n):=\{\,p\in\mathbb{P} : p\ge \pmin(n)\,\}.
\end{equation}
The quantity \(\EffLocMod_p(n)\) is regarded as defined only for \(p\in\Peff(n)\).

For any odd prime \(q_{\min}\), define the partial Euler product
\begin{equation}
\EffLocMod_p^{(q_{\min})}
   :=
   \prod_{\substack{
      q \in \mathbb{P} \\
      q_{\min} \le q \le p
   }}
   (q-1).
\end{equation}
The \emph{effective local modulus} at \(p\) for the even integer \(2n\) is
then
\begin{equation}
\EffLocMod_p(n)
   := \EffLocMod_p^{\bigl(p_{\min}(n)\bigr)}.
\end{equation}
\end{definition}
Thus \(\EffLocMod_p(n)\) encodes the cumulative residue structure imposed by the
odd primes below \(p\), with the only \(n\)-dependence arising from the
choice of base prime \(p_{\min}(n)\) according to whether \(3\mid n\).
This convention is tailored to the singular--series geometry underlying
Goldbach--type problems.

\begin{definition}[Effective Moduli Interval Max]
\begin{align}
\EffLocMod(2n;L) 
&:= \EffLocMod_{P_0(2n;L)}(n), \\[3pt]
P_0(2n;L)
&:= \max\{\,p\in\Peff(n) :\ p\mid n,\ \QpleL\,\}.
\end{align}
If the set is empty, define \(P_0(2n;L):=\pmin(n)\).
\end{definition}

\medskip
Table~\ref{tab:EffLocMod-values} lists the values of the effective local modulus
\(\EffLocMod_p(n)\) for the two admissible base primes \(p_{\min}(n)=3\) and
\(p_{\min}(n)=5\), corresponding respectively to the cases \(3\mid n\) and
\(3\nmid n\).
The rapid growth of \(\EffLocMod_p(n)\) reflects the cumulative restriction imposed
by successive odd primes on admissible residue classes in Goldbach-type
configurations.


\subsection{Prime Curvature Constants}

The prime curvature constants are now introduced, which encode the cumulative
medium-- and large--prime contribution to the geometric remainder envelope.

\begin{definition}[Prime Curvature Constants]
The base prime is fixed as \(q_{\min}=5\), and the auxiliary effective moduli
\(\EffLocMod_p^{(5)}\) from Definition~\ref{def:Qp} are recalled.  
The \emph{prime curvature constant} is defined by the convergent Euler product
\begin{equation}
\label{eq:def-OmegaPrime}
\OmegaPrime
   :=
   \prod_{\substack{p\in\Peff(5)}}
   (p-2)^{1/\EffLocMod_p^{(5)}} .
\end{equation}
Numerically, \(\OmegaPrime \approx \xOmegaPrime\).

For each even integer \(2n\) and Euler--cap admissible window scale \(L\),
the \emph{renormalized prime curvature factor} is defined as
\begin{equation}
\label{eq:def-OmegaPrimeNorm}
\OmegaPrimeNorm(2n;L)
   :=
   \OmegaPrime^{\kappa(n)}
   \prod_{\substack{p\in\Peff(n)\\ \EffLocMod_p(n)\le L}}
      (p-2)^{-1/\EffLocMod_p(n)}
\end{equation}
where the exponent \(\kappa(n)\in\{\tfrac12,1\}\) compensates for the
choice of base prime \(\pmin(n)\in\{3,5\}\) in Definition~\ref{def:Qp}.

Specifically, when \(3\nmid n\) one has \(\pmin(n)=5\) and \(\kappa(n)=1\).
When \(3\mid n\), the extra contribution from the prime
\(3\) is absorbed into the exponent \(\kappa(n)=\tfrac12\).
\end{definition}


\begin{definition}[Bounding Envelope Constants]
The envelope constant for \( L \ge \pmin(n) \) is given explicitly by
\begin{equation}
\label{eq:c-explicit}
c(2n;L)=c\!\bigl(2n;\EffLocMod_p(n)\bigr)
      \;=\;
      2\prod_{\substack{q \in \Peff(n) \\ \EffLocMod_q(n) > \EffLocMod_p(n)}} (q-2)^{\,\EffLocMod_p(n)/\EffLocMod_q(n)}.
\end{equation}
In particular, \(c(2n;L)\) is constant on each envelope interval
\([\EffLocMod_p(n),\,\EffLocMod_{p^+}(n))\).
\end{definition}

\subsection{Goldbach Singular Series Factors}

\begin{definition}[Goldbach Singular Series Factors]
\label{def:SGB-Ssem}
For an even integer \(2n\), the \emph{local semiprime correction
factor}~\cite{HardyLittlewood1923} is defined as
\begin{equation}
\Ssem(2n)
   :=
   \prod_{\substack{p\mid n \\ p>2}}
      \frac{p-1}{p-2}.
\end{equation}
The \emph{prime--pair constant}~\cite{HardyLittlewood1923} is defined as
\begin{equation}
C_2
   :=
   \prod_{p>2}\left(1-\frac{1}{(p-1)^2}\right).
\end{equation}
The corresponding \emph{Goldbach Singular Series Factor}~\cite{HardyLittlewood1923} is
\begin{equation}
\SGB(2n)
   :=
   2\,C_2\,\Ssem(2n).
\end{equation}
\end{definition}

\begin{definition}[Hardy-Littlewood Circle Method Correction Factor]
For \( n \ge 2 \) and \( n > M \ge 1 \), and a weight function \( \omega \), the following is defined:
 \begin{equation}
 \HLCorr(2n;M) 
 := \frac{1}{|\Ipar(n,M)|\;\omega^2(n)} \sum_{m \in \Ipar(n,M)} \omega(n-m)\omega(n+m).
\end{equation}
\label{def:HLCorr}
\end{definition}

\begin{definition}[Measured Goldbach Count]
The measured number of Goldbach pairs in the window is defined as
\begin{equation}
\Gmeas(2n;M)
:=
\sum_{m\in\Ipar(n,M)}
1_{\mathrm{prime}}(n-m)\,1_{\mathrm{prime}}(n+m).
\end{equation}
\end{definition}

\begin{definition}[Hardy--Littlewood Window Predictor (HL--Windowed)]
\label{def:GHL}
Let \(\omega(x)\) denote the standard prime weight
(e.g.\ \(\omega(x)=1/\log x\)).
The following is defined:
\begin{align}
\GHL(2n;M)
&:=
2\,C_2\,\Ssem(2n)
\sum_{m\in\Ipar(n,M)}
\omega(n-m)\,\omega(n+m), \\[3pt]
&=
2\,C_2\,\Ssem(2n)\,\HLCorr(2n;M)\,|\Ipar(n,M)|\,\omega^2(n).
\end{align}
where \(\Ssem(2n)\) is the classical Hardy--Littlewood Semiprime correction.
\end{definition}

When \( |\Ipar(n,M)|=2M \) (as in the standard symmetric parity-restricted window),
\begin{equation}
\GHL(2n;M)
= 2C_2\,\Ssem(2n)\,\HLCorr(2n;M)\,2M\,\omega^2(n).
\end{equation}

\begin{remark}
The predictor \( \GHL(2n;M) \) separates global semiprime structure
\( \Ssem(2n) \) from local window geometry \( \HLCorr(2n;M)) \),
with normalization determined by the block mass \( |\Ipar(n,M)|\omega^2(n) \).
\end{remark}

\begin{definition}[Predictor proxy]
\label{def:predictor-proxy}
The notation
\(
\Gproxy(2n;M)
\)
denotes an explicit analytic approximation to the windowed Goldbach predictor
\(G(2n;M)\) such that there exists a function
\(\eta(2n;M)\) with
\begin{equation}
\eta(2n;M)\to 0
\qquad (M\to\infty \text{ along admissible scales})
\end{equation}
for which
\begin{equation}
\label{eq:Gproxy-def}
\Gmeas(2n;M)
=
\Gproxy(2n;M)\,\bigl(1+\eta(2n;M)\bigr).
\end{equation}
The function \(\eta(2n;M)\) is referred to as the \emph{proxy correction}.
\end{definition}

\begin{definition}[Hardy--Littlewood window proxy predictor]
\label{def:GHLproxy}
The Hardy--Littlewood proxy predictor is defined as
\begin{equation}
\GHLproxy(2n;M)
:=2\,C_2\,\Ssem(2n)\,|\Ipar(n,M)|\,\omega^2(n).
\end{equation}
\end{definition}

\noindent
By Definitions~\ref{def:HLCorr} and \ref{def:GHLproxy}, the exact identity
\begin{equation}
\label{eq:GHL-factorization}
\GHL(2n;M)
=\GHLproxy(2n;M)\,\HLCorr(2n;M).
\end{equation}

\noindent
For \( \omega(x) = O\!\left(\frac{1}{\log x}\right), \) Lemma~\ref{lem:GHL-proxy-trivial} proves that \(\GHLproxy(2n;M)\) is a predictor
proxy in the same sense as Definition~\ref{def:predictor-proxy}.

\subsection{Complementary and Full Euler--Type Products}

In addition to the local semiprime correction factor \(\Ssem(2n)\) defined
above, it is occasionally convenient to refer to the complementary and full
Euler--type products obtained by modifying the divisibility condition on the
prime index.

\begin{definition}[Complementary and full local semiprime products]
\label{def:Ssem-complement-full}
The \emph{complementary} local semiprime product is defined as
\begin{equation}
\Ssem^{\complement}(2n)
   :=
   \prod_{\substack{p\in\Podd \\ p\nmid n}}
      \frac{p-1}{p-2},
\end{equation}
and the corresponding \emph{full} product by
\begin{equation}
\Ssem^{\bullet}(2n)
   :=
   \prod_{\substack{p\in\Podd}}
      \frac{p-1}{p-2}.
\end{equation}
\end{definition}

\begin{convention}[Notation]
A superscript \( \complement \) indicates replacement of the divisibility
condition \(p\mid n\) by \(p\nmid n\), while a superscript \( \bullet \)
indicates removal of the divisibility condition entirely. No additional
structure is implied.
\end{convention}

\subsection{Additional Series Operators}

The classical Hardy--Littlewood singular series \(\Ssem(2n)\) is naturally an
asymptotic object.  For finite--window analysis it is convenient to
introduce auxiliary operators that partition the series into base and
tail components relative to a square cutoff.

\begin{definition}[Cutoff Operators]
The square cutoff operator is defined as
\begin{equation}
\SquareCap(x):=x^2.
\end{equation}
The function \( \EffLocModCap(\cdot) \) may also be used as a cutoff operator.
\end{definition}

The tags \( \thead \) and \( \ttail \) are used to denote the base (small--prime)
and tail (large--prime) components relative to a given cutoff.

\begin{definition}[Cutoff Components of \( \Ssem \)]
Let \( M>0 \) and \( n\ge 2 \). The following are defined:
\begin{equation}
\SsemHead^{\SquareCap}(2n;M)
:= \prod_{\substack{p\in\Podd \\ p\mid n \\ p^2\le M}}
\frac{p-1}{p-2} , \qquad
\SsemTail^{\SquareCap}(2n;M)
:= \prod_{\substack{p\in\Podd \\ p\mid n \\ p^2> M}}
\frac{p-1}{p-2}.
\end{equation}
\begin{equation}
 \SsemHead^{\EffLocModCap}(2n;L)
:= \prod_{\substack{p\in\Peff(n) \\ p\mid n\\ \QpleL}}
\frac{p-1}{p-2}, \qquad
\SsemTail^{\EffLocModCap}(2n;L)
:= \prod_{\substack{p\in\Peff(n) \\ p\mid n \\ \EffLocMod_p(n) > L}}
\frac{p-1}{p-2}.
\end{equation}
\begin{equation}
\Ssem^{\SquareCap}(2n)
:= \Ssem(2n), \qquad
\Ssem^{\EffLocMod}(2n)
:= \prod_{\substack{p\in\Peff(n)\\ p\mid n}}
\frac{p-1}{p-2}.
\end{equation}

Empty products are interpreted as \(1\).
\end{definition}

\noindent
The following identities from \cite{Riemers2026-PCGH} hold,
\begin{align}
\label{ident:ssem}
\Ssem(2n) = \Ssem^{\SquareCap}(2n)
&= \SsemHead^{\SquareCap}(2n;M)\;
\SsemTail^{\SquareCap}(2n;M) \qquad \forall M > 0, \\[3pt]
\Ssem^{\EffLocMod}(2n)
&= \SsemHead^{\EffLocModCap}(2n;L)\;
\SsemTail^{\EffLocModCap}(2n;L) \qquad \forall L > 0, \\[3pt]
\Ssem^{\EffLocModCap}(2n)
&=\Ssem^{\SquareCap}(2n), \\[3pt]
\SsemHead^{\EffLocModCap}(2n;L)
&=
\SsemHead^{\SquareCap}(2n;(P_0(2n;L))^2) \qquad \forall L > 0, \\[3pt]
\SsemTail^{\EffLocModCap}(2n;L)
&=
\SsemTail^{\SquareCap}(2n;(P_0(2n;L))^2) \qquad \forall L > 0, \\[3pt]
\Ssem^{\SquareCap,\complement}(2n)
&=
\SsemHead^{\SquareCap,\complement}\!\left(2n;(\pmin(n)-1)^2\right)\;
\Ssem^{\EffLocModCap,\complement}(2n) \qquad \forall n \ge \pmin(n).
\end{align}

\subsection{Specialized Remainder Decomposition}

This leads up to the PCGC--Goldbach Remainder.

\begin{definition}[PCGC--Goldbach Remainder]\label{def:RHL}
For \( \ecap{M} > 0 \), the following is defined:
\begin{align}
\RHL(2n;L)
&:=
2\,\frac{\EffLocMod(2n;L)}{\SsemHead^{\EffLocModCap,\complement}(2n;L)}\,
\bigl(\OmegaPrimeNorm(2n;L)\bigr)^L,
\qquad L := \sqrt{2\ecap{M}} \\[3pt]
&=
2\,\frac{\EffLocMod(2n;L)}{\SsemHead^{\EffLocModCap,\complement}(2n;L)}\,\Xi(2n;L),
\qquad \Xi(2n;L):=\bigl(\OmegaPrimeNorm(2n;L)\bigr)^L.
\end{align}
\end{definition}

\noindent
The associated bounding envelope is defined as:

\begin{definition}[PCGC--Goldbach Bound Envelope]\label{def:RHLbound}
For \( 2 \ecap{M} \ge (\pmin(n)-1)^2 \), the following is defined:
\begin{equation}
\RHLbound(2n;L)
:=
\frac{c(2n;L)\,L}{\SsemHead^{\EffLocModCap,\complement}(2n;L)},
\qquad L := \sqrt{2\ecap{M}}.
\end{equation}
\end{definition}

\subsubsection*{Role of the Remainder Bound}

Lemma~(Overall Bounding Envelope) of \cite{Riemers2025} shows that
\begin{equation}
\RHL(2n;L)\le \RHLbound(2n;L)
\end{equation}
for all admissible \(L\).
The role of \(\RHLbound\) is therefore one of
\emph{pointwise control} (a certified upper bound), not approximation.
In particular, \(\RHLbound\) does not satisfy
\(\RHLbound(2n;L)/\RHL(2n;L)\to 1\),
and may overestimate \(\RHL(2n;L)\) by a scale-dependent factor.
This distinction is essential:
treating \(\RHLbound\) as an approximation would lead to systematic
misinterpretation of finite-scale behaviour and apparent overestimation
in numerical comparisons.

\section{Main Conjectures}
\label{sec:main-conjectures}

\subsection{PCGC--Goldbach}

Any conditional theorem, lemma, or corollary derived from any conjecture may be viewed,
in isolation, as a new weaker or equivalent conjectural statement.
The primary conjecture examined in this work---PCGC--Goldbach, introduced in
\cite{Riemers2026-PCGH}, is restated here using the \(\ecap{\cdot}\) operator.

\begin{conjecture}[Prime Curvature Geometry Conjecture for Goldbach (PCGC--Goldbach)]
\label{con:PCGC--Goldbach}
For every even integer \(2n \ge 4\) and every admissible window size
\(\ecap{M}\), the Goldbach pair counts satisfy
\begin{equation}
  \bigl|
    \Gmeas(2n;\ecap{M}) - \GHL(2n;\ecap{M})
  \bigr|
  \le
  \RHL(2n;L),
  \qquad L := \sqrt{2\ecap{M}},
\end{equation}
where the remainder term \(\RHL(2n;L)\) is defined in
Definition~\ref{def:RHL}.
\end{conjecture}

Throughout this paper, PCGC--Goldbach or the weaker PCGC--Goldbach Bounds
is treated as a working hypothesis.
Each reduction derived from this assumption yields a logically weaker statement
that can, in principle, be verified or falsified independently.
To the extent that such reduced statements are confirmed, either analytically
or empirically, they provide indirect support for the PCGC--Goldbach framework.

\subsection{PCGC--Goldbach Bounds}

The inequality \(\RHL(2n;L) \le \RHLbound(2n;L)\) has been proven in the
Bounding Envelope section of \cite{Riemers2026-PCGH}.
Discarding the sharper remainder \(\RHL\) in favour of its envelope leads to
a strictly weaker conjecture.

\begin{conjecture}[Prime Curvature Bounding Conjecture for Goldbach (PCGC--Goldbach Bounds)]
\label{con:PCGC--Goldbach-Bounds}
For every even integer \(2n \ge 4\) and every admissible window size
\(\ecap{M}\), the Goldbach pair counts satisfy
\begin{equation}
  \bigl|
    \Gmeas(2n;\ecap{M}) - \GHL(2n;\ecap{M})
  \bigr|
  \le
  \RHLbound(2n;L),
  \qquad L := \sqrt{2\ecap{M}},
\end{equation}
where the bounding envelope \(\RHLbound(2n;L)\) is defined in
Definition~\ref{def:RHLbound}.
\end{conjecture}

Most reductions in this paper depend only on PCGC--Goldbach Bounds rather than
the full PCGC--Goldbach conjecture.
Consequently, these reductions do not in themselves strengthen the case for
the sharper remainder \(\RHL\), but they substantially widen the scope of
independent validation for PCGC--Goldbach Bounds.

In particular, the heuristic and empirical analysis of \cite{Riemers2025}
shows that the \(\Lambda\)-curves for up to
\(2n = 23\#\) exhibit bounded convergence toward the predicted limit.
This behaviour does not follow from Hardy--Littlewood asymptotics alone,
nor from Goldbach's conjecture itself, but is naturally explained by a
bounding principle of the form asserted in PCGC--Goldbach Bounds.

\section{Geometric Framework}
\label{sec:geometric-framework}

\subsection{Goldbach Full Reduction}

\begin{theorem}[PCGC--Goldbach Bounds implies Strong Goldbach]
\label{thm:strong-goldbach}
Assume that Conjecture~\ref{con:PCGC--Goldbach-Bounds},
PCGC--Goldbach Bounds, holds for all \(n \ge \xGoldbachN\).
With the window choice
\begin{equation}
M := \ecap{n/2}, \qquad L := \sqrt{2M},
\end{equation}
assume in particular that
\begin{equation}
\label{eq:PCGC--Goldbach-bounds}
\Gmeas(2n;M) \;\ge\; \GHL(2n;M) \;-\; \RHLbound(2n;L)
\end{equation}
holds for all such \(n\).

Let \(N(2n)\) denote the number of ordered Goldbach representations of \(2n\),
including the diagonal pair when \(n\) is prime, i.e.
\begin{equation}
N(2n) := \Gmeas(2n) + 1_{\mathrm{prime}}(n).
\end{equation}
Then
\begin{equation}
N(2n) \ge 1 \qquad \forall n \ge 2.
\end{equation}
\end{theorem}

\begin{proof}
Let \(n_0 := \xGoldbachN \).

\paragraph{Case 1: \(2 \le n < n_0\).}
By direct verification \cite{Riemers2025,Silvia}, \(N(2n)\ge 1\) on this finite range.

\paragraph{Case 2: \(n \ge n_0\).}
Set \(M:=\ecap{n/2}\) and \(L:=\sqrt{2M}\).  Since the full count contains any sub-count,
\begin{equation}
\label{eq:full-dominates-window}
\Gmeas(2n)\;\ge\;\Gmeas(2n;M).
\end{equation}
Combining \eqref{eq:full-dominates-window} with \eqref{eq:PCGC--Goldbach-bounds} yields
\begin{equation}
\label{eq:meas-lower-1}
\Gmeas(2n)\;\ge\;\GHL(2n;M)\;-\;\RHLbound(2n;L).
\end{equation}

\paragraph{Step 1: Expand \(\GHL(2n;M)\).}
By the definition of \(\GHL(2n;M)\) and Lemma~\ref{lem:hlcorr-norm-identity},
\begin{align}
\GHL(2n;M)
&= 2C_2\,\Ssem(2n)\,\sum_{m\in\Ipar(n,M)} \omega(n-m)\,\omega(n+m) \notag\\
&= 2C_2\,\Ssem(2n)\,\bigl(2M\,\omega(n)^2\,\HLCorr(2n;M)\bigr). \label{eq:ghl-expand}
\end{align}

\paragraph{Step 2: Lower bound the \(\omega\) factor.}
Assume \(\omega(n)=\pi(n)/n\).  Then by Rosser--Schoenfeld \cite{RosserSchoenfeld1962},
\(\pi(n) > n/\log n\) for all \(n\ge 17\), hence \(\omega(n)^2 \ge 1/\log^2 n\) for all \(n\ge 17\).
Therefore, for all \(n\ge n_0\),
\begin{equation}
\label{eq:ghl-lower}
\GHL(2n;M)
\;\ge\; 2C_2\,\Ssem(2n)\,\frac{2M}{\log^2 n}\,\HLCorr(2n;M).
\end{equation}

\paragraph{Step 3: Replace \(\HLCorr\) by a lower bound.}
By Lemma~\ref{lem:HLCorr-min}, \( \HLCorr(2n;M) \ge 1 \),
\begin{equation}
\label{eq:ghl-lower-2}
\GHL(2n;M)
\;\ge\; 2C_2\,\Ssem(2n)\,\frac{2M}{\log^2 n}.
\end{equation}

\paragraph{Step 4: Upper bound the envelope term.}
By the definition of the envelope
\begin{equation}
\label{eq:env-upper}
\RHLbound(2n;L)
:= \frac{c(2n;L)\,L}{\SsemHead^{\EffLocModCap,\complement}(2n;L)}.
\end{equation}
Applying \( \Ssem \) identity Equations~\ref{ident:ssem},
\begin{align}
\RHLbound(2n;L)
&= \frac{c(2n;L)\,L\,\SsemHead^{\EffLocModCap}(2n;L)\,\SsemTail^{\EffLocModCap}(2n;L)}{\SsemHead^{\EffLocModCap,\complement}(2n;L)\,\SsemHead^{\EffLocModCap}(2n;L)\,\SsemTail^{\EffLocModCap}(2n;L)}, \\[3pt]
&= \frac{c(2n;L)\,L\,\Ssem(2n)}{\SsemHead^{\EffLocModCap,\bullet}(2n;L)\,\SsemTail^{\EffLocModCap}(2n;L)}.
\end{align}

Since by all product terms of the \( \Ssem  \) are greater than \( 1 \), 
\( \SsemTail^{\EffLocModCap,\complement}(2n;L) \ge 1 \) this bound maybe weakened the bound,
\begin{equation}
\label{eq:env-upper-2}
\RHLbound(2n;L)
\;\le\;
\frac{c(2n;L)\,L\Ssem(2n)}{\SsemHead^{\EffLocModCap,\bullet}(2n;L)}.
\end{equation}

\paragraph{Step 5: Combine and normalize in \(L\).}
Insert \eqref{eq:ghl-lower-2} and \eqref{eq:env-upper-2} into \eqref{eq:meas-lower-1}:
\begin{align}
\Gmeas(2n)
&\ge 2C_2\,\Ssem(2n)\,\frac{2M}{\log^2 n} \;-\; \frac{c(2n;L)\,L\,\Ssem(2n)}{\SsemHead^{\EffLocModCap,\bullet}(2n;L)} \notag\\
&\ge 2C_2\,\frac{2M}{\log^2 n} \;-\; \frac{c(2n;L)\,L}{\SsemHead^{\EffLocModCap,\bullet}(2n;L)}
\qquad(\text{since }\Ssem(2n)>0). \label{eq:meas-lower-2}
\end{align}
Now use \(2M=L^2\):
\begin{equation}
\label{eq:meas-lower-L}
\Gmeas(2n)
\;\ge\;
L\left(2C_2\,\frac{L}{\log^2 n} - \frac{c(2n;L)}{\SsemHead^{\EffLocModCap,\bullet}(2n;L)}\right).
\end{equation}

\paragraph{Step 6: Monotonicity Reduction to \(n=n_0\).}
Since the function \(\sqrt{n}/\log^2 n\) is increasing for all \(n\ge n_0\),
and since the envelope coefficient \(c(2n;L)\) is uniformly bounded above
on \(n\ge n_0\) at the scale \(L=\sqrt{2\ecap{n/2}}\) by the
\emph{Monotonic Envelope Lemma} of \cite{Riemers2026-PCGH}, and
\( \SsemHead^{\EffLocModCap,\bullet}(2n;L) \) is monotonic increasing,
the bracket in \eqref{eq:meas-lower-L} attains its minimum at \(n=n_0\).
Hence, for all \(n\ge n_0\),
\begin{equation}
\label{eq:meas-lower-n0}
\Gmeas(2n)
\;\ge\;
L\left(2C_2\,\frac{L}{\log^2 n} - \frac{c(2n;L)}{\SsemHead^{\EffLocModCap,\bullet}(2n;L)}\right)
\;\ge\;
\sqrt{n_0}\left(2C_2\,\frac{\sqrt{n_0}}{\log^2 n_0} - \frac{c(2n_0;\sqrt{n_0})}{\SsemHead^{\EffLocModCap,\bullet}(2n_0;\sqrt{n_0})}\right).
\end{equation}
Evaluating the right-hand side of \eqref{eq:meas-lower-n0} at \(n=n_0\) using
the highest of the tabulated values
\begin{align}
\frac{c(2n_0;\sqrt{n_0})}{\SsemHead^{\EffLocModCap,\bullet}(2n_0;\sqrt{n_0})}
&= \xCratioDivC \qquad n\mid 3
\qquad
\text{(Table~\ref{tab:c-ratios})} \\
&=\xCratioNDivC \qquad n\nmid 3
\qquad
\text{(worst-case value)},
\end{align}
together with the known Hardy--Littlewood prime pair constant
\begin{equation}
C_2 = \xTwinPrimeConstant,
\end{equation}
yields a strictly positive value.

Since the bracket in \eqref{eq:meas-lower-L} is minimized at \(n=n_0\),
it follows that \(\Gmeas(2n)>0\) for all \(n\ge n_0\).
Because \(\Gmeas(2n)\in\mathbb{Z}_{\ge 0}\), it is concluded that
\(\Gmeas(2n)\ge 1\) for all \(n\ge n_0\).

Combining Cases 1 and 2 completes the proof.
\end{proof}

\begin{remark}[Choice of \(\alpha\).]
In the proof, \(\alpha=\tfrac12\) is taken solely to remain comfortably below
the Euler cap \(\ecap{\cdot}\) and to permit direct comparison with existing
computational data.
No structural feature of the argument depends on this specific choice.

The same proof applies for any fixed \(0<\alpha\le 1\), provided the
PCGC--Goldbach Bounds Conjecture is assumed in the corresponding range of window
sizes.
Keeping with the principle of minimizing assumptions, Conjecture~\ref{con:PCGC--Goldbach-Bounds} 
is therefore assumed only for the values of \(\alpha\) and \(n\) actually required in the argument.
\end{remark}

\subsection{HL--A Reductions}

\subsubsection{Euler Capped HL--Windowed Reduction}

\begin{theorem}[PCGC--Goldbach Bounds Implies Capped HL--Windowed Proxy]
\label{thm:pcgc-to-HL-windowed-cap}
Assume the PCGC--Goldbach Bounds \eqref{eq:PCGC--Goldbach-bounds} hold for all \( n\ge n_\ast \)
in the window scaling regime
\begin{equation}
\label{eq:M-scaling}
M:=\ecap{n\alpha},\qquad 0<\alpha \le 1,\qquad M\ge 1.
\end{equation}
Let \( \GHLproxy(2n;M) \) denote the Hardy--Littlewood proxy predictor
(as defined in \S\ref{def:predictor-proxy}).
Then there exist an explicit \( n_0\ge n_\ast \) and an explicit envelope
\( \eta(2n;M)\to 0 \) such that for all \( n\ge n_0 \),
\begin{equation}
\label{eq:HLA-cap-proxy}
\left|
\frac{\Gmeas(2n;M)}{\GHLproxy(2n;M)}-1
\right|
\le \eta(2n;M),
\qquad (n\to\infty \text{ with } M=\ecap{n\alpha}).
\end{equation}
Equivalently,
\begin{equation}
\label{eq:HLA-cap-asymp}
\Gmeas(2n;M)
=
\left(1+O(\eta(2n;M))\right)\,
\GHLproxy(2n;M),
\qquad (n\to\infty \text{ with } M=\ecap{n\alpha}).
\end{equation}
\end{theorem}

\begin{proof}
Fix \(n\ge n_\ast\) and let \(M=\ecap{n\alpha}\) with \(0<\alpha\le 1\) and \(M\ge 1\),
and set \(L:=\sqrt{2\ecap{M}}\).

By PCGC--Goldbach Bounds \eqref{eq:PCGC--Goldbach-bounds},
\begin{equation}\label{eq:pcgc-step}
\bigl|\Gmeas(2n;M)-\GHL(2n;M)\bigr|\le \RHLbound(2n;L).
\end{equation}
Divide \eqref{eq:pcgc-step} by \(\GHLproxy(2n;M)>0\) to obtain
\begin{equation}\label{eq:divide-proxy}
\left|
\frac{\Gmeas(2n;M)}{\GHLproxy(2n;M)}-\frac{\GHL(2n;M)}{\GHLproxy(2n;M)}
\right|
\le
\frac{\RHLbound(2n;L)}{\GHLproxy(2n;M)}.
\end{equation}

By the exact factorization \eqref{eq:GHL-factorization},
\begin{equation}
\frac{\GHL(2n;M)}{\GHLproxy(2n;M)}=\HLCorr(2n;M).
\end{equation}
Hence, by the triangle inequality applied to
\(\frac{\Gmeas}{\GHLproxy}-1=
\left(\frac{\Gmeas}{\GHLproxy}-\HLCorr\right)+(\HLCorr-1)\),
it follows that
\begin{equation}\label{eq:split}
\left|
\frac{\Gmeas(2n;M)}{\GHLproxy(2n;M)}-1
\right|
\le
\frac{\RHLbound(2n;L)}{\GHLproxy(2n;M)}
+
\bigl|\HLCorr(2n;M)-1\bigr|.
\end{equation}

\smallskip
\noindent\textbf{Bounding the PCC Term.}
Using Definition~\ref{def:RHLbound} and Definition~\ref{def:GHLproxy},
and the inequalities \(\Ssem(2n)\ge 1\) and \(\Ssem^{\EffLocModCap,\complement}(2n;L)\ge 1\),
\begin{align}
\frac{\RHLbound(2n;L)}{\GHLproxy(2n;M)}
&=
\frac{c(2n;L)\,\EffLocMod(2n;L)\,L}{\Ssem^{\EffLocModCap,\complement}(2n;L)}
\cdot
\frac{1}{2C_2\,\Ssem(2n)\,|\Ipar(n,M)|\,\omega(n)^2}
\\[3pt]
&\le
\frac{c(2n;L)\,L}{2C_2\,|\Ipar(n,M)|\,\omega(n)^2}.
\end{align}
In the standard symmetric parity window, \(|\Ipar(n,M)|=2M\), so
\begin{equation}\label{eq:pcgc-term-simplified}
\frac{\RHLbound(2n;L)}{\GHLproxy(2n;M)}
\le
\frac{c(2n;L)}{4C_2}\cdot\frac{L}{M}\cdot\frac{1}{\omega(n)^2}.
\end{equation}
Since \(L=\sqrt{2\ecap{M}}\) and \(M=\ecap{n\alpha}\), it follows that \(L/M \asymp 1/L\),
so \eqref{eq:pcgc-term-simplified} gives
\begin{equation}\label{eq:pcgc-term-L}
\frac{\RHLbound(2n;L)}{\GHLproxy(2n;M)}
\le
\frac{c(2n;L)}{4C_2}\cdot\frac{1}{L\,\omega(n)^2}.
\end{equation}
By the \emph{Monotonic Envelope Lemma} of \cite{Riemers2026-PCGH},
\(c(2n;L)\) is nonincreasing in \(L\), hence for all admissible \((n,L)\),
\begin{equation}
c(2n;L)\le c(6;2),
\end{equation}
so
\begin{equation}\label{eq:pcgc-term-final}
\frac{\RHLbound(2n;L)}{\GHLproxy(2n;M)}
\le
\frac{c(6;2)}{4C_2}\cdot\frac{1}{L\,\omega(n)^2}.
\end{equation}

Finally, using the explicit inequality \(\pi(n)\ge n/\log n\) for \(n\ge 17\)
(Rosser--Schoenfeld \cite{RosserSchoenfeld1962}), it follows that
\(\omega(n)=\pi(n)/n\ge 1/\log n\), hence for \(n\ge 17\),
\begin{equation}\label{eq:pcgc-term-log}
\frac{\RHLbound(2n;L)}{\GHLproxy(2n;M)}
\le
\frac{c(6;2)}{4C_2}\cdot \frac{\log^2 n}{L}.
\end{equation}

\smallskip
\noindent\textbf{Bounding the \(\HLCorr\) Term.}
By Lemma~\ref{lem:HLCorr-upper-envelope} (Endpoint Envelope for Fixed \(\alpha\)),
\begin{equation}
\HLCorr(2n;M)\le U_\alpha(n),
\qquad
U_\alpha(n)\to 1 \ (n\to\infty),
\end{equation}
and by Lemma~\ref{lem:HLCorr-min}, \(\HLCorr(2n;M)\ge 1\).
Therefore
\begin{equation}\label{eq:hlcorr-term}
0\le \HLCorr(2n;M)-1 \le U_\alpha(n)-1,
\qquad U_\alpha(n)-1\to 0.
\end{equation}

Combining \eqref{eq:split}, \eqref{eq:pcgc-term-log}, and \eqref{eq:hlcorr-term},
the following is defined
\begin{equation}
\eta(2n;M)
:=
\frac{c(6;2)}{4C_2}\cdot\frac{\log^2 n}{\sqrt{2\ecap{n\alpha}}}
+\bigl(U_\alpha(n)-1\bigr),
\end{equation}
which satisfies \(\eta(2n;M)\to 0\) as \(n\to\infty\) with \(M=\ecap{n\alpha}\).
This proves \eqref{eq:HLA-cap-proxy} and hence \eqref{eq:HLA-cap-asymp}.
\end{proof}

\subsubsection{Uncapped Capped HL--Windowed Reduction}

\begin{lemma}[PCGC--Goldbach Implies HL--Windowed Proxy at Uncapped Scale]
\label{lem:pcgc-to-HL-windowed-uncap}
Assume the hypotheses of Theorem~\ref{thm:pcgc-to-HL-windowed-cap}.
In the window scaling regime
\begin{equation}
\label{eq:M-scaling-uncap}
M:=n\alpha,\qquad 0<\alpha\le 1,\qquad M\ge 1,
\end{equation}
let \( \GHLproxy(2n;M) \) denote the Hardy--Littlewood proxy predictor
(as defined in \S\ref{def:predictor-proxy}).
Then there exists an explicit \( n_0\ge n_\ast \) and an explicit envelope
\( \tilde\eta(2n;M)\to 0 \) such that for all \( n\ge n_0 \),
\begin{equation}
\label{eq:HLA-uncap-proxy}
\left|
\frac{\Gmeas(2n;M)}{\GHLproxy(2n;M)}-1
\right|
\le \tilde\eta(2n;M),
\qquad (n\to\infty \text{ with } M=n\alpha).
\end{equation}
Equivalently,
\begin{equation}
\label{eq:HLA-uncap-asymp}
\Gmeas(2n;M)
=
\left(1+O(\tilde\eta(2n;M))\right)\,
\GHLproxy(2n;M),
\qquad (n\to\infty \text{ with } M=n\alpha).
\end{equation}
\end{lemma}

\begin{proof}
Let \(M=n\alpha\) with \(0<\alpha\le 1\), and let \(\ecap{M}\) be the Euler-capped
window size from \eqref{eq:ecap-operator}.  Write
\begin{equation}
\Delta M := M-\ecap{M}\ \ge\ 0.
\end{equation}
By definition of the Euler cap, \(\Delta M = O(\sqrt{n})\) as \(n\to\infty\).

Let \(\Ipar(2n;M)\) and \(\Ipar(2n;\ecap{M})\) be the parity-admissible index sets.
Since the cap removes only the two endpoint intervals of length \(\Delta M\),
the set inclusion
\begin{equation}
\Ipar(2n;\ecap{M}) \subseteq \Ipar(2n;M),
\end{equation}
and therefore the difference set has cardinality
\begin{equation}
\bigl|\Ipar(2n;M)\setminus \Ipar(2n;\ecap{M})\bigr|
\le 2\Delta M = O(\sqrt{n}).
\end{equation}
Each removed index \(m\) contributes at most \(1\) to the ordered pair count
\(\Gmeas(2n;M)\), hence
\begin{equation}\label{eq:meas-cap-diff}
0\le \Gmeas(2n;M)-\Gmeas(2n;\ecap{M}) \le O(\sqrt{n}).
\end{equation}

Now apply Theorem~\ref{thm:pcgc-to-HL-windowed-cap} at the capped scale \(\ecap{M}\):
there exists \(n_0\ge n_\ast\) and an envelope \(\eta(2n;\ecap{M})\to 0\) such that
\begin{equation}
\left|\frac{\Gmeas(2n;\ecap{M})}{\GHLproxy(2n;\ecap{M})}-1\right|
\le \eta_{\cap}(2n;\ecap{M}).
\end{equation}
Using \(\GHLproxy(2n;M)=2C_2\Ssem(2n)\,|\Ipar(n,M)|\,\omega(n)^2\) and
\(|\Ipar(n,M)|\asymp M\),
\begin{equation}
\GHLproxy(2n;\ecap{M}) \asymp \frac{\ecap{M}}{\log^2 n}
\asymp \frac{n}{\log^2 n},
\end{equation}
uniformly for fixed \(\alpha\in(0,1]\).  Dividing \eqref{eq:meas-cap-diff} by
\(\GHLproxy(2n;\ecap{M})\) therefore yields an additional relative error
\begin{equation}
\frac{O(\sqrt{n})}{\GHLproxy(2n;\ecap{M})}
=
O\!\left(\frac{\log^2 n}{\sqrt{n}}\right)
\to 0.
\end{equation}
The function \(\tilde\eta(2n;M)\) is defined as
\(\tilde\eta(2n;M) := \eta(2n;\ecap{M}) + O\!\left(\frac{\log^2 n}{\sqrt{n}}\right)\).
Then \(\tilde\eta(2n;M)\to 0\) as \(n\to\infty\) with \(M=n\alpha\).
\end{proof}

\subsubsection{Hardy--Littlewood (Conjecture A, Goldbach Form) Reduction}

\begin{corollary}[Hardy--Littlewood (Conjecture A, Goldbach Form)]
\label{cor:HL-A-standard}
Assuming the hypotheses of Lemma~\ref{lem:pcgc-to-HL-windowed-uncap}, the following is defined:
\begin{equation}
N(2n):=\Gmeas(2n)+1_{\mathrm{prime}}(n).
\end{equation}
Then the classical Hardy--Littlewood prediction (Conjecture~A) follows:
\begin{equation}
\label{eq:HL-A-standard}
N(2n)
\;\sim\;
2C_2\,\Ssem(2n)\,\frac{2n}{\log^2(2n)}
\qquad (n\to\infty).
\end{equation}
Equivalently,
\begin{equation}
N(2n)
=
\left(1+o(1)\right)\,2C_2\,\Ssem(2n)\,\frac{2n}{\log^2(2n)}.
\end{equation}
\end{corollary}

\begin{proof}
Apply Lemma~\ref{lem:pcgc-to-HL-windowed-uncap} with \(\alpha=1\), hence \(M=n\) and
\(\Gmeas(2n;M)=\Gmeas(2n)\).  With the Hardy--Littlewood normalization
\(\omega(x)=1/\log(2x)\), the proxy predictor becomes
\begin{equation}
\GHLproxy(2n;M)=2C_2\,\Ssem(2n)\,\frac{2n}{\log^2(2n)}.
\end{equation}
Thus
\begin{equation}
\Gmeas(2n)=\left(1+o(1)\right)\,2C_2\,\Ssem(2n)\,\frac{2n}{\log^2(2n)}.
\end{equation}
Finally, \(0\le 1_{\mathrm{prime}}(n)\le 1\), so adding the bounded term does not
change the asymptotic:
\begin{equation}
N(2n)=\Gmeas(2n)+1_{\mathrm{prime}}(n)\sim \Gmeas(2n).
\end{equation}
\end{proof}

\section{Short Intervals}
\label{sec:short-intervals}

In earlier work~\cite{Riemers2025} the Goldbach problem was reduced to a
short--interval Bombieri--Vinogradov--like hypothesis~\cite{Bombieri1965,Vinogradov1965}.  That
hypothesis is recalled here in a form adapted to Goldbach windows, and fit
into the geometric structure developed in this paper.

\paragraph{Scope.}
In an ideal development one would aim to deduce a short-interval
Bombieri--Vinogradov statement of the form
Hypothesis~\ref{hyp:SIBV-Goldbach} from PCGC--Goldbach itself.
Such a deduction is plausible in principle, but it would require extending
the PCC framework beyond centred Goldbach windows to families of
off-centred windows and residue-class statistics.
That extension is outside the scope of this present paper; here it is only
recorded that PCGC--Goldbach--Bounds already implies a relative agreement
statement in the same short-interval scaling regime.

\subsection{Short-Interval Bombieri--Vinogradov Like Hypothesis}

Under PCGC--Goldbach and the short--interval hypothesis below, the
relative agreement theorem again applies, yielding asymptotic control of
\(\Gmeas(2n;M)\) for all windows above the short--interval threshold.

\begin{hypothesis}[Short-Interval Bombieri--Vinogradov in Goldbach Windows]
\label{hyp:SIBV-Goldbach}
There exists \(\theta>\tfrac12\) and \(\varepsilon>0\) such that the following holds.
For every \(A>0\) and all sufficiently large \(n\),
\begin{equation}\label{eq:SIBV-Goldbach}
\sum_{q\le (2n)^{\theta}}
\ \max_{\substack{(a,q)=1}}\ \max_{M\ge (2n)^{\frac{1}{2}+\frac{\varepsilon}{2}}}
\left|\ P(n,M;q,a)\ -\ \frac{2M}{\varphi(q)\log n}\ \right|
\ \ \ll_{A,\varepsilon}\ \ \frac{2n}{(\log n)^A},
\end{equation}
where
\begin{equation}
P(n,M;q,a)
:= \#\{\,p \text{ prime} : n-M \le p \le n+M,\ p\equiv a\!\!\pmod q\,\}.
\end{equation}
\end{hypothesis}

\noindent
Hypothesis~\ref{hyp:SIBV-Goldbach} is \emph{not} assumed in this paper.
It is included only to indicate the classical window-size regime in which
short-interval equidistribution is expected to hold.

\subsection{PCGC--Goldbach Bounds Implies Windowed HL Agreement in the SIBV Range}

\begin{lemma}[PCGC--Goldbach Bounds Implies Windowed HL Agreement in the SIBV Range]
\label{lem:PCGC-to-HLwindowed-in-SIBV-range}
Assume Conjecture~\ref{con:PCGC--Goldbach-Bounds}.  Let \(n\) be sufficiently large and
let \(M\) satisfy
\begin{equation}\label{eq:SIBV-range}
(2n)^{\frac12+\frac{\varepsilon}{2}} \ \le\ M \ \le\ n
\end{equation}
for some fixed \(\varepsilon>0\).
Then there exists an explicit envelope \(\eta_{\mathrm{SIBV\text{-}rng}}(2n;M)\to 0\)
(as \(n\to\infty\) with \(M\) in \eqref{eq:SIBV-range}) such that
\begin{equation}\label{eq:HLwindowed-relative}
\left|\frac{\Gmeas(2n;M)}{\GHL(2n;M)}-1\right|
\ \le\ \eta_{\mathrm{SIBV\text{-}rng}}(2n;M).
\end{equation}
In particular,
\begin{equation}\label{eq:HLwindowed-asymp}
\Gmeas(2n;M)
=
\GHL(2n;M)\,\bigl(1+O(\eta_{\mathrm{SIBV\text{-}rng}}(2n;M))\bigr).
\end{equation}
\end{lemma}
\begin{proof}
From Conjecture~\ref{con:PCGC--Goldbach-Bounds},
\begin{equation}
\bigl|\Gmeas(2n;M)-\GHL(2n;M)\bigr|
\le \RHLbound(2n;L),
\qquad L=\sqrt{2\ecap{M}}.
\end{equation}
Assuming \(\GHL(2n;M)>0\) (which holds for all sufficiently large \(n\) in the regimes used
in this paper), divide by \(\GHL(2n;M)\) to obtain
\begin{equation}
\left|\frac{\Gmeas(2n;M)}{\GHL(2n;M)}-1\right|
\le \frac{\RHLbound(2n;L)}{\GHL(2n;M)}.
\end{equation}
The factorization \(\GHL(2n;M)=\GHLproxy(2n;M)\,\HLCorr(2n;M)\)
from \eqref{eq:GHL-factorization}, together with \(\HLCorr(2n;M)\ge 1\)
(Lemma~\ref{lem:HLCorr-min}), to get
\begin{equation}
\frac{\RHLbound(2n;L)}{\GHL(2n;M)}
\le \frac{\RHLbound(2n;L)}{\GHLproxy(2n;M)}.
\end{equation}
Using the definition of \(\RHLbound\) and \(\GHLproxy\),
and the fact that the semiprime correction factors are \(\ge 1\),
this yields the explicit bound
\begin{equation}
\frac{\RHLbound(2n;L)}{\GHLproxy(2n;M)}
\ll \frac{c(2n;L)\,L}{M\,\omega(n)^2}.
\end{equation}
Finally, in this paper \(\omega(n)\asymp 1/\log n\) is taken, so for \(n\) large
(e.g.\ using any explicit \(\omega(n)\ge 1/\log n\) bound in your preferred range),
\begin{equation}
\frac{c(2n;L)\,L}{M\,\omega(n)^2}
\ll \frac{c(2n;L)\,\log^2 n}{\sqrt{M}}.
\end{equation}
In the SIBV range \eqref{eq:SIBV-range}, \(\sqrt{M}\ge (2n)^{\frac14+\frac{\varepsilon}{4}}\),
hence
\begin{equation}
\eta_{\mathrm{SIBV\text{-}rng}}(2n;M)
:= O\!\left(\frac{c(2n;L)\,\log^2 n}{(2n)^{\frac14+\frac{\varepsilon}{4}}}\right)
\to 0
\qquad (n\to\infty).
\end{equation}
This proves \eqref{eq:HLwindowed-relative} and \eqref{eq:HLwindowed-asymp}.
\end{proof}

\section{Conclusion}

This paper has established a collection of conditional reductions from
PCGC--Goldbach to several classical
problems in analytic number theory.  Under the assumption that
PCGC--Goldbach Bounds holds, the following has been shown:

\begin{itemize}
\item \textbf{Goldbach's Conjecture} follows for all even integers \(2n \ge 4\),
with explicit verification required only up to a computable finite threshold
\(n_0 = \xGoldbachN \).  This reduction shows that the geometric bounding framework
provides sufficient control over remainder terms to guarantee the existence of
Goldbach representations.

\item \textbf{Hardy--Littlewood Conjecture A (Goldbach form)} is recovered as an
asymptotic consequence.  In particular, PCGC--Goldbach Bounds implies the standard
Hardy--Littlewood prediction for Goldbach pair counts, demonstrating consistency
between the geometric framework and classical circle--method heuristics.

\item \textbf{Short--Interval Relative Agreement} holds in the
Bombieri--Vinogradov scaling regime \(M \ge (2n)^{\frac12+\varepsilon}\).
This shows that PCGC--Goldbach Bounds provides asymptotic control in window sizes
where classical analytic methods encounter fundamental barriers.
\end{itemize}

Together, these reductions show that PCGC--Goldbach, if validated, provides a
unified geometric foundation for understanding several problems that have
traditionally been approached using distinct analytic techniques.  Unlike
purely asymptotic methods, the geometric framework supplies explicit bounding
structures that translate directly into quantitative results.

\subsection{Significance and Future Directions}

The results presented here indicate that PCGC--Goldbach is not merely a
reformulation of existing conjectures, but a geometric framework from which
classical results emerge as consequences.  The geometric
structure underlying prime pair counting may be more fundamental than the
analytic techniques traditionally used to study these problems.

Several natural directions remain open:

\begin{itemize}
\item \textbf{Very Short Windows.}  PCGC--Goldbach Bounds implies that intervals of
size \(O(\log^4 n / n)\) can still satisfy Goldbach pair conditions.
Analyzing this regime requires highly customized data validation and is
sensitive to the precise choice of weight functions.  A careful treatment new fine--scale 
structure should be integrated.

\item \textbf{Unconditional Validation.}  An unconditional proof of
PCGC--Goldbach would immediately yield unconditional proofs of Goldbach's
conjecture and Hardy--Littlewood asymptotics via the reductions established
here.

\item \textbf{Extensions Beyond Centered Windows.}  Extending the geometric
framework to off--centered windows or residue--class statistics would enable
reductions to full Bombieri--Vinogradov--type hypotheses, rather than relative
agreement results alone.

\item \textbf{Other Additive Prime Problems.}  Generalizing the framework to
twin primes, prime tuples, and related problems could provide a unified
geometric approach to a broader class of questions in additive number theory.
\end{itemize}

Computational validation of PCGC--Goldbach up to \(2n = 23\#\)
\cite{Riemers2026-PCGC-Empirical} provides strong empirical support for the framework.
The reductions presented here show that, such validation extends to all even
integers, so the classical results discussed above as geometric
consequences.

\subsection{Final Remarks}

All results in this paper are explicitly conditional on PCGC--Goldbach.
No unconditional claims are made.  Rather, it has been shown that the geometric
framework, once validated, provides a coherent and powerful foundation for
understanding classical problems in analytic number theory.

The reductions presented here clarify both the scope and the limitations of the
framework: it offers strong quantitative control that implies classical results;
extending it to full distributional hypotheses would require additional
geometric structure beyond the present setting.

The geometric perspective introduced by PCGC--Goldbach represents a shift from
purely analytic methods toward a structural understanding of prime pair
counting.  Whether this perspective ultimately leads to unconditional proofs
remains an open question; note the results here demonstrate that it already
unifies and simplifies a wide range of classical phenomena.

\clearpage
\appendix
\counterwithin{equation}{section}
\numberwithin{table}{section}
\numberwithin{lemma}{section}
\numberwithin{conjecture}{section}
\numberwithin{corollary}{section}
\numberwithin{definition}{section}

\clearpage
\section{Density Function Properties}

The following lemmas collect basic structural properties of \( \HLCorr \). 
While only a subset is used directly in subsequent proofs, the full collection 
characterizes its monotonicity, normalization, and bounding behaviuor, and 
will be useful in future refinements.

\subsection{\(\HLCorr\) Minimum}

\begin{lemma}[\(\HLCorr\) Minimum for Logarithmic Weights]
\label{lem:HLCorr-min}
Assume \(\omega(t)=\kappa/\log t\) for \(t>1\) with \(\kappa>0\).
Let \(n\ge 3\) and \(M\ge 1\), and define
\begin{equation}
\label{eq:HLCorr-def}
\HLCorr(2n;M)
:= \frac{1}{2M}\sum_{m\in\Ipar(n,M)}
\frac{\omega(n-m)\,\omega(n+m)}{\omega(n)^2}.
\end{equation}
Then \(\HLCorr(2n;M)\ge 1\).
If moreover \(0\notin\Ipar(n,M)\), then \(\HLCorr(2n;M)>1\).
\end{lemma}

\begin{proof}
Fix \(m\) with \(0<|m|<n\). Using \(\omega(t)=\kappa/\log t\),
\begin{equation}
\frac{\omega(n-m)\omega(n+m)}{\omega(n)^2}
=
\frac{\log^2 n}{\log(n-m)\,\log(n+m)}.
\end{equation}
Since \((n-m)(n+m)=n^2-m^2<n^2\), it follows that
\begin{equation}
\log(n-m)+\log(n+m)=\log(n^2-m^2) < 2\log n.
\end{equation}
By AM--GM, for positive \(a,b\), \(ab\le \bigl(\tfrac{a+b}{2}\bigr)^2\), hence
\begin{equation}
\log(n-m)\,\log(n+m)
\le \left(\frac{\log(n-m)+\log(n+m)}{2}\right)^2
< \log^2 n.
\end{equation}
Therefore each term satisfies
\begin{equation}
\frac{\omega(n-m)\omega(n+m)}{\omega(n)^2} > 1.
\end{equation}
Averaging over \(2M\) terms in \eqref{eq:HLCorr-def} gives \(\HLCorr(2n;M)>1\) whenever
\(0\notin\Ipar(n,M)\), and in particular \(\HLCorr(2n;M)\ge 1\).
\end{proof}

\begin{remark}[Range of applicability in this paper]
\label{rem:HLCorr-cutoff}
Lemma~\ref{lem:HLCorr-min} is a purely algebraic inequality for the model weight
\(\omega(t)=\kappa/\log t\).  In the present work it is only invoked for \(n\ge 17\),
which is the range in which this logarithmic weight is assumed to be a reliable
proxy for the prime density.
\end{remark}

\subsection{\( \HLCorr \) Normalization Identity}

\begin{lemma}[Normalization Identity for \(\omega\)--Correlation]
\label{lem:hlcorr-norm-identity}
For any \(n\) and window size \(M\),
\begin{equation}
\sum_{m\in\Ipar(n,M)} \omega(n-m)\,\omega(n+m)
=
|\Ipar(n,M)|\,\omega(n)^2\,\HLCorr(2n;M).
\end{equation}
\end{lemma}

\begin{proof}
By definition of \(\HLCorr(2n;M)\),
\begin{equation}
\HLCorr(2n;M)
:= \frac{1}{|\Ipar(n,M)|}
\sum_{m\in\Ipar(n,M)}
\frac{\omega(n-m)\,\omega(n+m)}{\omega(n)^2}.
\end{equation}
Multiplying both sides by \(|\Ipar(n,M)|\,\omega(n)^2\) yields
\begin{equation}
|\Ipar(n,M)|\,\omega(n)^2\,\HLCorr(2n;M)
= \sum_{m\in\Ipar(n,M)} \omega(n-m)\,\omega(n+m),
\end{equation}
which proves the claim.
\end{proof}

\subsection{\( \HLCorr \) Lower Bound}

\begin{corollary}[Crude Lower Bound for the \(\omega\)--Correlation Sum]
\label{cor:omega-corr-lower}
Assume \(\omega(n) := \pi(n)/n\).  
Then for all \(n\ge 17\),
\begin{equation}
\sum_{m\in\Ipar(n,M)} \omega(n-m)\,\omega(n+m)
\;\ge\; \frac{|\Ipar(n,M)|}{\log^2 n}\,\HLCorr(2n;M).
\end{equation}
\end{corollary}

\begin{proof}
By Lemma~\ref{lem:hlcorr-norm-identity},
\begin{equation}
\sum_{m\in\Ipar(n,M)} \omega(n-m)\,\omega(n+m)
= |\Ipar(n,M)|\,\omega(n)^2\,\HLCorr(2n;M).
\end{equation}
For \(n\ge 17\), Rosser and Schoenfeld \cite{RosserSchoenfeld1962}
prove the explicit bound
\begin{equation}
\pi(n) > \frac{n}{\log n},
\end{equation}
which implies
\begin{equation}
\omega(n)=\frac{\pi(n)}{n}>\frac{1}{\log n}.
\end{equation}
Squaring and substituting into the previous identity gives
\begin{equation}
|\Ipar(n,M)|\,\omega(n)^2\,\HLCorr(2n;M)
\;\ge\;
\frac{|\Ipar(n,M)|}{\log^2 n}\,\HLCorr(2n;M),
\end{equation}
as claimed.
\end{proof}

\subsection{\( \HLCorr \) Envelope}

\begin{lemma}[\(\HLCorr\) Endpoint Envelope for Fixed \(\alpha\)]
\label{lem:HLCorr-upper-envelope}
Assume \(\omega(t)\propto 1/\log t\) and so that
\begin{equation}
\HLCorr(2n;M)
=\frac{1}{|\Ipar(n,m)|}\sum_{m\in\Ipar(n,M)} 
\frac{\log^2 n}{\log(n-m)\log(n+m)}.
\end{equation}
Let \(0<\alpha<1\) and set \(M:=\lfloor \alpha n\rfloor\), with \( |\Ipar(n,M)| = 2M \).  Then
\begin{equation}
\label{eq:HLCorr-upper-envelope}
\HLCorr(2n;M)\ \le\ U_\alpha(n)
:=\frac{\log^2 n}{\log(n-M)\,\log(n+M)}.
\end{equation}
Moreover, \(U_\alpha(n)\to 1\) as \(n\to\infty\), and \(U_\alpha(n)\) is strictly decreasing
for all sufficiently large \(n\).
\end{lemma}

\begin{proof}
For fixed \(n\) and \(m\in[1,M]\), it follows that \(n-m\ge n-M\) and \(n+m\le n+M\).
Since \(\log\) is increasing on \((1,\infty)\),
\begin{equation}
\log(n-m)\ge \log(n-M)
\quad\text{and}\quad
\log(n+m)\le \log(n+M).
\end{equation}
Hence
\begin{equation}
\log(n-m)\log(n+m)\ \ge\ \log(n-M)\log(n+M),
\end{equation}
so each summand satisfies
\begin{equation}
\frac{\log^2 n}{\log(n-m)\log(n+m)}
\ \le\
\frac{\log^2 n}{\log(n-M)\log(n+M)}=U_\alpha(n).
\end{equation}
Averaging over \(m\in\Ipar(n,M)\) preserves the inequality, proving
\eqref{eq:HLCorr-upper-envelope}.
Finally, since \(M=\lfloor \alpha n\rfloor=(\alpha+o(1))n\), it follows that
\begin{equation}
\log(n\pm M)=\log n + \log(1\pm \alpha) + o(1),
\end{equation}
which implies \(U_\alpha(n)\to 1\).  Eventual monotone decrease of the explicit
function \(U_\alpha(n)\) follows by elementary calculus.
\end{proof}

\subsection{\( \HLCorr \) Limit}

\begin{lemma}[\(\HLCorr\) Approaches \(1\) at Logarithmic Rate for Fixed \(\alpha\)]
\label{lem:HLCorr-rate-fixed-alpha}
Assume \(\omega(t) \propto \frac{1}{\log{t}} \) for \(t>1\).
Fix \(0<\alpha<1\) and set \(M:=\lfloor \alpha n\rfloor\).
Then for all sufficiently large \(n\),
\begin{equation}
\label{eq:HLCorr-rate-fixed-alpha}
0 \ \le\ \HLCorr(2n;M)-1
\ \le\ U_\alpha(n)-1
\ =\ O_\alpha\!\Bigl(\frac{1}{\log n}\Bigr),
\qquad (n\to\infty),
\end{equation}
where \(U_\alpha(n)\) is the endpoint envelope from Lemma~\ref{lem:HLCorr-upper-envelope}.
In particular, \(\HLCorr(2n;M)\to 1\) as \(n\to\infty\) with \(M=\lfloor \alpha n\rfloor\).
\end{lemma}

\begin{proof}
By Lemma~\ref{lem:HLCorr-min}, \(\HLCorr(2n;M)\ge 1\), hence
\(\HLCorr(2n;M)-1\ge 0\).

By Lemma~\ref{lem:HLCorr-upper-envelope}, for \(M=\lfloor \alpha n\rfloor\),
\begin{equation}
\label{eq:HLCorr-upper-envelope-recall}
\HLCorr(2n;M)\ \le\ U_\alpha(n)
:=\frac{\log^2 n}{\log(n-M)\,\log(n+M)}.
\end{equation}
Subtracting \(1\) yields
\begin{equation}
0\le \HLCorr(2n;M)-1 \le U_\alpha(n)-1.
\end{equation}

It remains to bound \(U_\alpha(n)-1\).
Since \(M=\lfloor \alpha n\rfloor\), it follows that \(M/n=\alpha+O(1/n)\), so
\begin{equation}
\log(n\pm M)
=\log n + \log\!\Bigl(1\pm \frac{M}{n}\Bigr)
=\log n + \log(1\pm \alpha) + O\!\Bigl(\frac{1}{n}\Bigr).
\end{equation}
Let \(A_\alpha:=\log(1-\alpha)\) and \(B_\alpha:=\log(1+\alpha)\), which are constants
depending only on \(\alpha\) (with \(A_\alpha<0<B_\alpha\)).
Then
\begin{equation}
\log(n-M)\,\log(n+M)
=\bigl(\log n + A_\alpha + O(1/n)\bigr)\bigl(\log n + B_\alpha + O(1/n)\bigr)
=\log^2 n + (A_\alpha+B_\alpha)\log n + O_\alpha(1).
\end{equation}
Therefore,
\begin{equation}
U_\alpha(n)
=\frac{\log^2 n}{\log^2 n + (A_\alpha+B_\alpha)\log n + O_\alpha(1)}
=\frac{1}{1 + \frac{A_\alpha+B_\alpha}{\log n} + O_\alpha\!\bigl(\frac{1}{\log^2 n}\bigr)}.
\end{equation}
Using \( \frac{1}{1+z}=1+O(z)\) for \(z\to 0\) gives
\begin{equation}
U_\alpha(n)=1+O_\alpha\!\Bigl(\frac{1}{\log n}\Bigr),
\end{equation}
hence \(U_\alpha(n)-1=O_\alpha(1/\log n)\). Combining with the earlier inequality
proves \eqref{eq:HLCorr-rate-fixed-alpha}.
\end{proof}

\subsection{\( \GHLproxy \) Asymptotic Exactness}

\begin{lemma}[HL Window Proxy is Asymptotically Exact]
\label{lem:GHL-proxy-trivial}
Let \(n\in\mathbb{N}\) and \(M\) satisfy \(1\le M<n\).  Assume \(\Ipar(n;M)\) is as in
Definition~\ref{def:parity-admissible} and take \(\omega(t) \propto \frac{1}{\log{t}}\).
Let \(\GHL(2n;M)\), \(\GHLproxy(2n;M)\), and \(\HLCorr(2n;M)\) be as in
Definitions~\ref{def:GHL}--\ref{def:GHLproxy}.

The following is defined:
\begin{equation}
\etaHL(2n;M):=\HLCorr(2n;M)-1.
\end{equation}
Then
\begin{equation}
\label{eq:GHL-proxy-eta}
\GHL(2n;M)=\GHLproxy(2n;M)\bigl(1+\etaHL(2n;M)\bigr).
\end{equation}
Moreover, along any admissible sequence \((n,M)\) with \(M\to\infty\),
\begin{equation}
\etaHL(2n;M)\to 0.
\end{equation}
Quantitatively,
\begin{equation}
\etaHL(2n;M)=O\!\Bigl(\frac{1}{\log{n}}\Bigr),
\qquad (n\to\infty),
\end{equation}
uniformly for admissible \(M<n\).  If in addition \(M=o(n)\), then
\begin{equation}
\label{eq:GHL-eta-smallM}
\etaHL(2n;M)
=
O\!\Bigl(\frac{M^2}{n^2\log{n}}\Bigr)
+O\!\Bigl(\frac{1}{\log^2{n}}\Bigr).
\end{equation}
\end{lemma}

\begin{proof}
By \eqref{eq:GHL-factorization},
\begin{equation}
\frac{\GHL(2n;M)}{\GHLproxy(2n;M)}=\HLCorr(2n;M),
\end{equation}
so \eqref{eq:GHL-proxy-eta} holds with \(\etaHL(2n;M):=\HLCorr(2n;M)-1\).
For fixed \(0<\alpha<1\) and \(M=\lfloor \alpha n\rfloor\), Lemmas~\ref{lem:HLCorr-min}
and \ref{lem:HLCorr-upper-envelope} give
\(0\le \etaHL(2n;M)\le U_\alpha(n)-1\),
and Lemma~\ref{lem:HLCorr-rate-fixed-alpha} yields
\(\etaHL(2n;M)=O_\alpha(\frac{1}{\log{n}})\to 0\).
\end{proof}

\clearpage
\section{Tables of Calculated Constants}

\noindent
All numerical constants in the following tables were computed using the
standard Unix \texttt{bc -l} arbitrary-precision calculator.
No custom programs or numerical libraries were used.
Values were independently spot-checked against a handheld scientific calculator.
Python scripts confirm the results.

\begin{table}[htbp]
\centering
\begin{tabular}{c r r}
\toprule
\(p\) &
\(\EffLocMod_p^{(3)}\) &
\(\EffLocMod_p^{(5)}\) \\
\midrule
\(2\)  & \(1\)  & \(1\) \\
\(3\)  & \(1\cdot(3-1) = \xEffLocModDivA\) & \(\xEffLocModNDivA\) \\
\(5\)  & \(\xEffLocModDivA\cdot(5-1) = \xEffLocModDivB\) & \(\xEffLocModNDivA\cdot(5-1) = \xEffLocModNDivB\) \\
\(7\)  & \(\xEffLocModDivB\cdot(7-1) = \xEffLocModDivC\) & \(\xEffLocModNDivB\cdot(7-1) = \xEffLocModNDivC\) \\
\(11\) & \(\xEffLocModDivC\cdot(11-1) = \xEffLocModDivD\) & \(\xEffLocModNDivC\cdot(11-1) = \xEffLocModNDivD\) \\
\(13\) & \(\xEffLocModDivD\cdot(13-1) = \xEffLocModDivE\) & \(\xEffLocModNDivD\cdot(13-1) = \xEffLocModNDivE\) \\
\(17\) & \(\xEffLocModDivE\cdot(17-1) = \xEffLocModDivF\) & \(\xEffLocModNDivE\cdot(17-1) = \xEffLocModNDivF\) \\
\(19\) & \(\xEffLocModDivF\cdot(19-1) = \xEffLocModDivG\) & \(\xEffLocModNDivF\cdot(19-1) = \xEffLocModNDivG\) \\
\(23\) & \(\xEffLocModDivG\cdot(23-1) = \xEffLocModDivH\) & \(\xEffLocModNDivG\cdot(23-1) = \xEffLocModNDivH\) \\
\(29\) & \(\xEffLocModDivH\cdot(29-1) = \xEffLocModDivI\) & \(\xEffLocModNDivH\cdot(29-1) = \xEffLocModNDivI\) \\
\(31\) & \(\xEffLocModDivI\cdot(29-1) = \xEffLocModDivJ\) & \(\xEffLocModNDivI\cdot(29-1) = \xEffLocModNDivJ\) \\
\(37\) & \(\xEffLocModDivJ\cdot(29-1) = \xEffLocModDivK\) & \(\xEffLocModNDivJ\cdot(29-1) = \xEffLocModNDivK\) \\
\(41\) & \(\xEffLocModDivK\cdot(29-1) = \xEffLocModDivL\) & \(\xEffLocModNDivK\cdot(29-1) = \xEffLocModNDivL\) \\
\bottomrule
\end{tabular}
\caption{Effective local moduli for odd primes \(p \le 41\).}
\label{tab:EffLocMod-values}
\end{table}

\begin{table}[htbp]
\centering
\begin{tabular}{r r r r}
\toprule
\multicolumn{2}{c}{\(n \nmid 3\)} &
\multicolumn{2}{c}{\(n \mid 3\)} \\
\cmidrule(lr){1-2} \cmidrule(lr){3-4}
\(\EffLocMod(2n;L)\) & \(\OmegaPrimeNorm(2n;L)\) &
\(\EffLocMod(2n;L)\) & \(\OmegaPrimeNorm(2n;L)\) \\
\midrule
\(\xEffLocModNDivA\)   & \(\xOmegaPrimeNormA\) &
\(\xEffLocModDivA\)  & \(\xOmegaPrimeNormASqrt\) \\
\(\xEffLocModNDivB\)  & \(\xOmegaPrimeNormB\) &
\(\xEffLocModDivB\)  & \(\xOmegaPrimeNormBSqrt\) \\
\(\xEffLocModNDivC\)  & \(\xOmegaPrimeNormC\) &
\(\xEffLocModDivC\)  & \(\xOmegaPrimeNormCSqrt\) \\
\(\xEffLocModNDivD\) & \(\xOmegaPrimeNormD\) &
\(\xEffLocModDivD\) & \(\xOmegaPrimeNormDSqrt\) \\
\(\xEffLocModNDivE\) & \(\xOmegaPrimeNormE\) &
\(\xEffLocModDivE\) & \(\xOmegaPrimeNormESqrt\) \\
\(\xEffLocModNDivF\) & \(\xOmegaPrimeNormF\) &
\(\xEffLocModDivF\) & \(\xOmegaPrimeNormFSqrt\) \\
\(\xEffLocModNDivG\) & \(\xOmegaPrimeNormG\) &
\(\xEffLocModDivG\) & \(\xOmegaPrimeNormGSqrt\) \\
\(\xEffLocModNDivH\) & \(\xOmegaPrimeNormH\) &
\(\xEffLocModDivH\) & \(\xOmegaPrimeNormHSqrt\) \\
\(\xEffLocModNDivI\) & \(\xOmegaPrimeNormI\) &
\(\xEffLocModDivI\) & \(\xOmegaPrimeNormISqrt\) \\
\(\xEffLocModNDivJ\) & \(\xOmegaPrimeNormJ\) &
\(\xEffLocModDivJ\) & \(\xOmegaPrimeNormJSqrt\) \\
\(\xEffLocModNDivK\) & \(\xOmegaPrimeNormK\) &
\(\xEffLocModDivK\) & \(\xOmegaPrimeNormKSqrt\) \\
%\(\xEffLocModNDivL\) & \(\xOmegaPrimeNormL\) &
%\(\xEffLocModDivL\) & \(\xOmegaPrimeNormLSqrt\) \\
\bottomrule
\end{tabular}
\caption{Normalized prime curvature constants \(\OmegaPrimeNorm(2n;L)\) as functions of the effective local modulus.}
\label{tab:OmegaPrimeNorm-values}
\end{table}


\begin{table}[htbp]
\centering
\begin{tabular}{r r r}
\toprule
\(\substack{3\nmid n \\ \EffLocMod(2n;L)}\) &
\(\substack{3\mid n \\ \EffLocMod(2n;L)}\) & \(c(2n;L)\) \\
\midrule
\(\xEffLocModNDivA\) & \(\xEffLocModDivA\)  & \(\xCvalueA\) \\
\(\xEffLocModNDivB\) & \(\xEffLocModDivC\)  & \(\xCvalueB\) \\
\(\xEffLocModNDivC\) & \(\xEffLocModDivC\) & \(\xCvalueC\) \\
\(\xEffLocModNDivD\) & \(\xEffLocModDivD\) & \(\xCvalueD\) \\
\(\xEffLocModNDivE\) & \(\xEffLocModDivE\) & \(\xCvalueE\) \\
\(\xEffLocModNDivF\) & \(\xEffLocModDivF\) & \(\xCvalueF\) \\
\(\xEffLocModNDivG\) & \(\xEffLocModDivG\) & \(\xCvalueG\) \\
\(\xEffLocModNDivH\) & \(\xEffLocModDivH\) & \(\xCvalueH\) \\
\(\xEffLocModNDivI\) & \(\xEffLocModDivI\) & \(\xCvalueI\) \\
\(\xEffLocModNDivJ\) & \(\xEffLocModDivJ\) & \(\xCvalueJ\) \\
\(\xEffLocModNDivK\) & \(\xEffLocModDivK\) & \(\xCvalueK\) \\
\(\xEffLocModNDivL\) & \(\xEffLocModDivL\) & \(\xCvalueL\) \\
\bottomrule
\end{tabular}
\caption{Bounding envelope constants \(c(2n;L)\) associated with effective local moduli.}
\label{tab:c-values}
\end{table}

\begin{table}[htbp]
\centering
\centering
\begin{tabular}{r r r r}
\toprule
\multicolumn{2}{c}{\(n \nmid 3\)} &
\multicolumn{2}{c}{\(n \mid 3\)} \\
\cmidrule(lr){1-2} \cmidrule(lr){3-4}
\(\EffLocMod(2n;L)\) & \(c(2n;L)\;/\;\SsemHead^{\EffLocModCap,\bullet}(2n;L) \) &
\(\EffLocMod(2n;L)\) & \(c(2n;L)\;/\;\SsemHead^{\EffLocModCap,\bullet}(2n;L) \) \\
\midrule
\(\xEffLocModNDivA\)        & \(\xCratioNDivA\) &
\(\xEffLocModDivA\)        & \(\xCratioDivA\) \\
\(\xEffLocModNDivB\)        & \(\xCratioNDivB\) &
\(\xEffLocModDivB\)        & \(\xCratioDivB\) \\
\(\xEffLocModNDivC\)       & \(\xCratioNDivC\) &
\(\xEffLocModDivC\)       & \(\xCratioDivC\) \\
\(\xEffLocModNDivD\)      & \(\xCratioNDivD\) &
\(\xEffLocModDivD\)      & \(\xCratioDivD\) \\
\(\xEffLocModNDivE\)     & \(\xCratioNDivE\) &
\(\xEffLocModDivE\)     & \(\xCratioDivE\) \\
\(\xEffLocModNDivF\)    & \(\xCratioNDivF\) &
\(\xEffLocModDivF\)    & \(\xCratioDivF\) \\
\(\xEffLocModNDivG\)   & \(\xCratioNDivG\) &
\(\xEffLocModDivG\)  & \(\xCratioDivG\) \\
\(\xEffLocModNDivH\) & \(\xCratioNDivH\) &
\(\xEffLocModDivH\) & \(\xCratioDivH\) \\
\(\xEffLocModNDivI\)  & \(\xCratioNDivI\) &
\(\xEffLocModDivI\) & \(\xCratioDivI\) \\
\(\xEffLocModNDivJ\)  & \(\xCratioNDivJ\) &
\(\xEffLocModDivJ\) & \(\xCratioDivJ\) \\
\(\xEffLocModNDivK\)  & \(\xCratioNDivK\) &
\(\xEffLocModDivK\) & \(\xCratioDivK\) \\
\(\xEffLocModNDivL\)  & \(\xCratioNDivL\) &
\(\xEffLocModDivL\) & \(\xCratioDivL\) \\
\bottomrule
\end{tabular}
\caption{Bounding envelope constant ratio \(c(2n;L)\;/\;\SsemHead^{\EffLocModCap,\bullet}(2n;L)\) associated with effective local moduli.}
\label{tab:c-ratios}
\end{table}

\clearpage
\bibliographystyle{plain}
\bibliography{pcgc-goldbach-reductions}

\end{document}

